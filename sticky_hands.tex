\documentclass[12pt,a4paper]{article}
\usepackage[utf8]{inputenc}
\usepackage[T1]{fontenc}
\usepackage[magyar]{babel}
\usepackage[outer=25mm,inner=25mm,top=25mm,bottom=25mm]{geometry}

\linespread{1.5}
\frenchspacing

\usepackage{amsmath}
\usepackage{graphicx}

\usepackage{indentfirst}
\usepackage{colortbl}
\usepackage[capposition=bottom]{floatrow}
%\usepackage{natbib}
\usepackage[nottoc]{tocbibind}
%sajat
\usepackage{hhline}
\usepackage{colortbl}

%HIVATKOZAS
\usepackage[nottoc]{tocbibind}
\usepackage{natbib}
\setcitestyle{agms}

% HIVATKOZASOK SZINE
\usepackage[colorlinks=true]{hyperref}
\definecolor{oxfordblue}{rgb}{0.0, 0.13, 0.28}
\hypersetup{
     colorlinks=true,
     linkcolor=black,
     filecolor=blue,
     citecolor = oxfordblue,      
     urlcolor=cyan,
     }
\pagebreak

% MATEMATIKAI
\usepackage{amssymb}

%sajat+
%\setlength\parskip{1\baselineskip}

\usepackage{pgf}

\begin{document}

% \setcounter{page}{1}
% \setcounter{tocdepth}{3}

\begin{figure}[h]
\includegraphics[width=50mm]{figures/rajk_green.png}
\end{figure}

\noindent\rule{\textwidth}{1pt}
\vspace{1mm}
\begin{center}
{\Huge \textbf{Ragadós kezek}}

{\large A pornográfia és a szexuális viselkedés\\
kapcsolatának vizsgálata}
\end{center}
\noindent\rule{\textwidth}{1pt}

\begin{center}
{\large Készítette: Pap Sebestyén}\\
{\large Kurzus: Szexuálpszichológia - 2021 ősz}\\
{\large Kurzustartó: Váradi Fanni}
\end{center}

\vspace{10mm}

\noindent \textbf{Absztrakt:} A pornográfia egyre inkább meghatározza az internet használati szokásaink, melynek hatása így fontosabb kérdéssé válik. Ez a kutatás azt vizsgálja, hogy milyen kapcsolat található az amerikai társadalomban a PornHub weboldalon található pornográf tartalmak nézettségi adatai, valamint a megfigyelhető szexuális viselkedés között. Az eredmények rávilágítanak, hogy a pornográf videók nézettsége fordított kapcsolatban van az óvszerhasználattal, valamint a hardcore videók nézettségének aránya összefügg a munkahelyi szexuális zaklatásokkal. További eredményként a tanulmány bemutatja, hogy a pornóipar nem követi a társadalomban megfigyelhető szexuális orientációk arányát. Az elemzéshez a dolgozat egy becslést alkalmaz a nézettségre, valamint egyszerű vizualizációs technikákat használ a kapcsolatok feltárására. A vizsgált témák további egyén szintű elemzésével számszerűsíthetőek a kimutatott kapcsolatok.

\thispagestyle{empty}

\pagebreak

\tableofcontents

\thispagestyle{empty}

\pagebreak
\setcounter{page}{1}
% \setcounter{tocdepth}{3}
\listoffigures
%\listoftables

%\thispagestyle{empty}

\pagebreak

\setcounter{tocdepth}{2}
% \setcounter{page}{1}

\section{Bevezetés}

A 21. században a föld lakosságának jelentős része rendelkezik már internet kapcsolattal. Az ezen keresztül tapasztalható információ áramlásnak azonban jelentős hányada a pornográf tartalmak fogyasztására irányul. Ez a nagy mértékű erotikus tartalom fogyasztás egyértelműen befolyásoló erővel bír az egyénekre nézve. Ugyanakkor nem egyértelmű, hogy a pornográf tartalmak határozzák-e meg a valós attitűdöket, vagy pedig a kapcsolat fordított irányú és a pornóipar a szexuális attitűdöket követi.

Ebben a dolgozatban a PornHub oldalon tapasztalható amerikai fogyasztási adatok és a társadalomban megfigyelhető jelenségek közötti kapcsolatokat fogom vizsgálni. Kutatási kérdésem az, hogy milyen folyamatokra van hatással a pornográf tartalmak fogyasztása, és ezek a hatásmechanizmusok milyen irányt mutatnak.

A dolgozat arra a következtetésre jut, hogy a pornográf videók erősen összefüggenek az óvszerhasználattal, valamint a szexuális zaklatással is. Ezen túlmenően azonban a nézettség nem követi a lakosság valódi tulajdonságait, így például a homoszexualitás arányát sem.

A kutatási kérdés megválaszolásához a tanulmány egy becslést végez az éves szinten tapasztalható nézettségi adatokra vonatkozóan, majd azokat felhasználva vizualizációs technikák segítségével hasonlítja össze a pornográf tartalom fogyasztását a társadalmi jellemzőkkel.

A dolgozat első fejezete bemutatja röviden a pornográfia különböző definícióit és történetét, majd pedig a második fejezetben kitér a pornográfia egyénekre gyakorolt hatására is. Ezen hatások bemutatásánál a dolgozat elkülöníti a két nemnél megfigyelhető jelenségeket, valamint azokat is, amelyek közösek. A tanulmány harmadik fejezet bemutatja a nézettségi adatok becslésének lépéseit, majd pedig ezeket felhasználva elemzi azok társadalomra gyakorolt hatását. 

Az utolsó összegző fejezetben a dolgozat értékeli az eredményeket is rávilágít azok fontosságára is. Emellett kiemeli az elemzés limitációit is, amelyek további kutatások szükségességére engednek következtetni.

Az adatokon elvégzett műveleteket és az elemzést a Python programnyelv segítségével végeztem el.\footnote{Az elemzéshez felhasznált adatok és scriptek elérhetőek a következő linken https://github.com/papsebestyen/porn-influence-analysis.git}

\section{A pornográfia fogalma és története}

A 19. században a fényképezés, majd a mozgókép találmányait gyorsan kamatoztatták a pornográfia készítésében. A pornográf filmek legkésőbb az 1920-as években széles körben elérhetővé váltak, és az 1960-as években népszerűségük hatalmas felfutásnak indult. A videokazetták kifejlesztése az 1980-as években, a DVD-k elterjedése pedig az 1990-es években lehetővé tette a pornográf filmek széles körű elterjedését. A pornográf képek és filmek még szélesebb körben elérhetővé váltak az internet megjelenésével az 1990-es években. A pornóipar az egyik legjövedelmezőbb iparág lett az interneten. \citep{jenkins2021_pornograpy}

Amellett, hogy hatalmas piacteret biztosít a sokféle ízlésnek a kereskedelmi pornográfia számára, az internet sok amatőrt is arra ösztönzött, hogy tegyen közzé képeket magáról, amelyek gyakran megkérdőjelezték a szépségről és a szexuális vonzerőről alkotott hagyományos elképzeléseket. A webkamerák használata még tovább nyitotta az ipart az amatőrök előtt, lehetővé téve az egyének számára, hogy élőben ábrázolják magukat, gyakran térítés ellenében. \citep{jenkins2021_pornograpy}

A pornográfiában jellemzően heteroszexuális tartalmak keletkeznek, annak ellenére, hogy keresleti oldalon ugyanúgy igény mutatkozik a homoszexuális tartalmakra is. Az iparágban főként férfi heteroszexuális színészek dolgoznak, így elég nagy a kereslet a meleg pornószínészekre. Ebből fakadóan megjelent a \textit{fizetett meleg} (\textit{gay-for-pay}) fogalma is, ami olyan színészeket takar, akik heteroszexuálisok, de pénzért homoszexuális szerepeket vállalnak \citep{kiss2019understanding}. Ez nem is meglepő, hiszen a homoszexuális tartalmakért magasabb fizetést kaphat egy pornószínész \citep{escoffier2003gay}.

A pornográf tartalmak vizsgálatakor fontos a fogalom alapos tisztázása, hisz annak számtalan definíciója létezik. A Pornográfiai Bizottság \citeyearpar[228–229. o.]{united1986attorney} szerint a pornográf tartalom elsősorban szexuális jellegű, és szexuális izgalom céljára készült. Ugyanakkor az internet és a média elérhetőségének növekedésével a pornográf tartalmak definíciója is változott. Például Barron és Kimmel \citeyearpar{barron2000sexual} a pornográfiát úgy definiálja, mint minden olyan szexuális tartalmú anyagot, amely felnőtt használatra van korlátozva. Fontos belátni azonban, hogy ezen korlátok betartatása nem feltétlenül tud megvalósulni az online pornográfia világában.

A pornográfia meghatározásába mások a pornográf tartalmak konkrét megvalósulását is beleveszik. Így például Russel \citeyearpar{russell1993against} úgy határozza meg a pornográfiát, mint olyan médiatartalmat, amely a szexet és a nemiszervek ábrázolását bántalmazással vagy lealacsonyítással kombinálja oly módon, hogy az támogatja, elnézi vagy akár ösztönzi is az ilyen típusú viselkedést. Senn \citeyearpar{senn1993women} pedig megkülönbözteti a pornográfiát az erotikától. Véleménye szerint az erotika szexuális jellegű, de nem lealacsonyító vagy erőszakos, míg a pornográfia a szexualitást összeköti a szexizmussal, rasszizmussal, homofóbiával és erőszakkal is.

\section{A pornográfia hatása}

A pornográfia már széles körben elterjedt és ez a folyamat egyre inkább csak növekszik. Ugyanakkor ezzel párhuzamosan megjelentek azok a vélemények is, amelyek aggodalmukat fejezik ki a termék fogyasztásának hatásaival kapcsolatban. Ez a fajta nyugtalanság azonban nem csak a pornográf, de a nyíltan szexuális tartalmú médiatermékekre is átterjedt, így a televízió, a női és férfi magazinok, valamint az internet mind részét képezik az ezzel kapcsolatos kutatásoknak, melyek szintén számos negatív hatással lehetnek a szexualitásra. \citep{attwood2005people}

Az emberek szexuális viselkedését leginkább az határozza meg, hogy milyen szexuális forgatókönyvekkel találkoznak. A szexuális forgatókönyvek olyan társadalmilag felépített ötletek vagy iránymutatások, amelyek azzal foglalkoznak, hogy hogyan bontakozik ki a szex és milyen következményekkel járhat. Egy szexuális médiaszocializáció modell, a szexuális forgatókönyv-szerzési, aktiválási, alkalmazási modellje ($_3$AM) azt sugallja, hogy a pornográfiának való nagyobb kitettség bizonyos szexuális viselkedések nagyobb valószínűségét eredményezheti, beleértve a kockázattal járó szexuális viselkedéseket is. \citep{gagnon2017sexual, wright2011mass}

Habár számos eszköz rendelkezésre áll a biztonságos együttléthez, de a legkönnyebben elérhető és leggyakrabban használt az óvszer alkalmazása. Ugyanakkor a pornográf tartalmakban gyakran előforduló motívum az, hogy az aktusban résztvevő felek nem gondoskodnak a biztonságos szexről, valamint az esetleges terhesség megelőzéséről, vagy ezek jelenlétére semmi sem utal.  Ez komoly problémát jelenthet, hisz a védekezés hiányának nem csak a tinédzser szülőség lehet a következménye, hanem a szexuális úton terjedő betegségek terjedését is magával vonzhatja. Ezt támasztja alá Wright és társainak \citeyearpar{wright2020adolescent} kutatása, melyben azt figyelték meg, hogy nagyobb eséllyel történik együttlét óvszer nélkül azoknál a fiataloknál, akikkel a szüleik keveset vagy egyáltalán nem beszélgettek a szexuális egészségről.

A pornográfiában megjelenő motívumok sok esetben az egyének valódi szexuális életébe is beleolvadhatnak. Herbenick és társai \citeyearpar{herbenick2020diverse} arra mutatnak rá, hogy a pornográf tartalom fogyasztása mindkét nemnél növeli az olyan agresszív és megalázó attitűdök előfordulását, mint a fojtogatás, a degradáló nevek használata vagy a nem kívánt tevékenységek kikényszerítése. A jelenségre különös figyelmet kell fordítani, hiszen növeli a szexuális együttlétek kockázatát, valamint hatással lehet a szexuális kielégülés folyamataira is.

A pornográf tartalmak fogyasztása habár egyértelmű hatással van az egyénre, ugyanakkor fontos belátni, hogy ez különböző mértékeket és formákat ölthet a férfiak és a nők esetén. Ez nem is meglepő, hiszen mind az online, mind pedig az offline pornográf tartalmak alapvetően a férfi fogyasztókat célozzák \citep{boies2002university}.

A pornográf tartalmak egy jelentős hányadában olyan jelenetek is szerepelnek, amelyekben a férfiak erőszakosan és megalázóan viselkednek nőkkel. Ezek a tartalmak ugyanakkor jelentős változásokat okozhatnak a férfiak szexualitásában és összefügghetnek a szexuális erőszak és agresszió előfordulásával. Malamuth és társai \citeyearpar{malamuth2000pornography} például arra mutattak rá, hogy az erőszakos pornográf tartalmak fogyasztása erősen összefügg a szexuális agresszióval. Tanulmányuk szerint az agresszívabb férfiak szexualitására sokkal károsabb hatással van a pornográf tartalmak ezen műfaja, mint azokra a férfiakra, akik általánosságban kevésbé agresszívek.

Habár a pornográf tartalom fogyasztás nőkre gyakorolt hatásánál vizsgálata sokáig nem volt kutatott téma \citep{weinberg2010pornography}, de számos tanulmány igazolja, hogy nők fogyasztanak erotikus tartalmakat \citep{sabina2008nature}.

A nők esetén kevésbé jellemző az agresszív és káros viselkedésminták átvétele a pornográfiából \citep{bridges2016sexual}, de hatással lehet a szexuális aktivitásukra és annak formájára. Wright \citeyearpar{wright2013internet} tanulmánya például arra mutatott rá, hogy a pornográf tartalmak fogyasztása növeli a házasságon kívüli szexuális együttlétek számát.

\section{Adatok}

A pornográf tartalmakra vonatkozó adatot a pornhub.com oldalról gyűjtöttem be a Python programnyelv segítségével \citep{pornhub2021}.\footnote{A letöltést segítő kódrészletek a következő linken érhetők el: https://github.com/papsebestyen/pornstar\_data} A PornHub weboldal használata azért volt indokolt, mivel az az online pornográfia legnagyobb szereplője az online pornográfia megjelenésétől kezdve.

A letöltött adat közel 3,5 millió videóra vonatkozó információt tartalmazott a 2007 és 2021 között. Ahogy az \ref{video.all.series}. ábra is mutatja, az idő során a PornHub oldalára egyre több és több tartalom került fel, melyek mindegyike ingyenesen elérhető az interneten. Az diagram alapján az is megfigyelhető, hogy a nézettség is folyamatosan egyre nagyobb lett az időben. Fontos figyelembe venni, hogy nézettség a weboldal 2021-es időpontjában lett mérve, tehát minden megtekintést összegez eddig az időpontig. A későbbiekben ennek feloldásaként a dolgozat részletezi az egyes évekhez tartozó nézettsége kiszámítását is.

\begin{figure}[h]
    \caption[A pornográf videók száma és nézettsége 2021-ben]{\footnotesize{A PornHub weboldalon megjelenő videók száma és nézettsége 2021-ben. Forrás: saját ábra}}
    \label{video.all.series}
    \begin{center}
        %% Creator: Matplotlib, PGF backend
%%
%% To include the figure in your LaTeX document, write
%%   \input{<filename>.pgf}
%%
%% Make sure the required packages are loaded in your preamble
%%   \usepackage{pgf}
%%
%% Figures using additional raster images can only be included by \input if
%% they are in the same directory as the main LaTeX file. For loading figures
%% from other directories you can use the `import` package
%%   \usepackage{import}
%%
%% and then include the figures with
%%   \import{<path to file>}{<filename>.pgf}
%%
%% Matplotlib used the following preamble
%%
\begingroup%
\makeatletter%
\begin{pgfpicture}%
\pgfpathrectangle{\pgfpointorigin}{\pgfqpoint{6.024979in}{2.086271in}}%
\pgfusepath{use as bounding box, clip}%
\begin{pgfscope}%
\pgfsetbuttcap%
\pgfsetmiterjoin%
\pgfsetlinewidth{0.000000pt}%
\definecolor{currentstroke}{rgb}{1.000000,1.000000,1.000000}%
\pgfsetstrokecolor{currentstroke}%
\pgfsetstrokeopacity{0.000000}%
\pgfsetdash{}{0pt}%
\pgfpathmoveto{\pgfqpoint{0.000000in}{0.000000in}}%
\pgfpathlineto{\pgfqpoint{6.024979in}{0.000000in}}%
\pgfpathlineto{\pgfqpoint{6.024979in}{2.086271in}}%
\pgfpathlineto{\pgfqpoint{0.000000in}{2.086271in}}%
\pgfpathclose%
\pgfusepath{}%
\end{pgfscope}%
\begin{pgfscope}%
\pgfsetbuttcap%
\pgfsetmiterjoin%
\definecolor{currentfill}{rgb}{1.000000,1.000000,1.000000}%
\pgfsetfillcolor{currentfill}%
\pgfsetlinewidth{0.000000pt}%
\definecolor{currentstroke}{rgb}{0.000000,0.000000,0.000000}%
\pgfsetstrokecolor{currentstroke}%
\pgfsetstrokeopacity{0.000000}%
\pgfsetdash{}{0pt}%
\pgfpathmoveto{\pgfqpoint{0.553704in}{0.650826in}}%
\pgfpathlineto{\pgfqpoint{5.436553in}{0.650826in}}%
\pgfpathlineto{\pgfqpoint{5.436553in}{1.840036in}}%
\pgfpathlineto{\pgfqpoint{0.553704in}{1.840036in}}%
\pgfpathclose%
\pgfusepath{fill}%
\end{pgfscope}%
\begin{pgfscope}%
\pgfsetbuttcap%
\pgfsetroundjoin%
\definecolor{currentfill}{rgb}{0.000000,0.000000,0.000000}%
\pgfsetfillcolor{currentfill}%
\pgfsetlinewidth{0.803000pt}%
\definecolor{currentstroke}{rgb}{0.000000,0.000000,0.000000}%
\pgfsetstrokecolor{currentstroke}%
\pgfsetdash{}{0pt}%
\pgfsys@defobject{currentmarker}{\pgfqpoint{0.000000in}{-0.048611in}}{\pgfqpoint{0.000000in}{0.000000in}}{%
\pgfpathmoveto{\pgfqpoint{0.000000in}{0.000000in}}%
\pgfpathlineto{\pgfqpoint{0.000000in}{-0.048611in}}%
\pgfusepath{stroke,fill}%
}%
\begin{pgfscope}%
\pgfsys@transformshift{0.980604in}{0.650826in}%
\pgfsys@useobject{currentmarker}{}%
\end{pgfscope}%
\end{pgfscope}%
\begin{pgfscope}%
\definecolor{textcolor}{rgb}{0.000000,0.000000,0.000000}%
\pgfsetstrokecolor{textcolor}%
\pgfsetfillcolor{textcolor}%
\pgftext[x=0.980604in,y=0.553603in,,top]{\color{textcolor}\rmfamily\fontsize{10.000000}{12.000000}\selectfont \(\displaystyle {2008}\)}%
\end{pgfscope}%
\begin{pgfscope}%
\pgfsetbuttcap%
\pgfsetroundjoin%
\definecolor{currentfill}{rgb}{0.000000,0.000000,0.000000}%
\pgfsetfillcolor{currentfill}%
\pgfsetlinewidth{0.803000pt}%
\definecolor{currentstroke}{rgb}{0.000000,0.000000,0.000000}%
\pgfsetstrokecolor{currentstroke}%
\pgfsetdash{}{0pt}%
\pgfsys@defobject{currentmarker}{\pgfqpoint{0.000000in}{-0.048611in}}{\pgfqpoint{0.000000in}{0.000000in}}{%
\pgfpathmoveto{\pgfqpoint{0.000000in}{0.000000in}}%
\pgfpathlineto{\pgfqpoint{0.000000in}{-0.048611in}}%
\pgfusepath{stroke,fill}%
}%
\begin{pgfscope}%
\pgfsys@transformshift{1.589627in}{0.650826in}%
\pgfsys@useobject{currentmarker}{}%
\end{pgfscope}%
\end{pgfscope}%
\begin{pgfscope}%
\definecolor{textcolor}{rgb}{0.000000,0.000000,0.000000}%
\pgfsetstrokecolor{textcolor}%
\pgfsetfillcolor{textcolor}%
\pgftext[x=1.589627in,y=0.553603in,,top]{\color{textcolor}\rmfamily\fontsize{10.000000}{12.000000}\selectfont \(\displaystyle {2010}\)}%
\end{pgfscope}%
\begin{pgfscope}%
\pgfsetbuttcap%
\pgfsetroundjoin%
\definecolor{currentfill}{rgb}{0.000000,0.000000,0.000000}%
\pgfsetfillcolor{currentfill}%
\pgfsetlinewidth{0.803000pt}%
\definecolor{currentstroke}{rgb}{0.000000,0.000000,0.000000}%
\pgfsetstrokecolor{currentstroke}%
\pgfsetdash{}{0pt}%
\pgfsys@defobject{currentmarker}{\pgfqpoint{0.000000in}{-0.048611in}}{\pgfqpoint{0.000000in}{0.000000in}}{%
\pgfpathmoveto{\pgfqpoint{0.000000in}{0.000000in}}%
\pgfpathlineto{\pgfqpoint{0.000000in}{-0.048611in}}%
\pgfusepath{stroke,fill}%
}%
\begin{pgfscope}%
\pgfsys@transformshift{2.197817in}{0.650826in}%
\pgfsys@useobject{currentmarker}{}%
\end{pgfscope}%
\end{pgfscope}%
\begin{pgfscope}%
\definecolor{textcolor}{rgb}{0.000000,0.000000,0.000000}%
\pgfsetstrokecolor{textcolor}%
\pgfsetfillcolor{textcolor}%
\pgftext[x=2.197817in,y=0.553603in,,top]{\color{textcolor}\rmfamily\fontsize{10.000000}{12.000000}\selectfont \(\displaystyle {2012}\)}%
\end{pgfscope}%
\begin{pgfscope}%
\pgfsetbuttcap%
\pgfsetroundjoin%
\definecolor{currentfill}{rgb}{0.000000,0.000000,0.000000}%
\pgfsetfillcolor{currentfill}%
\pgfsetlinewidth{0.803000pt}%
\definecolor{currentstroke}{rgb}{0.000000,0.000000,0.000000}%
\pgfsetstrokecolor{currentstroke}%
\pgfsetdash{}{0pt}%
\pgfsys@defobject{currentmarker}{\pgfqpoint{0.000000in}{-0.048611in}}{\pgfqpoint{0.000000in}{0.000000in}}{%
\pgfpathmoveto{\pgfqpoint{0.000000in}{0.000000in}}%
\pgfpathlineto{\pgfqpoint{0.000000in}{-0.048611in}}%
\pgfusepath{stroke,fill}%
}%
\begin{pgfscope}%
\pgfsys@transformshift{2.806840in}{0.650826in}%
\pgfsys@useobject{currentmarker}{}%
\end{pgfscope}%
\end{pgfscope}%
\begin{pgfscope}%
\definecolor{textcolor}{rgb}{0.000000,0.000000,0.000000}%
\pgfsetstrokecolor{textcolor}%
\pgfsetfillcolor{textcolor}%
\pgftext[x=2.806840in,y=0.553603in,,top]{\color{textcolor}\rmfamily\fontsize{10.000000}{12.000000}\selectfont \(\displaystyle {2014}\)}%
\end{pgfscope}%
\begin{pgfscope}%
\pgfsetbuttcap%
\pgfsetroundjoin%
\definecolor{currentfill}{rgb}{0.000000,0.000000,0.000000}%
\pgfsetfillcolor{currentfill}%
\pgfsetlinewidth{0.803000pt}%
\definecolor{currentstroke}{rgb}{0.000000,0.000000,0.000000}%
\pgfsetstrokecolor{currentstroke}%
\pgfsetdash{}{0pt}%
\pgfsys@defobject{currentmarker}{\pgfqpoint{0.000000in}{-0.048611in}}{\pgfqpoint{0.000000in}{0.000000in}}{%
\pgfpathmoveto{\pgfqpoint{0.000000in}{0.000000in}}%
\pgfpathlineto{\pgfqpoint{0.000000in}{-0.048611in}}%
\pgfusepath{stroke,fill}%
}%
\begin{pgfscope}%
\pgfsys@transformshift{3.415030in}{0.650826in}%
\pgfsys@useobject{currentmarker}{}%
\end{pgfscope}%
\end{pgfscope}%
\begin{pgfscope}%
\definecolor{textcolor}{rgb}{0.000000,0.000000,0.000000}%
\pgfsetstrokecolor{textcolor}%
\pgfsetfillcolor{textcolor}%
\pgftext[x=3.415030in,y=0.553603in,,top]{\color{textcolor}\rmfamily\fontsize{10.000000}{12.000000}\selectfont \(\displaystyle {2016}\)}%
\end{pgfscope}%
\begin{pgfscope}%
\pgfsetbuttcap%
\pgfsetroundjoin%
\definecolor{currentfill}{rgb}{0.000000,0.000000,0.000000}%
\pgfsetfillcolor{currentfill}%
\pgfsetlinewidth{0.803000pt}%
\definecolor{currentstroke}{rgb}{0.000000,0.000000,0.000000}%
\pgfsetstrokecolor{currentstroke}%
\pgfsetdash{}{0pt}%
\pgfsys@defobject{currentmarker}{\pgfqpoint{0.000000in}{-0.048611in}}{\pgfqpoint{0.000000in}{0.000000in}}{%
\pgfpathmoveto{\pgfqpoint{0.000000in}{0.000000in}}%
\pgfpathlineto{\pgfqpoint{0.000000in}{-0.048611in}}%
\pgfusepath{stroke,fill}%
}%
\begin{pgfscope}%
\pgfsys@transformshift{4.024053in}{0.650826in}%
\pgfsys@useobject{currentmarker}{}%
\end{pgfscope}%
\end{pgfscope}%
\begin{pgfscope}%
\definecolor{textcolor}{rgb}{0.000000,0.000000,0.000000}%
\pgfsetstrokecolor{textcolor}%
\pgfsetfillcolor{textcolor}%
\pgftext[x=4.024053in,y=0.553603in,,top]{\color{textcolor}\rmfamily\fontsize{10.000000}{12.000000}\selectfont \(\displaystyle {2018}\)}%
\end{pgfscope}%
\begin{pgfscope}%
\pgfsetbuttcap%
\pgfsetroundjoin%
\definecolor{currentfill}{rgb}{0.000000,0.000000,0.000000}%
\pgfsetfillcolor{currentfill}%
\pgfsetlinewidth{0.803000pt}%
\definecolor{currentstroke}{rgb}{0.000000,0.000000,0.000000}%
\pgfsetstrokecolor{currentstroke}%
\pgfsetdash{}{0pt}%
\pgfsys@defobject{currentmarker}{\pgfqpoint{0.000000in}{-0.048611in}}{\pgfqpoint{0.000000in}{0.000000in}}{%
\pgfpathmoveto{\pgfqpoint{0.000000in}{0.000000in}}%
\pgfpathlineto{\pgfqpoint{0.000000in}{-0.048611in}}%
\pgfusepath{stroke,fill}%
}%
\begin{pgfscope}%
\pgfsys@transformshift{4.632243in}{0.650826in}%
\pgfsys@useobject{currentmarker}{}%
\end{pgfscope}%
\end{pgfscope}%
\begin{pgfscope}%
\definecolor{textcolor}{rgb}{0.000000,0.000000,0.000000}%
\pgfsetstrokecolor{textcolor}%
\pgfsetfillcolor{textcolor}%
\pgftext[x=4.632243in,y=0.553603in,,top]{\color{textcolor}\rmfamily\fontsize{10.000000}{12.000000}\selectfont \(\displaystyle {2020}\)}%
\end{pgfscope}%
\begin{pgfscope}%
\pgfsetbuttcap%
\pgfsetroundjoin%
\definecolor{currentfill}{rgb}{0.000000,0.000000,0.000000}%
\pgfsetfillcolor{currentfill}%
\pgfsetlinewidth{0.803000pt}%
\definecolor{currentstroke}{rgb}{0.000000,0.000000,0.000000}%
\pgfsetstrokecolor{currentstroke}%
\pgfsetdash{}{0pt}%
\pgfsys@defobject{currentmarker}{\pgfqpoint{0.000000in}{-0.048611in}}{\pgfqpoint{0.000000in}{0.000000in}}{%
\pgfpathmoveto{\pgfqpoint{0.000000in}{0.000000in}}%
\pgfpathlineto{\pgfqpoint{0.000000in}{-0.048611in}}%
\pgfusepath{stroke,fill}%
}%
\begin{pgfscope}%
\pgfsys@transformshift{5.241266in}{0.650826in}%
\pgfsys@useobject{currentmarker}{}%
\end{pgfscope}%
\end{pgfscope}%
\begin{pgfscope}%
\definecolor{textcolor}{rgb}{0.000000,0.000000,0.000000}%
\pgfsetstrokecolor{textcolor}%
\pgfsetfillcolor{textcolor}%
\pgftext[x=5.241266in,y=0.553603in,,top]{\color{textcolor}\rmfamily\fontsize{10.000000}{12.000000}\selectfont \(\displaystyle {2022}\)}%
\end{pgfscope}%
\begin{pgfscope}%
\pgfsetbuttcap%
\pgfsetroundjoin%
\definecolor{currentfill}{rgb}{0.000000,0.000000,0.000000}%
\pgfsetfillcolor{currentfill}%
\pgfsetlinewidth{0.803000pt}%
\definecolor{currentstroke}{rgb}{0.000000,0.000000,0.000000}%
\pgfsetstrokecolor{currentstroke}%
\pgfsetdash{}{0pt}%
\pgfsys@defobject{currentmarker}{\pgfqpoint{-0.048611in}{0.000000in}}{\pgfqpoint{-0.000000in}{0.000000in}}{%
\pgfpathmoveto{\pgfqpoint{-0.000000in}{0.000000in}}%
\pgfpathlineto{\pgfqpoint{-0.048611in}{0.000000in}}%
\pgfusepath{stroke,fill}%
}%
\begin{pgfscope}%
\pgfsys@transformshift{0.553704in}{0.704881in}%
\pgfsys@useobject{currentmarker}{}%
\end{pgfscope}%
\end{pgfscope}%
\begin{pgfscope}%
\definecolor{textcolor}{rgb}{0.000000,0.000000,0.000000}%
\pgfsetstrokecolor{textcolor}%
\pgfsetfillcolor{textcolor}%
\pgftext[x=0.279012in, y=0.656655in, left, base]{\color{textcolor}\rmfamily\fontsize{10.000000}{12.000000}\selectfont \(\displaystyle {0.0}\)}%
\end{pgfscope}%
\begin{pgfscope}%
\pgfsetbuttcap%
\pgfsetroundjoin%
\definecolor{currentfill}{rgb}{0.000000,0.000000,0.000000}%
\pgfsetfillcolor{currentfill}%
\pgfsetlinewidth{0.803000pt}%
\definecolor{currentstroke}{rgb}{0.000000,0.000000,0.000000}%
\pgfsetstrokecolor{currentstroke}%
\pgfsetdash{}{0pt}%
\pgfsys@defobject{currentmarker}{\pgfqpoint{-0.048611in}{0.000000in}}{\pgfqpoint{-0.000000in}{0.000000in}}{%
\pgfpathmoveto{\pgfqpoint{-0.000000in}{0.000000in}}%
\pgfpathlineto{\pgfqpoint{-0.048611in}{0.000000in}}%
\pgfusepath{stroke,fill}%
}%
\begin{pgfscope}%
\pgfsys@transformshift{0.553704in}{1.186311in}%
\pgfsys@useobject{currentmarker}{}%
\end{pgfscope}%
\end{pgfscope}%
\begin{pgfscope}%
\definecolor{textcolor}{rgb}{0.000000,0.000000,0.000000}%
\pgfsetstrokecolor{textcolor}%
\pgfsetfillcolor{textcolor}%
\pgftext[x=0.279012in, y=1.138086in, left, base]{\color{textcolor}\rmfamily\fontsize{10.000000}{12.000000}\selectfont \(\displaystyle {0.5}\)}%
\end{pgfscope}%
\begin{pgfscope}%
\pgfsetbuttcap%
\pgfsetroundjoin%
\definecolor{currentfill}{rgb}{0.000000,0.000000,0.000000}%
\pgfsetfillcolor{currentfill}%
\pgfsetlinewidth{0.803000pt}%
\definecolor{currentstroke}{rgb}{0.000000,0.000000,0.000000}%
\pgfsetstrokecolor{currentstroke}%
\pgfsetdash{}{0pt}%
\pgfsys@defobject{currentmarker}{\pgfqpoint{-0.048611in}{0.000000in}}{\pgfqpoint{-0.000000in}{0.000000in}}{%
\pgfpathmoveto{\pgfqpoint{-0.000000in}{0.000000in}}%
\pgfpathlineto{\pgfqpoint{-0.048611in}{0.000000in}}%
\pgfusepath{stroke,fill}%
}%
\begin{pgfscope}%
\pgfsys@transformshift{0.553704in}{1.667741in}%
\pgfsys@useobject{currentmarker}{}%
\end{pgfscope}%
\end{pgfscope}%
\begin{pgfscope}%
\definecolor{textcolor}{rgb}{0.000000,0.000000,0.000000}%
\pgfsetstrokecolor{textcolor}%
\pgfsetfillcolor{textcolor}%
\pgftext[x=0.279012in, y=1.619516in, left, base]{\color{textcolor}\rmfamily\fontsize{10.000000}{12.000000}\selectfont \(\displaystyle {1.0}\)}%
\end{pgfscope}%
\begin{pgfscope}%
\definecolor{textcolor}{rgb}{0.000000,0.000000,0.000000}%
\pgfsetstrokecolor{textcolor}%
\pgfsetfillcolor{textcolor}%
\pgftext[x=0.223457in,y=1.245431in,,bottom,rotate=90.000000]{\color{textcolor}\rmfamily\fontsize{10.000000}{12.000000}\selectfont Videók száma}%
\end{pgfscope}%
\begin{pgfscope}%
\definecolor{textcolor}{rgb}{0.000000,0.000000,0.000000}%
\pgfsetstrokecolor{textcolor}%
\pgfsetfillcolor{textcolor}%
\pgftext[x=0.553704in,y=1.881702in,left,base]{\color{textcolor}\rmfamily\fontsize{10.000000}{12.000000}\selectfont \(\displaystyle \times{10^{5}}{}\)}%
\end{pgfscope}%
\begin{pgfscope}%
\pgfpathrectangle{\pgfqpoint{0.553704in}{0.650826in}}{\pgfqpoint{4.882849in}{1.189210in}}%
\pgfusepath{clip}%
\pgfsetrectcap%
\pgfsetroundjoin%
\pgfsetlinewidth{1.505625pt}%
\definecolor{currentstroke}{rgb}{0.121569,0.466667,0.705882}%
\pgfsetstrokecolor{currentstroke}%
\pgfsetdash{}{0pt}%
\pgfpathmoveto{\pgfqpoint{0.775652in}{0.704890in}}%
\pgfpathlineto{\pgfqpoint{1.663776in}{0.706325in}}%
\pgfpathlineto{\pgfqpoint{1.765418in}{0.708655in}}%
\pgfpathlineto{\pgfqpoint{1.816240in}{0.707788in}}%
\pgfpathlineto{\pgfqpoint{1.842067in}{0.717321in}}%
\pgfpathlineto{\pgfqpoint{1.867061in}{0.716888in}}%
\pgfpathlineto{\pgfqpoint{1.892888in}{0.711746in}}%
\pgfpathlineto{\pgfqpoint{1.918716in}{0.711496in}}%
\pgfpathlineto{\pgfqpoint{1.942043in}{0.714461in}}%
\pgfpathlineto{\pgfqpoint{1.967871in}{0.711746in}}%
\pgfpathlineto{\pgfqpoint{2.018692in}{0.709791in}}%
\pgfpathlineto{\pgfqpoint{2.043686in}{0.712680in}}%
\pgfpathlineto{\pgfqpoint{2.069513in}{0.712699in}}%
\pgfpathlineto{\pgfqpoint{2.095341in}{0.715982in}}%
\pgfpathlineto{\pgfqpoint{2.146162in}{0.715482in}}%
\pgfpathlineto{\pgfqpoint{2.222811in}{0.718871in}}%
\pgfpathlineto{\pgfqpoint{2.297793in}{0.722357in}}%
\pgfpathlineto{\pgfqpoint{2.323620in}{0.716059in}}%
\pgfpathlineto{\pgfqpoint{2.348614in}{0.724658in}}%
\pgfpathlineto{\pgfqpoint{2.374442in}{0.724359in}}%
\pgfpathlineto{\pgfqpoint{2.425263in}{0.721230in}}%
\pgfpathlineto{\pgfqpoint{2.451090in}{0.724128in}}%
\pgfpathlineto{\pgfqpoint{2.476084in}{0.724966in}}%
\pgfpathlineto{\pgfqpoint{2.501911in}{0.769883in}}%
\pgfpathlineto{\pgfqpoint{2.527739in}{0.807868in}}%
\pgfpathlineto{\pgfqpoint{2.551067in}{0.777548in}}%
\pgfpathlineto{\pgfqpoint{2.576894in}{0.740699in}}%
\pgfpathlineto{\pgfqpoint{2.601888in}{0.752225in}}%
\pgfpathlineto{\pgfqpoint{2.627715in}{0.738167in}}%
\pgfpathlineto{\pgfqpoint{2.652709in}{0.752128in}}%
\pgfpathlineto{\pgfqpoint{2.678537in}{0.735249in}}%
\pgfpathlineto{\pgfqpoint{2.704364in}{0.751108in}}%
\pgfpathlineto{\pgfqpoint{2.729358in}{0.748970in}}%
\pgfpathlineto{\pgfqpoint{2.755185in}{0.736809in}}%
\pgfpathlineto{\pgfqpoint{2.780179in}{0.746072in}}%
\pgfpathlineto{\pgfqpoint{2.806006in}{0.735297in}}%
\pgfpathlineto{\pgfqpoint{2.831834in}{0.773494in}}%
\pgfpathlineto{\pgfqpoint{2.855162in}{0.764289in}}%
\pgfpathlineto{\pgfqpoint{2.880989in}{0.775853in}}%
\pgfpathlineto{\pgfqpoint{2.905983in}{0.779743in}}%
\pgfpathlineto{\pgfqpoint{2.931810in}{0.773407in}}%
\pgfpathlineto{\pgfqpoint{2.956804in}{0.773379in}}%
\pgfpathlineto{\pgfqpoint{2.982631in}{0.776354in}}%
\pgfpathlineto{\pgfqpoint{3.008459in}{0.775930in}}%
\pgfpathlineto{\pgfqpoint{3.033453in}{0.782564in}}%
\pgfpathlineto{\pgfqpoint{3.059280in}{0.771154in}}%
\pgfpathlineto{\pgfqpoint{3.084274in}{0.774871in}}%
\pgfpathlineto{\pgfqpoint{3.110101in}{0.764472in}}%
\pgfpathlineto{\pgfqpoint{3.135929in}{0.743405in}}%
\pgfpathlineto{\pgfqpoint{3.159257in}{0.742798in}}%
\pgfpathlineto{\pgfqpoint{3.185084in}{0.749788in}}%
\pgfpathlineto{\pgfqpoint{3.210078in}{0.751108in}}%
\pgfpathlineto{\pgfqpoint{3.235905in}{0.748075in}}%
\pgfpathlineto{\pgfqpoint{3.260899in}{0.754025in}}%
\pgfpathlineto{\pgfqpoint{3.286726in}{0.788775in}}%
\pgfpathlineto{\pgfqpoint{3.312554in}{0.756866in}}%
\pgfpathlineto{\pgfqpoint{3.337548in}{0.766417in}}%
\pgfpathlineto{\pgfqpoint{3.363375in}{0.761824in}}%
\pgfpathlineto{\pgfqpoint{3.388369in}{0.783816in}}%
\pgfpathlineto{\pgfqpoint{3.414196in}{0.769498in}}%
\pgfpathlineto{\pgfqpoint{3.440024in}{0.769103in}}%
\pgfpathlineto{\pgfqpoint{3.464185in}{0.770962in}}%
\pgfpathlineto{\pgfqpoint{3.490012in}{0.792530in}}%
\pgfpathlineto{\pgfqpoint{3.515006in}{0.782449in}}%
\pgfpathlineto{\pgfqpoint{3.540833in}{0.789882in}}%
\pgfpathlineto{\pgfqpoint{3.565827in}{0.783989in}}%
\pgfpathlineto{\pgfqpoint{3.591655in}{0.793666in}}%
\pgfpathlineto{\pgfqpoint{3.617482in}{0.789304in}}%
\pgfpathlineto{\pgfqpoint{3.642476in}{0.787947in}}%
\pgfpathlineto{\pgfqpoint{3.668303in}{0.789843in}}%
\pgfpathlineto{\pgfqpoint{3.693297in}{0.786887in}}%
\pgfpathlineto{\pgfqpoint{3.719125in}{0.795505in}}%
\pgfpathlineto{\pgfqpoint{3.744952in}{0.800194in}}%
\pgfpathlineto{\pgfqpoint{3.768280in}{0.795813in}}%
\pgfpathlineto{\pgfqpoint{3.794107in}{0.815359in}}%
\pgfpathlineto{\pgfqpoint{3.819101in}{0.814993in}}%
\pgfpathlineto{\pgfqpoint{3.844928in}{0.825864in}}%
\pgfpathlineto{\pgfqpoint{3.869922in}{0.831612in}}%
\pgfpathlineto{\pgfqpoint{3.895750in}{0.829976in}}%
\pgfpathlineto{\pgfqpoint{3.921577in}{0.861134in}}%
\pgfpathlineto{\pgfqpoint{3.946571in}{0.838478in}}%
\pgfpathlineto{\pgfqpoint{3.972398in}{0.845632in}}%
\pgfpathlineto{\pgfqpoint{3.997392in}{0.863724in}}%
\pgfpathlineto{\pgfqpoint{4.023219in}{0.866930in}}%
\pgfpathlineto{\pgfqpoint{4.049047in}{0.875817in}}%
\pgfpathlineto{\pgfqpoint{4.072375in}{0.874026in}}%
\pgfpathlineto{\pgfqpoint{4.098202in}{0.881267in}}%
\pgfpathlineto{\pgfqpoint{4.149023in}{0.916874in}}%
\pgfpathlineto{\pgfqpoint{4.174017in}{0.921101in}}%
\pgfpathlineto{\pgfqpoint{4.199845in}{0.920610in}}%
\pgfpathlineto{\pgfqpoint{4.225672in}{0.937931in}}%
\pgfpathlineto{\pgfqpoint{4.250666in}{0.945644in}}%
\pgfpathlineto{\pgfqpoint{4.276493in}{0.974193in}}%
\pgfpathlineto{\pgfqpoint{4.301487in}{0.980028in}}%
\pgfpathlineto{\pgfqpoint{4.327314in}{1.015557in}}%
\pgfpathlineto{\pgfqpoint{4.353142in}{1.041140in}}%
\pgfpathlineto{\pgfqpoint{4.376470in}{1.052242in}}%
\pgfpathlineto{\pgfqpoint{4.402297in}{1.109128in}}%
\pgfpathlineto{\pgfqpoint{4.427291in}{1.114029in}}%
\pgfpathlineto{\pgfqpoint{4.453118in}{1.157416in}}%
\pgfpathlineto{\pgfqpoint{4.478112in}{1.163270in}}%
\pgfpathlineto{\pgfqpoint{4.503939in}{1.230044in}}%
\pgfpathlineto{\pgfqpoint{4.529767in}{1.250187in}}%
\pgfpathlineto{\pgfqpoint{4.554761in}{1.242542in}}%
\pgfpathlineto{\pgfqpoint{4.580588in}{1.271707in}}%
\pgfpathlineto{\pgfqpoint{4.605582in}{1.273113in}}%
\pgfpathlineto{\pgfqpoint{4.631409in}{1.283550in}}%
\pgfpathlineto{\pgfqpoint{4.657237in}{1.327052in}}%
\pgfpathlineto{\pgfqpoint{4.681398in}{1.308373in}}%
\pgfpathlineto{\pgfqpoint{4.707225in}{1.421904in}}%
\pgfpathlineto{\pgfqpoint{4.732219in}{1.508561in}}%
\pgfpathlineto{\pgfqpoint{4.758046in}{1.485135in}}%
\pgfpathlineto{\pgfqpoint{4.783040in}{1.448527in}}%
\pgfpathlineto{\pgfqpoint{4.808868in}{1.503333in}}%
\pgfpathlineto{\pgfqpoint{4.834695in}{1.513876in}}%
\pgfpathlineto{\pgfqpoint{4.859689in}{1.508523in}}%
\pgfpathlineto{\pgfqpoint{4.885516in}{1.550869in}}%
\pgfpathlineto{\pgfqpoint{4.910510in}{1.573631in}}%
\pgfpathlineto{\pgfqpoint{4.936338in}{1.619983in}}%
\pgfpathlineto{\pgfqpoint{4.962165in}{1.524131in}}%
\pgfpathlineto{\pgfqpoint{4.985493in}{1.456461in}}%
\pgfpathlineto{\pgfqpoint{5.011320in}{1.523890in}}%
\pgfpathlineto{\pgfqpoint{5.036314in}{1.481360in}}%
\pgfpathlineto{\pgfqpoint{5.062141in}{1.552593in}}%
\pgfpathlineto{\pgfqpoint{5.087135in}{1.569635in}}%
\pgfpathlineto{\pgfqpoint{5.112963in}{1.668897in}}%
\pgfpathlineto{\pgfqpoint{5.138790in}{1.703107in}}%
\pgfpathlineto{\pgfqpoint{5.163784in}{1.761091in}}%
\pgfpathlineto{\pgfqpoint{5.189611in}{1.776169in}}%
\pgfpathlineto{\pgfqpoint{5.214605in}{1.785981in}}%
\pgfpathlineto{\pgfqpoint{5.214605in}{1.785981in}}%
\pgfusepath{stroke}%
\end{pgfscope}%
\begin{pgfscope}%
\pgfsetrectcap%
\pgfsetmiterjoin%
\pgfsetlinewidth{0.803000pt}%
\definecolor{currentstroke}{rgb}{0.000000,0.000000,0.000000}%
\pgfsetstrokecolor{currentstroke}%
\pgfsetdash{}{0pt}%
\pgfpathmoveto{\pgfqpoint{0.553704in}{0.650826in}}%
\pgfpathlineto{\pgfqpoint{0.553704in}{1.840036in}}%
\pgfusepath{stroke}%
\end{pgfscope}%
\begin{pgfscope}%
\pgfsetrectcap%
\pgfsetmiterjoin%
\pgfsetlinewidth{0.803000pt}%
\definecolor{currentstroke}{rgb}{0.000000,0.000000,0.000000}%
\pgfsetstrokecolor{currentstroke}%
\pgfsetdash{}{0pt}%
\pgfpathmoveto{\pgfqpoint{5.436553in}{0.650826in}}%
\pgfpathlineto{\pgfqpoint{5.436553in}{1.840036in}}%
\pgfusepath{stroke}%
\end{pgfscope}%
\begin{pgfscope}%
\pgfsetrectcap%
\pgfsetmiterjoin%
\pgfsetlinewidth{0.803000pt}%
\definecolor{currentstroke}{rgb}{0.000000,0.000000,0.000000}%
\pgfsetstrokecolor{currentstroke}%
\pgfsetdash{}{0pt}%
\pgfpathmoveto{\pgfqpoint{0.553704in}{0.650826in}}%
\pgfpathlineto{\pgfqpoint{5.436553in}{0.650826in}}%
\pgfusepath{stroke}%
\end{pgfscope}%
\begin{pgfscope}%
\pgfsetrectcap%
\pgfsetmiterjoin%
\pgfsetlinewidth{0.803000pt}%
\definecolor{currentstroke}{rgb}{0.000000,0.000000,0.000000}%
\pgfsetstrokecolor{currentstroke}%
\pgfsetdash{}{0pt}%
\pgfpathmoveto{\pgfqpoint{0.553704in}{1.840036in}}%
\pgfpathlineto{\pgfqpoint{5.436553in}{1.840036in}}%
\pgfusepath{stroke}%
\end{pgfscope}%
\begin{pgfscope}%
\pgfsetbuttcap%
\pgfsetroundjoin%
\definecolor{currentfill}{rgb}{0.000000,0.000000,0.000000}%
\pgfsetfillcolor{currentfill}%
\pgfsetlinewidth{0.803000pt}%
\definecolor{currentstroke}{rgb}{0.000000,0.000000,0.000000}%
\pgfsetstrokecolor{currentstroke}%
\pgfsetdash{}{0pt}%
\pgfsys@defobject{currentmarker}{\pgfqpoint{0.000000in}{0.000000in}}{\pgfqpoint{0.048611in}{0.000000in}}{%
\pgfpathmoveto{\pgfqpoint{0.000000in}{0.000000in}}%
\pgfpathlineto{\pgfqpoint{0.048611in}{0.000000in}}%
\pgfusepath{stroke,fill}%
}%
\begin{pgfscope}%
\pgfsys@transformshift{5.436553in}{0.704881in}%
\pgfsys@useobject{currentmarker}{}%
\end{pgfscope}%
\end{pgfscope}%
\begin{pgfscope}%
\definecolor{textcolor}{rgb}{0.000000,0.000000,0.000000}%
\pgfsetstrokecolor{textcolor}%
\pgfsetfillcolor{textcolor}%
\pgftext[x=5.533775in, y=0.656655in, left, base]{\color{textcolor}\rmfamily\fontsize{10.000000}{12.000000}\selectfont \(\displaystyle {0}\)}%
\end{pgfscope}%
\begin{pgfscope}%
\pgfsetbuttcap%
\pgfsetroundjoin%
\definecolor{currentfill}{rgb}{0.000000,0.000000,0.000000}%
\pgfsetfillcolor{currentfill}%
\pgfsetlinewidth{0.803000pt}%
\definecolor{currentstroke}{rgb}{0.000000,0.000000,0.000000}%
\pgfsetstrokecolor{currentstroke}%
\pgfsetdash{}{0pt}%
\pgfsys@defobject{currentmarker}{\pgfqpoint{0.000000in}{0.000000in}}{\pgfqpoint{0.048611in}{0.000000in}}{%
\pgfpathmoveto{\pgfqpoint{0.000000in}{0.000000in}}%
\pgfpathlineto{\pgfqpoint{0.048611in}{0.000000in}}%
\pgfusepath{stroke,fill}%
}%
\begin{pgfscope}%
\pgfsys@transformshift{5.436553in}{1.104371in}%
\pgfsys@useobject{currentmarker}{}%
\end{pgfscope}%
\end{pgfscope}%
\begin{pgfscope}%
\definecolor{textcolor}{rgb}{0.000000,0.000000,0.000000}%
\pgfsetstrokecolor{textcolor}%
\pgfsetfillcolor{textcolor}%
\pgftext[x=5.533775in, y=1.056146in, left, base]{\color{textcolor}\rmfamily\fontsize{10.000000}{12.000000}\selectfont \(\displaystyle {2}\)}%
\end{pgfscope}%
\begin{pgfscope}%
\pgfsetbuttcap%
\pgfsetroundjoin%
\definecolor{currentfill}{rgb}{0.000000,0.000000,0.000000}%
\pgfsetfillcolor{currentfill}%
\pgfsetlinewidth{0.803000pt}%
\definecolor{currentstroke}{rgb}{0.000000,0.000000,0.000000}%
\pgfsetstrokecolor{currentstroke}%
\pgfsetdash{}{0pt}%
\pgfsys@defobject{currentmarker}{\pgfqpoint{0.000000in}{0.000000in}}{\pgfqpoint{0.048611in}{0.000000in}}{%
\pgfpathmoveto{\pgfqpoint{0.000000in}{0.000000in}}%
\pgfpathlineto{\pgfqpoint{0.048611in}{0.000000in}}%
\pgfusepath{stroke,fill}%
}%
\begin{pgfscope}%
\pgfsys@transformshift{5.436553in}{1.503861in}%
\pgfsys@useobject{currentmarker}{}%
\end{pgfscope}%
\end{pgfscope}%
\begin{pgfscope}%
\definecolor{textcolor}{rgb}{0.000000,0.000000,0.000000}%
\pgfsetstrokecolor{textcolor}%
\pgfsetfillcolor{textcolor}%
\pgftext[x=5.533775in, y=1.455636in, left, base]{\color{textcolor}\rmfamily\fontsize{10.000000}{12.000000}\selectfont \(\displaystyle {4}\)}%
\end{pgfscope}%
\begin{pgfscope}%
\definecolor{textcolor}{rgb}{0.000000,0.000000,0.000000}%
\pgfsetstrokecolor{textcolor}%
\pgfsetfillcolor{textcolor}%
\pgftext[x=5.755226in, y=0.930616in, left, base,rotate=90.000000]{\color{textcolor}\rmfamily\fontsize{10.000000}{12.000000}\selectfont Nézettség }%
\end{pgfscope}%
\begin{pgfscope}%
\definecolor{textcolor}{rgb}{0.000000,0.000000,0.000000}%
\pgfsetstrokecolor{textcolor}%
\pgfsetfillcolor{textcolor}%
\pgftext[x=5.897973in, y=0.971510in, left, base,rotate=90.000000]{\color{textcolor}\rmfamily\fontsize{10.000000}{12.000000}\selectfont 2020-ban}%
\end{pgfscope}%
\begin{pgfscope}%
\definecolor{textcolor}{rgb}{0.000000,0.000000,0.000000}%
\pgfsetstrokecolor{textcolor}%
\pgfsetfillcolor{textcolor}%
\pgftext[x=5.436553in,y=1.881702in,right,base]{\color{textcolor}\rmfamily\fontsize{10.000000}{12.000000}\selectfont \(\displaystyle \times{10^{9}}{}\)}%
\end{pgfscope}%
\begin{pgfscope}%
\pgfpathrectangle{\pgfqpoint{0.553704in}{0.650826in}}{\pgfqpoint{4.882849in}{1.189210in}}%
\pgfusepath{clip}%
\pgfsetrectcap%
\pgfsetroundjoin%
\pgfsetlinewidth{1.505625pt}%
\definecolor{currentstroke}{rgb}{1.000000,0.549020,0.000000}%
\pgfsetstrokecolor{currentstroke}%
\pgfsetdash{}{0pt}%
\pgfpathmoveto{\pgfqpoint{0.775652in}{0.705048in}}%
\pgfpathlineto{\pgfqpoint{1.005598in}{0.705025in}}%
\pgfpathlineto{\pgfqpoint{1.080580in}{0.714387in}}%
\pgfpathlineto{\pgfqpoint{1.106407in}{0.706094in}}%
\pgfpathlineto{\pgfqpoint{1.157229in}{0.705543in}}%
\pgfpathlineto{\pgfqpoint{1.183056in}{0.710322in}}%
\pgfpathlineto{\pgfqpoint{1.208050in}{0.705770in}}%
\pgfpathlineto{\pgfqpoint{1.258871in}{0.704881in}}%
\pgfpathlineto{\pgfqpoint{1.284698in}{0.704881in}}%
\pgfpathlineto{\pgfqpoint{1.310526in}{0.709597in}}%
\pgfpathlineto{\pgfqpoint{1.333854in}{0.705888in}}%
\pgfpathlineto{\pgfqpoint{1.384675in}{0.712860in}}%
\pgfpathlineto{\pgfqpoint{1.410502in}{0.715596in}}%
\pgfpathlineto{\pgfqpoint{1.435496in}{0.711879in}}%
\pgfpathlineto{\pgfqpoint{1.487151in}{0.712115in}}%
\pgfpathlineto{\pgfqpoint{1.512145in}{0.709878in}}%
\pgfpathlineto{\pgfqpoint{1.537972in}{0.711584in}}%
\pgfpathlineto{\pgfqpoint{1.562966in}{0.709870in}}%
\pgfpathlineto{\pgfqpoint{1.588793in}{0.711795in}}%
\pgfpathlineto{\pgfqpoint{1.614621in}{0.720326in}}%
\pgfpathlineto{\pgfqpoint{1.637949in}{0.749024in}}%
\pgfpathlineto{\pgfqpoint{1.663776in}{0.722063in}}%
\pgfpathlineto{\pgfqpoint{1.688770in}{0.735083in}}%
\pgfpathlineto{\pgfqpoint{1.714597in}{0.753265in}}%
\pgfpathlineto{\pgfqpoint{1.739591in}{0.745252in}}%
\pgfpathlineto{\pgfqpoint{1.765418in}{0.749143in}}%
\pgfpathlineto{\pgfqpoint{1.791246in}{0.740884in}}%
\pgfpathlineto{\pgfqpoint{1.816240in}{0.725955in}}%
\pgfpathlineto{\pgfqpoint{1.842067in}{0.863931in}}%
\pgfpathlineto{\pgfqpoint{1.867061in}{0.874439in}}%
\pgfpathlineto{\pgfqpoint{1.892888in}{0.740735in}}%
\pgfpathlineto{\pgfqpoint{1.918716in}{0.750673in}}%
\pgfpathlineto{\pgfqpoint{1.942043in}{0.802223in}}%
\pgfpathlineto{\pgfqpoint{1.967871in}{0.800021in}}%
\pgfpathlineto{\pgfqpoint{1.992865in}{0.745493in}}%
\pgfpathlineto{\pgfqpoint{2.018692in}{0.764319in}}%
\pgfpathlineto{\pgfqpoint{2.043686in}{0.791019in}}%
\pgfpathlineto{\pgfqpoint{2.069513in}{0.774891in}}%
\pgfpathlineto{\pgfqpoint{2.095341in}{0.782130in}}%
\pgfpathlineto{\pgfqpoint{2.120335in}{0.767975in}}%
\pgfpathlineto{\pgfqpoint{2.146162in}{0.796843in}}%
\pgfpathlineto{\pgfqpoint{2.171156in}{0.763796in}}%
\pgfpathlineto{\pgfqpoint{2.196983in}{0.791545in}}%
\pgfpathlineto{\pgfqpoint{2.222811in}{0.848296in}}%
\pgfpathlineto{\pgfqpoint{2.246972in}{0.812207in}}%
\pgfpathlineto{\pgfqpoint{2.272799in}{0.861234in}}%
\pgfpathlineto{\pgfqpoint{2.297793in}{0.858438in}}%
\pgfpathlineto{\pgfqpoint{2.323620in}{0.813848in}}%
\pgfpathlineto{\pgfqpoint{2.348614in}{0.826567in}}%
\pgfpathlineto{\pgfqpoint{2.374442in}{0.842331in}}%
\pgfpathlineto{\pgfqpoint{2.400269in}{0.821376in}}%
\pgfpathlineto{\pgfqpoint{2.425263in}{0.815761in}}%
\pgfpathlineto{\pgfqpoint{2.451090in}{0.838265in}}%
\pgfpathlineto{\pgfqpoint{2.476084in}{0.833985in}}%
\pgfpathlineto{\pgfqpoint{2.501911in}{0.860236in}}%
\pgfpathlineto{\pgfqpoint{2.527739in}{0.909303in}}%
\pgfpathlineto{\pgfqpoint{2.551067in}{0.923652in}}%
\pgfpathlineto{\pgfqpoint{2.576894in}{0.840975in}}%
\pgfpathlineto{\pgfqpoint{2.601888in}{0.874120in}}%
\pgfpathlineto{\pgfqpoint{2.627715in}{0.847836in}}%
\pgfpathlineto{\pgfqpoint{2.652709in}{0.880977in}}%
\pgfpathlineto{\pgfqpoint{2.678537in}{0.860722in}}%
\pgfpathlineto{\pgfqpoint{2.704364in}{0.836806in}}%
\pgfpathlineto{\pgfqpoint{2.729358in}{0.841857in}}%
\pgfpathlineto{\pgfqpoint{2.755185in}{0.831210in}}%
\pgfpathlineto{\pgfqpoint{2.780179in}{0.835816in}}%
\pgfpathlineto{\pgfqpoint{2.806006in}{0.844927in}}%
\pgfpathlineto{\pgfqpoint{2.831834in}{0.865696in}}%
\pgfpathlineto{\pgfqpoint{2.880989in}{0.879366in}}%
\pgfpathlineto{\pgfqpoint{2.905983in}{0.990795in}}%
\pgfpathlineto{\pgfqpoint{2.931810in}{0.896070in}}%
\pgfpathlineto{\pgfqpoint{2.956804in}{0.855937in}}%
\pgfpathlineto{\pgfqpoint{2.982631in}{0.868508in}}%
\pgfpathlineto{\pgfqpoint{3.008459in}{0.855910in}}%
\pgfpathlineto{\pgfqpoint{3.033453in}{0.888311in}}%
\pgfpathlineto{\pgfqpoint{3.059280in}{0.880024in}}%
\pgfpathlineto{\pgfqpoint{3.084274in}{0.929458in}}%
\pgfpathlineto{\pgfqpoint{3.110101in}{0.935980in}}%
\pgfpathlineto{\pgfqpoint{3.135929in}{0.933153in}}%
\pgfpathlineto{\pgfqpoint{3.159257in}{0.902601in}}%
\pgfpathlineto{\pgfqpoint{3.210078in}{1.026339in}}%
\pgfpathlineto{\pgfqpoint{3.235905in}{1.019126in}}%
\pgfpathlineto{\pgfqpoint{3.260899in}{1.065207in}}%
\pgfpathlineto{\pgfqpoint{3.286726in}{1.078130in}}%
\pgfpathlineto{\pgfqpoint{3.312554in}{0.977157in}}%
\pgfpathlineto{\pgfqpoint{3.337548in}{0.989560in}}%
\pgfpathlineto{\pgfqpoint{3.363375in}{0.952196in}}%
\pgfpathlineto{\pgfqpoint{3.414196in}{1.014190in}}%
\pgfpathlineto{\pgfqpoint{3.440024in}{0.992405in}}%
\pgfpathlineto{\pgfqpoint{3.464185in}{1.058491in}}%
\pgfpathlineto{\pgfqpoint{3.490012in}{1.005022in}}%
\pgfpathlineto{\pgfqpoint{3.515006in}{1.016292in}}%
\pgfpathlineto{\pgfqpoint{3.540833in}{1.130902in}}%
\pgfpathlineto{\pgfqpoint{3.565827in}{1.023114in}}%
\pgfpathlineto{\pgfqpoint{3.591655in}{0.969071in}}%
\pgfpathlineto{\pgfqpoint{3.617482in}{1.043033in}}%
\pgfpathlineto{\pgfqpoint{3.642476in}{0.987722in}}%
\pgfpathlineto{\pgfqpoint{3.668303in}{0.979245in}}%
\pgfpathlineto{\pgfqpoint{3.693297in}{0.996489in}}%
\pgfpathlineto{\pgfqpoint{3.719125in}{1.034160in}}%
\pgfpathlineto{\pgfqpoint{3.744952in}{1.015897in}}%
\pgfpathlineto{\pgfqpoint{3.768280in}{1.011910in}}%
\pgfpathlineto{\pgfqpoint{3.794107in}{1.110369in}}%
\pgfpathlineto{\pgfqpoint{3.819101in}{1.058322in}}%
\pgfpathlineto{\pgfqpoint{3.844928in}{1.061378in}}%
\pgfpathlineto{\pgfqpoint{3.869922in}{1.040368in}}%
\pgfpathlineto{\pgfqpoint{3.895750in}{1.068500in}}%
\pgfpathlineto{\pgfqpoint{3.921577in}{1.133118in}}%
\pgfpathlineto{\pgfqpoint{3.946571in}{1.047998in}}%
\pgfpathlineto{\pgfqpoint{3.997392in}{1.194659in}}%
\pgfpathlineto{\pgfqpoint{4.023219in}{1.118654in}}%
\pgfpathlineto{\pgfqpoint{4.049047in}{1.173735in}}%
\pgfpathlineto{\pgfqpoint{4.072375in}{1.127366in}}%
\pgfpathlineto{\pgfqpoint{4.098202in}{1.081615in}}%
\pgfpathlineto{\pgfqpoint{4.123196in}{1.095320in}}%
\pgfpathlineto{\pgfqpoint{4.149023in}{1.118168in}}%
\pgfpathlineto{\pgfqpoint{4.174017in}{1.143376in}}%
\pgfpathlineto{\pgfqpoint{4.199845in}{1.199619in}}%
\pgfpathlineto{\pgfqpoint{4.225672in}{1.181848in}}%
\pgfpathlineto{\pgfqpoint{4.250666in}{1.193261in}}%
\pgfpathlineto{\pgfqpoint{4.276493in}{1.335021in}}%
\pgfpathlineto{\pgfqpoint{4.301487in}{1.356245in}}%
\pgfpathlineto{\pgfqpoint{4.327314in}{1.360694in}}%
\pgfpathlineto{\pgfqpoint{4.353142in}{1.349688in}}%
\pgfpathlineto{\pgfqpoint{4.376470in}{1.276104in}}%
\pgfpathlineto{\pgfqpoint{4.402297in}{1.310067in}}%
\pgfpathlineto{\pgfqpoint{4.427291in}{1.306585in}}%
\pgfpathlineto{\pgfqpoint{4.453118in}{1.376477in}}%
\pgfpathlineto{\pgfqpoint{4.478112in}{1.387647in}}%
\pgfpathlineto{\pgfqpoint{4.503939in}{1.461117in}}%
\pgfpathlineto{\pgfqpoint{4.529767in}{1.482943in}}%
\pgfpathlineto{\pgfqpoint{4.554761in}{1.475040in}}%
\pgfpathlineto{\pgfqpoint{4.580588in}{1.481890in}}%
\pgfpathlineto{\pgfqpoint{4.605582in}{1.518370in}}%
\pgfpathlineto{\pgfqpoint{4.631409in}{1.449772in}}%
\pgfpathlineto{\pgfqpoint{4.657237in}{1.507661in}}%
\pgfpathlineto{\pgfqpoint{4.681398in}{1.567168in}}%
\pgfpathlineto{\pgfqpoint{4.707225in}{1.703663in}}%
\pgfpathlineto{\pgfqpoint{4.732219in}{1.725322in}}%
\pgfpathlineto{\pgfqpoint{4.758046in}{1.704473in}}%
\pgfpathlineto{\pgfqpoint{4.783040in}{1.605327in}}%
\pgfpathlineto{\pgfqpoint{4.808868in}{1.700800in}}%
\pgfpathlineto{\pgfqpoint{4.834695in}{1.761391in}}%
\pgfpathlineto{\pgfqpoint{4.859689in}{1.714551in}}%
\pgfpathlineto{\pgfqpoint{4.885516in}{1.717340in}}%
\pgfpathlineto{\pgfqpoint{4.910510in}{1.757050in}}%
\pgfpathlineto{\pgfqpoint{4.936338in}{1.785981in}}%
\pgfpathlineto{\pgfqpoint{4.962165in}{1.599014in}}%
\pgfpathlineto{\pgfqpoint{4.985493in}{1.582921in}}%
\pgfpathlineto{\pgfqpoint{5.011320in}{1.541692in}}%
\pgfpathlineto{\pgfqpoint{5.036314in}{1.338661in}}%
\pgfpathlineto{\pgfqpoint{5.062141in}{1.282174in}}%
\pgfpathlineto{\pgfqpoint{5.087135in}{1.206590in}}%
\pgfpathlineto{\pgfqpoint{5.112963in}{1.171540in}}%
\pgfpathlineto{\pgfqpoint{5.138790in}{1.096571in}}%
\pgfpathlineto{\pgfqpoint{5.163784in}{1.075068in}}%
\pgfpathlineto{\pgfqpoint{5.189611in}{0.940331in}}%
\pgfpathlineto{\pgfqpoint{5.214605in}{0.787408in}}%
\pgfpathlineto{\pgfqpoint{5.214605in}{0.787408in}}%
\pgfusepath{stroke}%
\end{pgfscope}%
\begin{pgfscope}%
\pgfsetrectcap%
\pgfsetmiterjoin%
\pgfsetlinewidth{0.803000pt}%
\definecolor{currentstroke}{rgb}{0.000000,0.000000,0.000000}%
\pgfsetstrokecolor{currentstroke}%
\pgfsetdash{}{0pt}%
\pgfpathmoveto{\pgfqpoint{0.553704in}{0.650826in}}%
\pgfpathlineto{\pgfqpoint{0.553704in}{1.840036in}}%
\pgfusepath{stroke}%
\end{pgfscope}%
\begin{pgfscope}%
\pgfsetrectcap%
\pgfsetmiterjoin%
\pgfsetlinewidth{0.803000pt}%
\definecolor{currentstroke}{rgb}{0.000000,0.000000,0.000000}%
\pgfsetstrokecolor{currentstroke}%
\pgfsetdash{}{0pt}%
\pgfpathmoveto{\pgfqpoint{5.436553in}{0.650826in}}%
\pgfpathlineto{\pgfqpoint{5.436553in}{1.840036in}}%
\pgfusepath{stroke}%
\end{pgfscope}%
\begin{pgfscope}%
\pgfsetrectcap%
\pgfsetmiterjoin%
\pgfsetlinewidth{0.803000pt}%
\definecolor{currentstroke}{rgb}{0.000000,0.000000,0.000000}%
\pgfsetstrokecolor{currentstroke}%
\pgfsetdash{}{0pt}%
\pgfpathmoveto{\pgfqpoint{0.553704in}{0.650826in}}%
\pgfpathlineto{\pgfqpoint{5.436553in}{0.650826in}}%
\pgfusepath{stroke}%
\end{pgfscope}%
\begin{pgfscope}%
\pgfsetrectcap%
\pgfsetmiterjoin%
\pgfsetlinewidth{0.803000pt}%
\definecolor{currentstroke}{rgb}{0.000000,0.000000,0.000000}%
\pgfsetstrokecolor{currentstroke}%
\pgfsetdash{}{0pt}%
\pgfpathmoveto{\pgfqpoint{0.553704in}{1.840036in}}%
\pgfpathlineto{\pgfqpoint{5.436553in}{1.840036in}}%
\pgfusepath{stroke}%
\end{pgfscope}%
\begin{pgfscope}%
\pgfsetbuttcap%
\pgfsetmiterjoin%
\pgfsetlinewidth{0.000000pt}%
\definecolor{currentstroke}{rgb}{0.800000,0.800000,0.800000}%
\pgfsetstrokecolor{currentstroke}%
\pgfsetstrokeopacity{0.000000}%
\pgfsetdash{}{0pt}%
\pgfpathmoveto{\pgfqpoint{1.648239in}{0.100000in}}%
\pgfpathlineto{\pgfqpoint{4.184507in}{0.100000in}}%
\pgfpathquadraticcurveto{\pgfqpoint{4.212285in}{0.100000in}}{\pgfqpoint{4.212285in}{0.127778in}}%
\pgfpathlineto{\pgfqpoint{4.212285in}{0.307562in}}%
\pgfpathquadraticcurveto{\pgfqpoint{4.212285in}{0.335339in}}{\pgfqpoint{4.184507in}{0.335339in}}%
\pgfpathlineto{\pgfqpoint{1.648239in}{0.335339in}}%
\pgfpathquadraticcurveto{\pgfqpoint{1.620461in}{0.335339in}}{\pgfqpoint{1.620461in}{0.307562in}}%
\pgfpathlineto{\pgfqpoint{1.620461in}{0.127778in}}%
\pgfpathquadraticcurveto{\pgfqpoint{1.620461in}{0.100000in}}{\pgfqpoint{1.648239in}{0.100000in}}%
\pgfpathclose%
\pgfusepath{}%
\end{pgfscope}%
\begin{pgfscope}%
\pgfsetrectcap%
\pgfsetroundjoin%
\pgfsetlinewidth{1.505625pt}%
\definecolor{currentstroke}{rgb}{0.121569,0.466667,0.705882}%
\pgfsetstrokecolor{currentstroke}%
\pgfsetdash{}{0pt}%
\pgfpathmoveto{\pgfqpoint{1.676017in}{0.231173in}}%
\pgfpathlineto{\pgfqpoint{1.953795in}{0.231173in}}%
\pgfusepath{stroke}%
\end{pgfscope}%
\begin{pgfscope}%
\definecolor{textcolor}{rgb}{0.000000,0.000000,0.000000}%
\pgfsetstrokecolor{textcolor}%
\pgfsetfillcolor{textcolor}%
\pgftext[x=2.064906in,y=0.182562in,left,base]{\color{textcolor}\rmfamily\fontsize{10.000000}{12.000000}\selectfont Videók száma}%
\end{pgfscope}%
\begin{pgfscope}%
\pgfsetrectcap%
\pgfsetroundjoin%
\pgfsetlinewidth{1.505625pt}%
\definecolor{currentstroke}{rgb}{1.000000,0.549020,0.000000}%
\pgfsetstrokecolor{currentstroke}%
\pgfsetdash{}{0pt}%
\pgfpathmoveto{\pgfqpoint{3.184506in}{0.231173in}}%
\pgfpathlineto{\pgfqpoint{3.462284in}{0.231173in}}%
\pgfusepath{stroke}%
\end{pgfscope}%
\begin{pgfscope}%
\definecolor{textcolor}{rgb}{0.000000,0.000000,0.000000}%
\pgfsetstrokecolor{textcolor}%
\pgfsetfillcolor{textcolor}%
\pgftext[x=3.573395in,y=0.182562in,left,base]{\color{textcolor}\rmfamily\fontsize{10.000000}{12.000000}\selectfont Nézettség}%
\end{pgfscope}%
\end{pgfpicture}%
\makeatother%
\endgroup%

    \end{center}
\end{figure}

Mivel a dolgozatomban szükséges a bizonyos kategóriába tartozó videók vizsgálata is, ezért megfigyeltem az amatőr, a homoszexuális és a hardcore kategóriába tartozó videók adatait is (\ref{video.cat.series}. ábra). Látszódik, hogy az amatőr kategóriában jóval több tartalom került feltöltésre, mint a másik két kategóriában. Ugyanakkor mindegyik kategóriáról elmondható, hogy igazán csak 2016-ot követően nőtt meg az adott kategóriába tartozó videók száma.

\begin{figure}[h]
    \caption[A pornográf videók száma kategóriánkként]{\footnotesize{A PornHub weboldalon megjelenő videók száma kategóriánkként. Forrás: saját ábra}}
    \label{video.cat.series}
    \begin{center}
        %% Creator: Matplotlib, PGF backend
%%
%% To include the figure in your LaTeX document, write
%%   \input{<filename>.pgf}
%%
%% Make sure the required packages are loaded in your preamble
%%   \usepackage{pgf}
%%
%% Figures using additional raster images can only be included by \input if
%% they are in the same directory as the main LaTeX file. For loading figures
%% from other directories you can use the `import` package
%%   \usepackage{import}
%%
%% and then include the figures with
%%   \import{<path to file>}{<filename>.pgf}
%%
%% Matplotlib used the following preamble
%%
\begingroup%
\makeatletter%
\begin{pgfpicture}%
\pgfpathrectangle{\pgfpointorigin}{\pgfqpoint{5.428528in}{2.086271in}}%
\pgfusepath{use as bounding box, clip}%
\begin{pgfscope}%
\pgfsetbuttcap%
\pgfsetmiterjoin%
\pgfsetlinewidth{0.000000pt}%
\definecolor{currentstroke}{rgb}{1.000000,1.000000,1.000000}%
\pgfsetstrokecolor{currentstroke}%
\pgfsetstrokeopacity{0.000000}%
\pgfsetdash{}{0pt}%
\pgfpathmoveto{\pgfqpoint{0.000000in}{0.000000in}}%
\pgfpathlineto{\pgfqpoint{5.428528in}{0.000000in}}%
\pgfpathlineto{\pgfqpoint{5.428528in}{2.086271in}}%
\pgfpathlineto{\pgfqpoint{0.000000in}{2.086271in}}%
\pgfpathclose%
\pgfusepath{}%
\end{pgfscope}%
\begin{pgfscope}%
\pgfsetbuttcap%
\pgfsetmiterjoin%
\definecolor{currentfill}{rgb}{1.000000,1.000000,1.000000}%
\pgfsetfillcolor{currentfill}%
\pgfsetlinewidth{0.000000pt}%
\definecolor{currentstroke}{rgb}{0.000000,0.000000,0.000000}%
\pgfsetstrokecolor{currentstroke}%
\pgfsetstrokeopacity{0.000000}%
\pgfsetdash{}{0pt}%
\pgfpathmoveto{\pgfqpoint{0.445679in}{0.650826in}}%
\pgfpathlineto{\pgfqpoint{5.328528in}{0.650826in}}%
\pgfpathlineto{\pgfqpoint{5.328528in}{1.840036in}}%
\pgfpathlineto{\pgfqpoint{0.445679in}{1.840036in}}%
\pgfpathclose%
\pgfusepath{fill}%
\end{pgfscope}%
\begin{pgfscope}%
\pgfsetbuttcap%
\pgfsetroundjoin%
\definecolor{currentfill}{rgb}{0.000000,0.000000,0.000000}%
\pgfsetfillcolor{currentfill}%
\pgfsetlinewidth{0.803000pt}%
\definecolor{currentstroke}{rgb}{0.000000,0.000000,0.000000}%
\pgfsetstrokecolor{currentstroke}%
\pgfsetdash{}{0pt}%
\pgfsys@defobject{currentmarker}{\pgfqpoint{0.000000in}{-0.048611in}}{\pgfqpoint{0.000000in}{0.000000in}}{%
\pgfpathmoveto{\pgfqpoint{0.000000in}{0.000000in}}%
\pgfpathlineto{\pgfqpoint{0.000000in}{-0.048611in}}%
\pgfusepath{stroke,fill}%
}%
\begin{pgfscope}%
\pgfsys@transformshift{0.847800in}{0.650826in}%
\pgfsys@useobject{currentmarker}{}%
\end{pgfscope}%
\end{pgfscope}%
\begin{pgfscope}%
\definecolor{textcolor}{rgb}{0.000000,0.000000,0.000000}%
\pgfsetstrokecolor{textcolor}%
\pgfsetfillcolor{textcolor}%
\pgftext[x=0.847800in,y=0.553603in,,top]{\color{textcolor}\rmfamily\fontsize{10.000000}{12.000000}\selectfont \(\displaystyle {2008}\)}%
\end{pgfscope}%
\begin{pgfscope}%
\pgfsetbuttcap%
\pgfsetroundjoin%
\definecolor{currentfill}{rgb}{0.000000,0.000000,0.000000}%
\pgfsetfillcolor{currentfill}%
\pgfsetlinewidth{0.803000pt}%
\definecolor{currentstroke}{rgb}{0.000000,0.000000,0.000000}%
\pgfsetstrokecolor{currentstroke}%
\pgfsetdash{}{0pt}%
\pgfsys@defobject{currentmarker}{\pgfqpoint{0.000000in}{-0.048611in}}{\pgfqpoint{0.000000in}{0.000000in}}{%
\pgfpathmoveto{\pgfqpoint{0.000000in}{0.000000in}}%
\pgfpathlineto{\pgfqpoint{0.000000in}{-0.048611in}}%
\pgfusepath{stroke,fill}%
}%
\begin{pgfscope}%
\pgfsys@transformshift{1.460387in}{0.650826in}%
\pgfsys@useobject{currentmarker}{}%
\end{pgfscope}%
\end{pgfscope}%
\begin{pgfscope}%
\definecolor{textcolor}{rgb}{0.000000,0.000000,0.000000}%
\pgfsetstrokecolor{textcolor}%
\pgfsetfillcolor{textcolor}%
\pgftext[x=1.460387in,y=0.553603in,,top]{\color{textcolor}\rmfamily\fontsize{10.000000}{12.000000}\selectfont \(\displaystyle {2010}\)}%
\end{pgfscope}%
\begin{pgfscope}%
\pgfsetbuttcap%
\pgfsetroundjoin%
\definecolor{currentfill}{rgb}{0.000000,0.000000,0.000000}%
\pgfsetfillcolor{currentfill}%
\pgfsetlinewidth{0.803000pt}%
\definecolor{currentstroke}{rgb}{0.000000,0.000000,0.000000}%
\pgfsetstrokecolor{currentstroke}%
\pgfsetdash{}{0pt}%
\pgfsys@defobject{currentmarker}{\pgfqpoint{0.000000in}{-0.048611in}}{\pgfqpoint{0.000000in}{0.000000in}}{%
\pgfpathmoveto{\pgfqpoint{0.000000in}{0.000000in}}%
\pgfpathlineto{\pgfqpoint{0.000000in}{-0.048611in}}%
\pgfusepath{stroke,fill}%
}%
\begin{pgfscope}%
\pgfsys@transformshift{2.072136in}{0.650826in}%
\pgfsys@useobject{currentmarker}{}%
\end{pgfscope}%
\end{pgfscope}%
\begin{pgfscope}%
\definecolor{textcolor}{rgb}{0.000000,0.000000,0.000000}%
\pgfsetstrokecolor{textcolor}%
\pgfsetfillcolor{textcolor}%
\pgftext[x=2.072136in,y=0.553603in,,top]{\color{textcolor}\rmfamily\fontsize{10.000000}{12.000000}\selectfont \(\displaystyle {2012}\)}%
\end{pgfscope}%
\begin{pgfscope}%
\pgfsetbuttcap%
\pgfsetroundjoin%
\definecolor{currentfill}{rgb}{0.000000,0.000000,0.000000}%
\pgfsetfillcolor{currentfill}%
\pgfsetlinewidth{0.803000pt}%
\definecolor{currentstroke}{rgb}{0.000000,0.000000,0.000000}%
\pgfsetstrokecolor{currentstroke}%
\pgfsetdash{}{0pt}%
\pgfsys@defobject{currentmarker}{\pgfqpoint{0.000000in}{-0.048611in}}{\pgfqpoint{0.000000in}{0.000000in}}{%
\pgfpathmoveto{\pgfqpoint{0.000000in}{0.000000in}}%
\pgfpathlineto{\pgfqpoint{0.000000in}{-0.048611in}}%
\pgfusepath{stroke,fill}%
}%
\begin{pgfscope}%
\pgfsys@transformshift{2.684723in}{0.650826in}%
\pgfsys@useobject{currentmarker}{}%
\end{pgfscope}%
\end{pgfscope}%
\begin{pgfscope}%
\definecolor{textcolor}{rgb}{0.000000,0.000000,0.000000}%
\pgfsetstrokecolor{textcolor}%
\pgfsetfillcolor{textcolor}%
\pgftext[x=2.684723in,y=0.553603in,,top]{\color{textcolor}\rmfamily\fontsize{10.000000}{12.000000}\selectfont \(\displaystyle {2014}\)}%
\end{pgfscope}%
\begin{pgfscope}%
\pgfsetbuttcap%
\pgfsetroundjoin%
\definecolor{currentfill}{rgb}{0.000000,0.000000,0.000000}%
\pgfsetfillcolor{currentfill}%
\pgfsetlinewidth{0.803000pt}%
\definecolor{currentstroke}{rgb}{0.000000,0.000000,0.000000}%
\pgfsetstrokecolor{currentstroke}%
\pgfsetdash{}{0pt}%
\pgfsys@defobject{currentmarker}{\pgfqpoint{0.000000in}{-0.048611in}}{\pgfqpoint{0.000000in}{0.000000in}}{%
\pgfpathmoveto{\pgfqpoint{0.000000in}{0.000000in}}%
\pgfpathlineto{\pgfqpoint{0.000000in}{-0.048611in}}%
\pgfusepath{stroke,fill}%
}%
\begin{pgfscope}%
\pgfsys@transformshift{3.296473in}{0.650826in}%
\pgfsys@useobject{currentmarker}{}%
\end{pgfscope}%
\end{pgfscope}%
\begin{pgfscope}%
\definecolor{textcolor}{rgb}{0.000000,0.000000,0.000000}%
\pgfsetstrokecolor{textcolor}%
\pgfsetfillcolor{textcolor}%
\pgftext[x=3.296473in,y=0.553603in,,top]{\color{textcolor}\rmfamily\fontsize{10.000000}{12.000000}\selectfont \(\displaystyle {2016}\)}%
\end{pgfscope}%
\begin{pgfscope}%
\pgfsetbuttcap%
\pgfsetroundjoin%
\definecolor{currentfill}{rgb}{0.000000,0.000000,0.000000}%
\pgfsetfillcolor{currentfill}%
\pgfsetlinewidth{0.803000pt}%
\definecolor{currentstroke}{rgb}{0.000000,0.000000,0.000000}%
\pgfsetstrokecolor{currentstroke}%
\pgfsetdash{}{0pt}%
\pgfsys@defobject{currentmarker}{\pgfqpoint{0.000000in}{-0.048611in}}{\pgfqpoint{0.000000in}{0.000000in}}{%
\pgfpathmoveto{\pgfqpoint{0.000000in}{0.000000in}}%
\pgfpathlineto{\pgfqpoint{0.000000in}{-0.048611in}}%
\pgfusepath{stroke,fill}%
}%
\begin{pgfscope}%
\pgfsys@transformshift{3.909060in}{0.650826in}%
\pgfsys@useobject{currentmarker}{}%
\end{pgfscope}%
\end{pgfscope}%
\begin{pgfscope}%
\definecolor{textcolor}{rgb}{0.000000,0.000000,0.000000}%
\pgfsetstrokecolor{textcolor}%
\pgfsetfillcolor{textcolor}%
\pgftext[x=3.909060in,y=0.553603in,,top]{\color{textcolor}\rmfamily\fontsize{10.000000}{12.000000}\selectfont \(\displaystyle {2018}\)}%
\end{pgfscope}%
\begin{pgfscope}%
\pgfsetbuttcap%
\pgfsetroundjoin%
\definecolor{currentfill}{rgb}{0.000000,0.000000,0.000000}%
\pgfsetfillcolor{currentfill}%
\pgfsetlinewidth{0.803000pt}%
\definecolor{currentstroke}{rgb}{0.000000,0.000000,0.000000}%
\pgfsetstrokecolor{currentstroke}%
\pgfsetdash{}{0pt}%
\pgfsys@defobject{currentmarker}{\pgfqpoint{0.000000in}{-0.048611in}}{\pgfqpoint{0.000000in}{0.000000in}}{%
\pgfpathmoveto{\pgfqpoint{0.000000in}{0.000000in}}%
\pgfpathlineto{\pgfqpoint{0.000000in}{-0.048611in}}%
\pgfusepath{stroke,fill}%
}%
\begin{pgfscope}%
\pgfsys@transformshift{4.520809in}{0.650826in}%
\pgfsys@useobject{currentmarker}{}%
\end{pgfscope}%
\end{pgfscope}%
\begin{pgfscope}%
\definecolor{textcolor}{rgb}{0.000000,0.000000,0.000000}%
\pgfsetstrokecolor{textcolor}%
\pgfsetfillcolor{textcolor}%
\pgftext[x=4.520809in,y=0.553603in,,top]{\color{textcolor}\rmfamily\fontsize{10.000000}{12.000000}\selectfont \(\displaystyle {2020}\)}%
\end{pgfscope}%
\begin{pgfscope}%
\pgfsetbuttcap%
\pgfsetroundjoin%
\definecolor{currentfill}{rgb}{0.000000,0.000000,0.000000}%
\pgfsetfillcolor{currentfill}%
\pgfsetlinewidth{0.803000pt}%
\definecolor{currentstroke}{rgb}{0.000000,0.000000,0.000000}%
\pgfsetstrokecolor{currentstroke}%
\pgfsetdash{}{0pt}%
\pgfsys@defobject{currentmarker}{\pgfqpoint{0.000000in}{-0.048611in}}{\pgfqpoint{0.000000in}{0.000000in}}{%
\pgfpathmoveto{\pgfqpoint{0.000000in}{0.000000in}}%
\pgfpathlineto{\pgfqpoint{0.000000in}{-0.048611in}}%
\pgfusepath{stroke,fill}%
}%
\begin{pgfscope}%
\pgfsys@transformshift{5.133397in}{0.650826in}%
\pgfsys@useobject{currentmarker}{}%
\end{pgfscope}%
\end{pgfscope}%
\begin{pgfscope}%
\definecolor{textcolor}{rgb}{0.000000,0.000000,0.000000}%
\pgfsetstrokecolor{textcolor}%
\pgfsetfillcolor{textcolor}%
\pgftext[x=5.133397in,y=0.553603in,,top]{\color{textcolor}\rmfamily\fontsize{10.000000}{12.000000}\selectfont \(\displaystyle {2022}\)}%
\end{pgfscope}%
\begin{pgfscope}%
\pgfsetbuttcap%
\pgfsetroundjoin%
\definecolor{currentfill}{rgb}{0.000000,0.000000,0.000000}%
\pgfsetfillcolor{currentfill}%
\pgfsetlinewidth{0.803000pt}%
\definecolor{currentstroke}{rgb}{0.000000,0.000000,0.000000}%
\pgfsetstrokecolor{currentstroke}%
\pgfsetdash{}{0pt}%
\pgfsys@defobject{currentmarker}{\pgfqpoint{-0.048611in}{0.000000in}}{\pgfqpoint{-0.000000in}{0.000000in}}{%
\pgfpathmoveto{\pgfqpoint{-0.000000in}{0.000000in}}%
\pgfpathlineto{\pgfqpoint{-0.048611in}{0.000000in}}%
\pgfusepath{stroke,fill}%
}%
\begin{pgfscope}%
\pgfsys@transformshift{0.445679in}{0.704869in}%
\pgfsys@useobject{currentmarker}{}%
\end{pgfscope}%
\end{pgfscope}%
\begin{pgfscope}%
\definecolor{textcolor}{rgb}{0.000000,0.000000,0.000000}%
\pgfsetstrokecolor{textcolor}%
\pgfsetfillcolor{textcolor}%
\pgftext[x=0.279012in, y=0.656644in, left, base]{\color{textcolor}\rmfamily\fontsize{10.000000}{12.000000}\selectfont \(\displaystyle {0}\)}%
\end{pgfscope}%
\begin{pgfscope}%
\pgfsetbuttcap%
\pgfsetroundjoin%
\definecolor{currentfill}{rgb}{0.000000,0.000000,0.000000}%
\pgfsetfillcolor{currentfill}%
\pgfsetlinewidth{0.803000pt}%
\definecolor{currentstroke}{rgb}{0.000000,0.000000,0.000000}%
\pgfsetstrokecolor{currentstroke}%
\pgfsetdash{}{0pt}%
\pgfsys@defobject{currentmarker}{\pgfqpoint{-0.048611in}{0.000000in}}{\pgfqpoint{-0.000000in}{0.000000in}}{%
\pgfpathmoveto{\pgfqpoint{-0.000000in}{0.000000in}}%
\pgfpathlineto{\pgfqpoint{-0.048611in}{0.000000in}}%
\pgfusepath{stroke,fill}%
}%
\begin{pgfscope}%
\pgfsys@transformshift{0.445679in}{1.272686in}%
\pgfsys@useobject{currentmarker}{}%
\end{pgfscope}%
\end{pgfscope}%
\begin{pgfscope}%
\definecolor{textcolor}{rgb}{0.000000,0.000000,0.000000}%
\pgfsetstrokecolor{textcolor}%
\pgfsetfillcolor{textcolor}%
\pgftext[x=0.279012in, y=1.224461in, left, base]{\color{textcolor}\rmfamily\fontsize{10.000000}{12.000000}\selectfont \(\displaystyle {5}\)}%
\end{pgfscope}%
\begin{pgfscope}%
\definecolor{textcolor}{rgb}{0.000000,0.000000,0.000000}%
\pgfsetstrokecolor{textcolor}%
\pgfsetfillcolor{textcolor}%
\pgftext[x=0.223457in,y=1.245431in,,bottom,rotate=90.000000]{\color{textcolor}\rmfamily\fontsize{10.000000}{12.000000}\selectfont Videók száma}%
\end{pgfscope}%
\begin{pgfscope}%
\definecolor{textcolor}{rgb}{0.000000,0.000000,0.000000}%
\pgfsetstrokecolor{textcolor}%
\pgfsetfillcolor{textcolor}%
\pgftext[x=0.445679in,y=1.881702in,left,base]{\color{textcolor}\rmfamily\fontsize{10.000000}{12.000000}\selectfont \(\displaystyle \times{10^{4}}{}\)}%
\end{pgfscope}%
\begin{pgfscope}%
\pgfpathrectangle{\pgfqpoint{0.445679in}{0.650826in}}{\pgfqpoint{4.882849in}{1.189210in}}%
\pgfusepath{clip}%
\pgfsetrectcap%
\pgfsetroundjoin%
\pgfsetlinewidth{1.505625pt}%
\definecolor{currentstroke}{rgb}{0.121569,0.466667,0.705882}%
\pgfsetstrokecolor{currentstroke}%
\pgfsetdash{}{0pt}%
\pgfpathmoveto{\pgfqpoint{0.667627in}{0.704881in}}%
\pgfpathlineto{\pgfqpoint{1.688326in}{0.705346in}}%
\pgfpathlineto{\pgfqpoint{1.714305in}{0.706970in}}%
\pgfpathlineto{\pgfqpoint{1.739445in}{0.705630in}}%
\pgfpathlineto{\pgfqpoint{1.865985in}{0.706084in}}%
\pgfpathlineto{\pgfqpoint{1.969061in}{0.707356in}}%
\pgfpathlineto{\pgfqpoint{1.994201in}{0.708765in}}%
\pgfpathlineto{\pgfqpoint{2.071298in}{0.707992in}}%
\pgfpathlineto{\pgfqpoint{2.121579in}{0.709832in}}%
\pgfpathlineto{\pgfqpoint{2.172698in}{0.709923in}}%
\pgfpathlineto{\pgfqpoint{2.198676in}{0.708560in}}%
\pgfpathlineto{\pgfqpoint{2.249795in}{0.710070in}}%
\pgfpathlineto{\pgfqpoint{2.352032in}{0.710354in}}%
\pgfpathlineto{\pgfqpoint{2.378011in}{0.714431in}}%
\pgfpathlineto{\pgfqpoint{2.403989in}{0.715681in}}%
\pgfpathlineto{\pgfqpoint{2.427454in}{0.713387in}}%
\pgfpathlineto{\pgfqpoint{2.453432in}{0.717554in}}%
\pgfpathlineto{\pgfqpoint{2.478572in}{0.714102in}}%
\pgfpathlineto{\pgfqpoint{2.504551in}{0.712319in}}%
\pgfpathlineto{\pgfqpoint{2.529691in}{0.713375in}}%
\pgfpathlineto{\pgfqpoint{2.555670in}{0.712240in}}%
\pgfpathlineto{\pgfqpoint{2.581648in}{0.715987in}}%
\pgfpathlineto{\pgfqpoint{2.606788in}{0.715283in}}%
\pgfpathlineto{\pgfqpoint{2.632767in}{0.712251in}}%
\pgfpathlineto{\pgfqpoint{2.657907in}{0.718292in}}%
\pgfpathlineto{\pgfqpoint{2.683885in}{0.711774in}}%
\pgfpathlineto{\pgfqpoint{2.709864in}{0.712364in}}%
\pgfpathlineto{\pgfqpoint{2.733328in}{0.711297in}}%
\pgfpathlineto{\pgfqpoint{2.759307in}{0.715419in}}%
\pgfpathlineto{\pgfqpoint{2.861544in}{0.719258in}}%
\pgfpathlineto{\pgfqpoint{2.887523in}{0.718281in}}%
\pgfpathlineto{\pgfqpoint{2.938641in}{0.721166in}}%
\pgfpathlineto{\pgfqpoint{2.963782in}{0.720359in}}%
\pgfpathlineto{\pgfqpoint{2.989760in}{0.721643in}}%
\pgfpathlineto{\pgfqpoint{3.015738in}{0.719496in}}%
\pgfpathlineto{\pgfqpoint{3.039203in}{0.719519in}}%
\pgfpathlineto{\pgfqpoint{3.065181in}{0.722915in}}%
\pgfpathlineto{\pgfqpoint{3.090322in}{0.722585in}}%
\pgfpathlineto{\pgfqpoint{3.141440in}{0.727253in}}%
\pgfpathlineto{\pgfqpoint{3.167419in}{0.732283in}}%
\pgfpathlineto{\pgfqpoint{3.193397in}{0.727730in}}%
\pgfpathlineto{\pgfqpoint{3.218538in}{0.727343in}}%
\pgfpathlineto{\pgfqpoint{3.269656in}{0.730807in}}%
\pgfpathlineto{\pgfqpoint{3.295635in}{0.733987in}}%
\pgfpathlineto{\pgfqpoint{3.321613in}{0.735565in}}%
\pgfpathlineto{\pgfqpoint{3.345916in}{0.734657in}}%
\pgfpathlineto{\pgfqpoint{3.371894in}{0.737382in}}%
\pgfpathlineto{\pgfqpoint{3.397034in}{0.737848in}}%
\pgfpathlineto{\pgfqpoint{3.423013in}{0.742550in}}%
\pgfpathlineto{\pgfqpoint{3.448153in}{0.738813in}}%
\pgfpathlineto{\pgfqpoint{3.474131in}{0.740471in}}%
\pgfpathlineto{\pgfqpoint{3.500110in}{0.745922in}}%
\pgfpathlineto{\pgfqpoint{3.525250in}{0.746888in}}%
\pgfpathlineto{\pgfqpoint{3.551229in}{0.745230in}}%
\pgfpathlineto{\pgfqpoint{3.576369in}{0.747580in}}%
\pgfpathlineto{\pgfqpoint{3.602347in}{0.757154in}}%
\pgfpathlineto{\pgfqpoint{3.628326in}{0.759936in}}%
\pgfpathlineto{\pgfqpoint{3.651790in}{0.758369in}}%
\pgfpathlineto{\pgfqpoint{3.677769in}{0.766773in}}%
\pgfpathlineto{\pgfqpoint{3.702909in}{0.777527in}}%
\pgfpathlineto{\pgfqpoint{3.754028in}{0.785397in}}%
\pgfpathlineto{\pgfqpoint{3.780006in}{0.782649in}}%
\pgfpathlineto{\pgfqpoint{3.805984in}{0.790371in}}%
\pgfpathlineto{\pgfqpoint{3.831125in}{0.785772in}}%
\pgfpathlineto{\pgfqpoint{3.857103in}{0.789553in}}%
\pgfpathlineto{\pgfqpoint{3.882244in}{0.808473in}}%
\pgfpathlineto{\pgfqpoint{3.908222in}{0.814492in}}%
\pgfpathlineto{\pgfqpoint{3.934200in}{0.826530in}}%
\pgfpathlineto{\pgfqpoint{3.957665in}{0.822089in}}%
\pgfpathlineto{\pgfqpoint{3.983643in}{0.830618in}}%
\pgfpathlineto{\pgfqpoint{4.034762in}{0.869672in}}%
\pgfpathlineto{\pgfqpoint{4.059902in}{0.873409in}}%
\pgfpathlineto{\pgfqpoint{4.085881in}{0.879087in}}%
\pgfpathlineto{\pgfqpoint{4.111859in}{0.900028in}}%
\pgfpathlineto{\pgfqpoint{4.137000in}{0.913599in}}%
\pgfpathlineto{\pgfqpoint{4.162978in}{0.921980in}}%
\pgfpathlineto{\pgfqpoint{4.188118in}{0.934755in}}%
\pgfpathlineto{\pgfqpoint{4.214097in}{0.966019in}}%
\pgfpathlineto{\pgfqpoint{4.240075in}{0.989220in}}%
\pgfpathlineto{\pgfqpoint{4.263539in}{0.990515in}}%
\pgfpathlineto{\pgfqpoint{4.289518in}{1.060595in}}%
\pgfpathlineto{\pgfqpoint{4.314658in}{1.057063in}}%
\pgfpathlineto{\pgfqpoint{4.340637in}{1.104896in}}%
\pgfpathlineto{\pgfqpoint{4.365777in}{1.121329in}}%
\pgfpathlineto{\pgfqpoint{4.391755in}{1.181744in}}%
\pgfpathlineto{\pgfqpoint{4.417734in}{1.196757in}}%
\pgfpathlineto{\pgfqpoint{4.442874in}{1.193850in}}%
\pgfpathlineto{\pgfqpoint{4.468853in}{1.220208in}}%
\pgfpathlineto{\pgfqpoint{4.493993in}{1.220344in}}%
\pgfpathlineto{\pgfqpoint{4.519971in}{1.243000in}}%
\pgfpathlineto{\pgfqpoint{4.545950in}{1.296262in}}%
\pgfpathlineto{\pgfqpoint{4.570252in}{1.275411in}}%
\pgfpathlineto{\pgfqpoint{4.596230in}{1.377664in}}%
\pgfpathlineto{\pgfqpoint{4.621371in}{1.458214in}}%
\pgfpathlineto{\pgfqpoint{4.647349in}{1.445506in}}%
\pgfpathlineto{\pgfqpoint{4.672490in}{1.415412in}}%
\pgfpathlineto{\pgfqpoint{4.698468in}{1.457544in}}%
\pgfpathlineto{\pgfqpoint{4.724446in}{1.483085in}}%
\pgfpathlineto{\pgfqpoint{4.749587in}{1.459645in}}%
\pgfpathlineto{\pgfqpoint{4.775565in}{1.498427in}}%
\pgfpathlineto{\pgfqpoint{4.800706in}{1.522979in}}%
\pgfpathlineto{\pgfqpoint{4.826684in}{1.567031in}}%
\pgfpathlineto{\pgfqpoint{4.852662in}{1.474442in}}%
\pgfpathlineto{\pgfqpoint{4.876127in}{1.399093in}}%
\pgfpathlineto{\pgfqpoint{4.902105in}{1.471365in}}%
\pgfpathlineto{\pgfqpoint{4.927246in}{1.438193in}}%
\pgfpathlineto{\pgfqpoint{4.953224in}{1.508227in}}%
\pgfpathlineto{\pgfqpoint{4.978364in}{1.537084in}}%
\pgfpathlineto{\pgfqpoint{5.004343in}{1.651385in}}%
\pgfpathlineto{\pgfqpoint{5.030321in}{1.697390in}}%
\pgfpathlineto{\pgfqpoint{5.055461in}{1.763824in}}%
\pgfpathlineto{\pgfqpoint{5.081440in}{1.782040in}}%
\pgfpathlineto{\pgfqpoint{5.106580in}{1.785981in}}%
\pgfpathlineto{\pgfqpoint{5.106580in}{1.785981in}}%
\pgfusepath{stroke}%
\end{pgfscope}%
\begin{pgfscope}%
\pgfpathrectangle{\pgfqpoint{0.445679in}{0.650826in}}{\pgfqpoint{4.882849in}{1.189210in}}%
\pgfusepath{clip}%
\pgfsetrectcap%
\pgfsetroundjoin%
\pgfsetlinewidth{1.505625pt}%
\definecolor{currentstroke}{rgb}{1.000000,0.498039,0.054902}%
\pgfsetstrokecolor{currentstroke}%
\pgfsetdash{}{0pt}%
\pgfpathmoveto{\pgfqpoint{0.923221in}{0.704926in}}%
\pgfpathlineto{\pgfqpoint{1.637208in}{0.705596in}}%
\pgfpathlineto{\pgfqpoint{2.172698in}{0.711059in}}%
\pgfpathlineto{\pgfqpoint{2.198676in}{0.707811in}}%
\pgfpathlineto{\pgfqpoint{2.223816in}{0.710900in}}%
\pgfpathlineto{\pgfqpoint{2.275773in}{0.710797in}}%
\pgfpathlineto{\pgfqpoint{2.300914in}{0.709435in}}%
\pgfpathlineto{\pgfqpoint{2.352032in}{0.709537in}}%
\pgfpathlineto{\pgfqpoint{2.378011in}{0.719428in}}%
\pgfpathlineto{\pgfqpoint{2.403989in}{0.723335in}}%
\pgfpathlineto{\pgfqpoint{2.427454in}{0.716339in}}%
\pgfpathlineto{\pgfqpoint{2.453432in}{0.712262in}}%
\pgfpathlineto{\pgfqpoint{2.478572in}{0.716952in}}%
\pgfpathlineto{\pgfqpoint{2.504551in}{0.718224in}}%
\pgfpathlineto{\pgfqpoint{2.529691in}{0.725549in}}%
\pgfpathlineto{\pgfqpoint{2.555670in}{0.717191in}}%
\pgfpathlineto{\pgfqpoint{2.581648in}{0.719065in}}%
\pgfpathlineto{\pgfqpoint{2.606788in}{0.715192in}}%
\pgfpathlineto{\pgfqpoint{2.632767in}{0.714034in}}%
\pgfpathlineto{\pgfqpoint{2.657907in}{0.717293in}}%
\pgfpathlineto{\pgfqpoint{2.683885in}{0.713318in}}%
\pgfpathlineto{\pgfqpoint{2.709864in}{0.715135in}}%
\pgfpathlineto{\pgfqpoint{2.733328in}{0.714227in}}%
\pgfpathlineto{\pgfqpoint{2.759307in}{0.714783in}}%
\pgfpathlineto{\pgfqpoint{2.784447in}{0.713773in}}%
\pgfpathlineto{\pgfqpoint{2.810425in}{0.718168in}}%
\pgfpathlineto{\pgfqpoint{2.835566in}{0.719349in}}%
\pgfpathlineto{\pgfqpoint{2.861544in}{0.718361in}}%
\pgfpathlineto{\pgfqpoint{2.887523in}{0.721745in}}%
\pgfpathlineto{\pgfqpoint{2.912663in}{0.719916in}}%
\pgfpathlineto{\pgfqpoint{2.963782in}{0.718224in}}%
\pgfpathlineto{\pgfqpoint{3.039203in}{0.714727in}}%
\pgfpathlineto{\pgfqpoint{3.090322in}{0.718747in}}%
\pgfpathlineto{\pgfqpoint{3.116300in}{0.714477in}}%
\pgfpathlineto{\pgfqpoint{3.141440in}{0.716816in}}%
\pgfpathlineto{\pgfqpoint{3.167419in}{0.735304in}}%
\pgfpathlineto{\pgfqpoint{3.193397in}{0.717123in}}%
\pgfpathlineto{\pgfqpoint{3.218538in}{0.716646in}}%
\pgfpathlineto{\pgfqpoint{3.244516in}{0.717906in}}%
\pgfpathlineto{\pgfqpoint{3.269656in}{0.722699in}}%
\pgfpathlineto{\pgfqpoint{3.295635in}{0.718327in}}%
\pgfpathlineto{\pgfqpoint{3.321613in}{0.720019in}}%
\pgfpathlineto{\pgfqpoint{3.345916in}{0.717952in}}%
\pgfpathlineto{\pgfqpoint{3.397034in}{0.721813in}}%
\pgfpathlineto{\pgfqpoint{3.423013in}{0.724913in}}%
\pgfpathlineto{\pgfqpoint{3.448153in}{0.722971in}}%
\pgfpathlineto{\pgfqpoint{3.474131in}{0.735395in}}%
\pgfpathlineto{\pgfqpoint{3.525250in}{0.719019in}}%
\pgfpathlineto{\pgfqpoint{3.551229in}{0.725027in}}%
\pgfpathlineto{\pgfqpoint{3.576369in}{0.720121in}}%
\pgfpathlineto{\pgfqpoint{3.628326in}{0.722710in}}%
\pgfpathlineto{\pgfqpoint{3.651790in}{0.724584in}}%
\pgfpathlineto{\pgfqpoint{3.677769in}{0.738143in}}%
\pgfpathlineto{\pgfqpoint{3.702909in}{0.728014in}}%
\pgfpathlineto{\pgfqpoint{3.728887in}{0.728547in}}%
\pgfpathlineto{\pgfqpoint{3.754028in}{0.725174in}}%
\pgfpathlineto{\pgfqpoint{3.780006in}{0.724845in}}%
\pgfpathlineto{\pgfqpoint{3.805984in}{0.735895in}}%
\pgfpathlineto{\pgfqpoint{3.831125in}{0.738972in}}%
\pgfpathlineto{\pgfqpoint{3.857103in}{0.732795in}}%
\pgfpathlineto{\pgfqpoint{3.882244in}{0.733726in}}%
\pgfpathlineto{\pgfqpoint{3.908222in}{0.738416in}}%
\pgfpathlineto{\pgfqpoint{3.934200in}{0.740642in}}%
\pgfpathlineto{\pgfqpoint{3.957665in}{0.737099in}}%
\pgfpathlineto{\pgfqpoint{4.008784in}{0.738859in}}%
\pgfpathlineto{\pgfqpoint{4.034762in}{0.742561in}}%
\pgfpathlineto{\pgfqpoint{4.059902in}{0.742288in}}%
\pgfpathlineto{\pgfqpoint{4.085881in}{0.746229in}}%
\pgfpathlineto{\pgfqpoint{4.137000in}{0.747444in}}%
\pgfpathlineto{\pgfqpoint{4.162978in}{0.753134in}}%
\pgfpathlineto{\pgfqpoint{4.188118in}{0.750465in}}%
\pgfpathlineto{\pgfqpoint{4.214097in}{0.760504in}}%
\pgfpathlineto{\pgfqpoint{4.240075in}{0.757903in}}%
\pgfpathlineto{\pgfqpoint{4.263539in}{0.766852in}}%
\pgfpathlineto{\pgfqpoint{4.289518in}{0.773450in}}%
\pgfpathlineto{\pgfqpoint{4.314658in}{0.777255in}}%
\pgfpathlineto{\pgfqpoint{4.340637in}{0.777811in}}%
\pgfpathlineto{\pgfqpoint{4.365777in}{0.770736in}}%
\pgfpathlineto{\pgfqpoint{4.391755in}{0.785670in}}%
\pgfpathlineto{\pgfqpoint{4.442874in}{0.784886in}}%
\pgfpathlineto{\pgfqpoint{4.468853in}{0.788838in}}%
\pgfpathlineto{\pgfqpoint{4.519971in}{0.788236in}}%
\pgfpathlineto{\pgfqpoint{4.545950in}{0.794732in}}%
\pgfpathlineto{\pgfqpoint{4.570252in}{0.788906in}}%
\pgfpathlineto{\pgfqpoint{4.596230in}{0.811051in}}%
\pgfpathlineto{\pgfqpoint{4.621371in}{0.821113in}}%
\pgfpathlineto{\pgfqpoint{4.647349in}{0.822725in}}%
\pgfpathlineto{\pgfqpoint{4.672490in}{0.821885in}}%
\pgfpathlineto{\pgfqpoint{4.698468in}{0.826507in}}%
\pgfpathlineto{\pgfqpoint{4.724446in}{0.824633in}}%
\pgfpathlineto{\pgfqpoint{4.749587in}{0.825633in}}%
\pgfpathlineto{\pgfqpoint{4.775565in}{0.833275in}}%
\pgfpathlineto{\pgfqpoint{4.800706in}{0.833661in}}%
\pgfpathlineto{\pgfqpoint{4.826684in}{0.844723in}}%
\pgfpathlineto{\pgfqpoint{4.852662in}{0.834945in}}%
\pgfpathlineto{\pgfqpoint{4.876127in}{0.828256in}}%
\pgfpathlineto{\pgfqpoint{4.902105in}{0.838318in}}%
\pgfpathlineto{\pgfqpoint{4.927246in}{0.835376in}}%
\pgfpathlineto{\pgfqpoint{4.953224in}{0.840907in}}%
\pgfpathlineto{\pgfqpoint{4.978364in}{0.842894in}}%
\pgfpathlineto{\pgfqpoint{5.004343in}{0.862529in}}%
\pgfpathlineto{\pgfqpoint{5.030321in}{0.863438in}}%
\pgfpathlineto{\pgfqpoint{5.055461in}{0.867571in}}%
\pgfpathlineto{\pgfqpoint{5.081440in}{0.858975in}}%
\pgfpathlineto{\pgfqpoint{5.106580in}{0.858168in}}%
\pgfpathlineto{\pgfqpoint{5.106580in}{0.858168in}}%
\pgfusepath{stroke}%
\end{pgfscope}%
\begin{pgfscope}%
\pgfpathrectangle{\pgfqpoint{0.445679in}{0.650826in}}{\pgfqpoint{4.882849in}{1.189210in}}%
\pgfusepath{clip}%
\pgfsetrectcap%
\pgfsetroundjoin%
\pgfsetlinewidth{1.505625pt}%
\definecolor{currentstroke}{rgb}{0.172549,0.627451,0.172549}%
\pgfsetstrokecolor{currentstroke}%
\pgfsetdash{}{0pt}%
\pgfpathmoveto{\pgfqpoint{0.846962in}{0.704881in}}%
\pgfpathlineto{\pgfqpoint{1.688326in}{0.705312in}}%
\pgfpathlineto{\pgfqpoint{1.714305in}{0.707936in}}%
\pgfpathlineto{\pgfqpoint{1.739445in}{0.708765in}}%
\pgfpathlineto{\pgfqpoint{1.765424in}{0.705789in}}%
\pgfpathlineto{\pgfqpoint{2.352032in}{0.708799in}}%
\pgfpathlineto{\pgfqpoint{2.378011in}{0.715805in}}%
\pgfpathlineto{\pgfqpoint{2.403989in}{0.718588in}}%
\pgfpathlineto{\pgfqpoint{2.427454in}{0.713216in}}%
\pgfpathlineto{\pgfqpoint{2.453432in}{0.709662in}}%
\pgfpathlineto{\pgfqpoint{2.478572in}{0.710513in}}%
\pgfpathlineto{\pgfqpoint{2.504551in}{0.708288in}}%
\pgfpathlineto{\pgfqpoint{2.529691in}{0.708106in}}%
\pgfpathlineto{\pgfqpoint{2.555670in}{0.709775in}}%
\pgfpathlineto{\pgfqpoint{2.581648in}{0.708151in}}%
\pgfpathlineto{\pgfqpoint{2.683885in}{0.708424in}}%
\pgfpathlineto{\pgfqpoint{2.709864in}{0.733612in}}%
\pgfpathlineto{\pgfqpoint{2.733328in}{0.731659in}}%
\pgfpathlineto{\pgfqpoint{2.784447in}{0.733658in}}%
\pgfpathlineto{\pgfqpoint{2.810425in}{0.728002in}}%
\pgfpathlineto{\pgfqpoint{2.835566in}{0.730853in}}%
\pgfpathlineto{\pgfqpoint{2.861544in}{0.729649in}}%
\pgfpathlineto{\pgfqpoint{2.912663in}{0.731875in}}%
\pgfpathlineto{\pgfqpoint{2.938641in}{0.723403in}}%
\pgfpathlineto{\pgfqpoint{2.963782in}{0.727877in}}%
\pgfpathlineto{\pgfqpoint{2.989760in}{0.721620in}}%
\pgfpathlineto{\pgfqpoint{3.015738in}{0.708980in}}%
\pgfpathlineto{\pgfqpoint{3.065181in}{0.710752in}}%
\pgfpathlineto{\pgfqpoint{3.090322in}{0.713500in}}%
\pgfpathlineto{\pgfqpoint{3.116300in}{0.712421in}}%
\pgfpathlineto{\pgfqpoint{3.141440in}{0.714340in}}%
\pgfpathlineto{\pgfqpoint{3.167419in}{0.722687in}}%
\pgfpathlineto{\pgfqpoint{3.193397in}{0.713296in}}%
\pgfpathlineto{\pgfqpoint{3.218538in}{0.718610in}}%
\pgfpathlineto{\pgfqpoint{3.244516in}{0.715499in}}%
\pgfpathlineto{\pgfqpoint{3.269656in}{0.724345in}}%
\pgfpathlineto{\pgfqpoint{3.295635in}{0.713977in}}%
\pgfpathlineto{\pgfqpoint{3.321613in}{0.712853in}}%
\pgfpathlineto{\pgfqpoint{3.371894in}{0.726219in}}%
\pgfpathlineto{\pgfqpoint{3.397034in}{0.721518in}}%
\pgfpathlineto{\pgfqpoint{3.500110in}{0.716839in}}%
\pgfpathlineto{\pgfqpoint{3.525250in}{0.720234in}}%
\pgfpathlineto{\pgfqpoint{3.551229in}{0.717975in}}%
\pgfpathlineto{\pgfqpoint{3.602347in}{0.719746in}}%
\pgfpathlineto{\pgfqpoint{3.677769in}{0.718679in}}%
\pgfpathlineto{\pgfqpoint{3.728887in}{0.721745in}}%
\pgfpathlineto{\pgfqpoint{3.754028in}{0.728150in}}%
\pgfpathlineto{\pgfqpoint{3.780006in}{0.726276in}}%
\pgfpathlineto{\pgfqpoint{3.805984in}{0.744469in}}%
\pgfpathlineto{\pgfqpoint{3.831125in}{0.729319in}}%
\pgfpathlineto{\pgfqpoint{3.857103in}{0.736076in}}%
\pgfpathlineto{\pgfqpoint{3.882244in}{0.737507in}}%
\pgfpathlineto{\pgfqpoint{3.934200in}{0.737973in}}%
\pgfpathlineto{\pgfqpoint{3.957665in}{0.741618in}}%
\pgfpathlineto{\pgfqpoint{3.983643in}{0.740415in}}%
\pgfpathlineto{\pgfqpoint{4.008784in}{0.746638in}}%
\pgfpathlineto{\pgfqpoint{4.059902in}{0.747172in}}%
\pgfpathlineto{\pgfqpoint{4.085881in}{0.743663in}}%
\pgfpathlineto{\pgfqpoint{4.111859in}{0.746536in}}%
\pgfpathlineto{\pgfqpoint{4.137000in}{0.751294in}}%
\pgfpathlineto{\pgfqpoint{4.162978in}{0.761651in}}%
\pgfpathlineto{\pgfqpoint{4.188118in}{0.758108in}}%
\pgfpathlineto{\pgfqpoint{4.214097in}{0.758062in}}%
\pgfpathlineto{\pgfqpoint{4.240075in}{0.765512in}}%
\pgfpathlineto{\pgfqpoint{4.263539in}{0.766852in}}%
\pgfpathlineto{\pgfqpoint{4.289518in}{0.776369in}}%
\pgfpathlineto{\pgfqpoint{4.314658in}{0.776744in}}%
\pgfpathlineto{\pgfqpoint{4.340637in}{0.783398in}}%
\pgfpathlineto{\pgfqpoint{4.365777in}{0.779492in}}%
\pgfpathlineto{\pgfqpoint{4.391755in}{0.787271in}}%
\pgfpathlineto{\pgfqpoint{4.417734in}{0.792540in}}%
\pgfpathlineto{\pgfqpoint{4.442874in}{0.794573in}}%
\pgfpathlineto{\pgfqpoint{4.493993in}{0.805237in}}%
\pgfpathlineto{\pgfqpoint{4.519971in}{0.800796in}}%
\pgfpathlineto{\pgfqpoint{4.545950in}{0.805464in}}%
\pgfpathlineto{\pgfqpoint{4.570252in}{0.803624in}}%
\pgfpathlineto{\pgfqpoint{4.621371in}{0.837091in}}%
\pgfpathlineto{\pgfqpoint{4.647349in}{0.832957in}}%
\pgfpathlineto{\pgfqpoint{4.672490in}{0.822157in}}%
\pgfpathlineto{\pgfqpoint{4.698468in}{0.820363in}}%
\pgfpathlineto{\pgfqpoint{4.724446in}{0.821033in}}%
\pgfpathlineto{\pgfqpoint{4.749587in}{0.826938in}}%
\pgfpathlineto{\pgfqpoint{4.775565in}{0.825349in}}%
\pgfpathlineto{\pgfqpoint{4.800706in}{0.831015in}}%
\pgfpathlineto{\pgfqpoint{4.826684in}{0.844745in}}%
\pgfpathlineto{\pgfqpoint{4.852662in}{0.831458in}}%
\pgfpathlineto{\pgfqpoint{4.876127in}{0.826268in}}%
\pgfpathlineto{\pgfqpoint{4.902105in}{0.846358in}}%
\pgfpathlineto{\pgfqpoint{4.927246in}{0.836943in}}%
\pgfpathlineto{\pgfqpoint{4.978364in}{0.852933in}}%
\pgfpathlineto{\pgfqpoint{5.004343in}{0.868014in}}%
\pgfpathlineto{\pgfqpoint{5.030321in}{0.870115in}}%
\pgfpathlineto{\pgfqpoint{5.055461in}{0.899551in}}%
\pgfpathlineto{\pgfqpoint{5.081440in}{0.897666in}}%
\pgfpathlineto{\pgfqpoint{5.106580in}{0.909136in}}%
\pgfpathlineto{\pgfqpoint{5.106580in}{0.909136in}}%
\pgfusepath{stroke}%
\end{pgfscope}%
\begin{pgfscope}%
\pgfsetrectcap%
\pgfsetmiterjoin%
\pgfsetlinewidth{0.803000pt}%
\definecolor{currentstroke}{rgb}{0.000000,0.000000,0.000000}%
\pgfsetstrokecolor{currentstroke}%
\pgfsetdash{}{0pt}%
\pgfpathmoveto{\pgfqpoint{0.445679in}{0.650826in}}%
\pgfpathlineto{\pgfqpoint{0.445679in}{1.840036in}}%
\pgfusepath{stroke}%
\end{pgfscope}%
\begin{pgfscope}%
\pgfsetrectcap%
\pgfsetmiterjoin%
\pgfsetlinewidth{0.803000pt}%
\definecolor{currentstroke}{rgb}{0.000000,0.000000,0.000000}%
\pgfsetstrokecolor{currentstroke}%
\pgfsetdash{}{0pt}%
\pgfpathmoveto{\pgfqpoint{5.328528in}{0.650826in}}%
\pgfpathlineto{\pgfqpoint{5.328528in}{1.840036in}}%
\pgfusepath{stroke}%
\end{pgfscope}%
\begin{pgfscope}%
\pgfsetrectcap%
\pgfsetmiterjoin%
\pgfsetlinewidth{0.803000pt}%
\definecolor{currentstroke}{rgb}{0.000000,0.000000,0.000000}%
\pgfsetstrokecolor{currentstroke}%
\pgfsetdash{}{0pt}%
\pgfpathmoveto{\pgfqpoint{0.445679in}{0.650826in}}%
\pgfpathlineto{\pgfqpoint{5.328528in}{0.650826in}}%
\pgfusepath{stroke}%
\end{pgfscope}%
\begin{pgfscope}%
\pgfsetrectcap%
\pgfsetmiterjoin%
\pgfsetlinewidth{0.803000pt}%
\definecolor{currentstroke}{rgb}{0.000000,0.000000,0.000000}%
\pgfsetstrokecolor{currentstroke}%
\pgfsetdash{}{0pt}%
\pgfpathmoveto{\pgfqpoint{0.445679in}{1.840036in}}%
\pgfpathlineto{\pgfqpoint{5.328528in}{1.840036in}}%
\pgfusepath{stroke}%
\end{pgfscope}%
\begin{pgfscope}%
\pgfsetbuttcap%
\pgfsetmiterjoin%
\pgfsetlinewidth{0.000000pt}%
\definecolor{currentstroke}{rgb}{0.800000,0.800000,0.800000}%
\pgfsetstrokecolor{currentstroke}%
\pgfsetstrokeopacity{0.000000}%
\pgfsetdash{}{0pt}%
\pgfpathmoveto{\pgfqpoint{0.965175in}{0.100000in}}%
\pgfpathlineto{\pgfqpoint{4.651521in}{0.100000in}}%
\pgfpathquadraticcurveto{\pgfqpoint{4.679299in}{0.100000in}}{\pgfqpoint{4.679299in}{0.127778in}}%
\pgfpathlineto{\pgfqpoint{4.679299in}{0.307562in}}%
\pgfpathquadraticcurveto{\pgfqpoint{4.679299in}{0.335339in}}{\pgfqpoint{4.651521in}{0.335339in}}%
\pgfpathlineto{\pgfqpoint{0.965175in}{0.335339in}}%
\pgfpathquadraticcurveto{\pgfqpoint{0.937397in}{0.335339in}}{\pgfqpoint{0.937397in}{0.307562in}}%
\pgfpathlineto{\pgfqpoint{0.937397in}{0.127778in}}%
\pgfpathquadraticcurveto{\pgfqpoint{0.937397in}{0.100000in}}{\pgfqpoint{0.965175in}{0.100000in}}%
\pgfpathclose%
\pgfusepath{}%
\end{pgfscope}%
\begin{pgfscope}%
\pgfsetrectcap%
\pgfsetroundjoin%
\pgfsetlinewidth{1.505625pt}%
\definecolor{currentstroke}{rgb}{0.121569,0.466667,0.705882}%
\pgfsetstrokecolor{currentstroke}%
\pgfsetdash{}{0pt}%
\pgfpathmoveto{\pgfqpoint{0.992952in}{0.231173in}}%
\pgfpathlineto{\pgfqpoint{1.270730in}{0.231173in}}%
\pgfusepath{stroke}%
\end{pgfscope}%
\begin{pgfscope}%
\definecolor{textcolor}{rgb}{0.000000,0.000000,0.000000}%
\pgfsetstrokecolor{textcolor}%
\pgfsetfillcolor{textcolor}%
\pgftext[x=1.381841in,y=0.182562in,left,base]{\color{textcolor}\rmfamily\fontsize{10.000000}{12.000000}\selectfont Amatőr}%
\end{pgfscope}%
\begin{pgfscope}%
\pgfsetrectcap%
\pgfsetroundjoin%
\pgfsetlinewidth{1.505625pt}%
\definecolor{currentstroke}{rgb}{1.000000,0.498039,0.054902}%
\pgfsetstrokecolor{currentstroke}%
\pgfsetdash{}{0pt}%
\pgfpathmoveto{\pgfqpoint{2.126827in}{0.231173in}}%
\pgfpathlineto{\pgfqpoint{2.404605in}{0.231173in}}%
\pgfusepath{stroke}%
\end{pgfscope}%
\begin{pgfscope}%
\definecolor{textcolor}{rgb}{0.000000,0.000000,0.000000}%
\pgfsetstrokecolor{textcolor}%
\pgfsetfillcolor{textcolor}%
\pgftext[x=2.515716in,y=0.182562in,left,base]{\color{textcolor}\rmfamily\fontsize{10.000000}{12.000000}\selectfont Homoszexuális}%
\end{pgfscope}%
\begin{pgfscope}%
\pgfsetrectcap%
\pgfsetroundjoin%
\pgfsetlinewidth{1.505625pt}%
\definecolor{currentstroke}{rgb}{0.172549,0.627451,0.172549}%
\pgfsetstrokecolor{currentstroke}%
\pgfsetdash{}{0pt}%
\pgfpathmoveto{\pgfqpoint{3.682385in}{0.231173in}}%
\pgfpathlineto{\pgfqpoint{3.960162in}{0.231173in}}%
\pgfusepath{stroke}%
\end{pgfscope}%
\begin{pgfscope}%
\definecolor{textcolor}{rgb}{0.000000,0.000000,0.000000}%
\pgfsetstrokecolor{textcolor}%
\pgfsetfillcolor{textcolor}%
\pgftext[x=4.071273in,y=0.182562in,left,base]{\color{textcolor}\rmfamily\fontsize{10.000000}{12.000000}\selectfont Hardcore}%
\end{pgfscope}%
\end{pgfpicture}%
\makeatother%
\endgroup%

    \end{center}
\end{figure}

Ahhoz, hogy mérni tudjam a tényleges tartalomfogyasztást a PornHub oldalán, letöltöttem a semrush.com weboldal segítségével az oldal szerverére érkező kéréseket havi bontásban \citep{semrush_2021}. Ez azt jelenti, hogy az egyes hónapokban hány alkalommal nyitottak meg aloldalakat a weblapon, így jó mérőszám lehet a tartalom tényleges fogyasztására vonatkozóan. 
A nézettség fajlagosításához felhasználtam az amerikai népességi adatokat is \citep{fred_2021}.

\pagebreak
Mivel a kutatásomban az amerikai lakosság fogyasztására volt szükségem, ezért beszereztem az Amerikából érkező kéréseket is. Ahogy a \ref{traffic.series}. ábra is jól mutatja, az amerikai fogyasztók jelentős részét kitették az oldal teljes látogatottságának.

\begin{figure}[h]
    \caption[A PornHub forgalma]{\footnotesize{A PornHub weboldalon tapasztalható forgalom. Forrás: saját ábra}}
    \label{traffic.series}
    \begin{center}
        %% Creator: Matplotlib, PGF backend
%%
%% To include the figure in your LaTeX document, write
%%   \input{<filename>.pgf}
%%
%% Make sure the required packages are loaded in your preamble
%%   \usepackage{pgf}
%%
%% Figures using additional raster images can only be included by \input if
%% they are in the same directory as the main LaTeX file. For loading figures
%% from other directories you can use the `import` package
%%   \usepackage{import}
%%
%% and then include the figures with
%%   \import{<path to file>}{<filename>.pgf}
%%
%% Matplotlib used the following preamble
%%
\begingroup%
\makeatletter%
\begin{pgfpicture}%
\pgfpathrectangle{\pgfpointorigin}{\pgfqpoint{5.536553in}{2.088089in}}%
\pgfusepath{use as bounding box, clip}%
\begin{pgfscope}%
\pgfsetbuttcap%
\pgfsetmiterjoin%
\pgfsetlinewidth{0.000000pt}%
\definecolor{currentstroke}{rgb}{1.000000,1.000000,1.000000}%
\pgfsetstrokecolor{currentstroke}%
\pgfsetstrokeopacity{0.000000}%
\pgfsetdash{}{0pt}%
\pgfpathmoveto{\pgfqpoint{0.000000in}{0.000000in}}%
\pgfpathlineto{\pgfqpoint{5.536553in}{0.000000in}}%
\pgfpathlineto{\pgfqpoint{5.536553in}{2.088089in}}%
\pgfpathlineto{\pgfqpoint{0.000000in}{2.088089in}}%
\pgfpathclose%
\pgfusepath{}%
\end{pgfscope}%
\begin{pgfscope}%
\pgfsetbuttcap%
\pgfsetmiterjoin%
\definecolor{currentfill}{rgb}{1.000000,1.000000,1.000000}%
\pgfsetfillcolor{currentfill}%
\pgfsetlinewidth{0.000000pt}%
\definecolor{currentstroke}{rgb}{0.000000,0.000000,0.000000}%
\pgfsetstrokecolor{currentstroke}%
\pgfsetstrokeopacity{0.000000}%
\pgfsetdash{}{0pt}%
\pgfpathmoveto{\pgfqpoint{0.553704in}{0.652645in}}%
\pgfpathlineto{\pgfqpoint{5.436553in}{0.652645in}}%
\pgfpathlineto{\pgfqpoint{5.436553in}{1.841854in}}%
\pgfpathlineto{\pgfqpoint{0.553704in}{1.841854in}}%
\pgfpathclose%
\pgfusepath{fill}%
\end{pgfscope}%
\begin{pgfscope}%
\pgfsetbuttcap%
\pgfsetroundjoin%
\definecolor{currentfill}{rgb}{0.000000,0.000000,0.000000}%
\pgfsetfillcolor{currentfill}%
\pgfsetlinewidth{0.803000pt}%
\definecolor{currentstroke}{rgb}{0.000000,0.000000,0.000000}%
\pgfsetstrokecolor{currentstroke}%
\pgfsetdash{}{0pt}%
\pgfsys@defobject{currentmarker}{\pgfqpoint{0.000000in}{-0.048611in}}{\pgfqpoint{0.000000in}{0.000000in}}{%
\pgfpathmoveto{\pgfqpoint{0.000000in}{0.000000in}}%
\pgfpathlineto{\pgfqpoint{0.000000in}{-0.048611in}}%
\pgfusepath{stroke,fill}%
}%
\begin{pgfscope}%
\pgfsys@transformshift{0.775652in}{0.652645in}%
\pgfsys@useobject{currentmarker}{}%
\end{pgfscope}%
\end{pgfscope}%
\begin{pgfscope}%
\definecolor{textcolor}{rgb}{0.000000,0.000000,0.000000}%
\pgfsetstrokecolor{textcolor}%
\pgfsetfillcolor{textcolor}%
\pgftext[x=0.775652in,y=0.555422in,,top]{\color{textcolor}\rmfamily\fontsize{10.000000}{12.000000}\selectfont \(\displaystyle {2012}\)}%
\end{pgfscope}%
\begin{pgfscope}%
\pgfsetbuttcap%
\pgfsetroundjoin%
\definecolor{currentfill}{rgb}{0.000000,0.000000,0.000000}%
\pgfsetfillcolor{currentfill}%
\pgfsetlinewidth{0.803000pt}%
\definecolor{currentstroke}{rgb}{0.000000,0.000000,0.000000}%
\pgfsetstrokecolor{currentstroke}%
\pgfsetdash{}{0pt}%
\pgfsys@defobject{currentmarker}{\pgfqpoint{0.000000in}{-0.048611in}}{\pgfqpoint{0.000000in}{0.000000in}}{%
\pgfpathmoveto{\pgfqpoint{0.000000in}{0.000000in}}%
\pgfpathlineto{\pgfqpoint{0.000000in}{-0.048611in}}%
\pgfusepath{stroke,fill}%
}%
\begin{pgfscope}%
\pgfsys@transformshift{1.224204in}{0.652645in}%
\pgfsys@useobject{currentmarker}{}%
\end{pgfscope}%
\end{pgfscope}%
\begin{pgfscope}%
\definecolor{textcolor}{rgb}{0.000000,0.000000,0.000000}%
\pgfsetstrokecolor{textcolor}%
\pgfsetfillcolor{textcolor}%
\pgftext[x=1.224204in,y=0.555422in,,top]{\color{textcolor}\rmfamily\fontsize{10.000000}{12.000000}\selectfont \(\displaystyle {2013}\)}%
\end{pgfscope}%
\begin{pgfscope}%
\pgfsetbuttcap%
\pgfsetroundjoin%
\definecolor{currentfill}{rgb}{0.000000,0.000000,0.000000}%
\pgfsetfillcolor{currentfill}%
\pgfsetlinewidth{0.803000pt}%
\definecolor{currentstroke}{rgb}{0.000000,0.000000,0.000000}%
\pgfsetstrokecolor{currentstroke}%
\pgfsetdash{}{0pt}%
\pgfsys@defobject{currentmarker}{\pgfqpoint{0.000000in}{-0.048611in}}{\pgfqpoint{0.000000in}{0.000000in}}{%
\pgfpathmoveto{\pgfqpoint{0.000000in}{0.000000in}}%
\pgfpathlineto{\pgfqpoint{0.000000in}{-0.048611in}}%
\pgfusepath{stroke,fill}%
}%
\begin{pgfscope}%
\pgfsys@transformshift{1.671531in}{0.652645in}%
\pgfsys@useobject{currentmarker}{}%
\end{pgfscope}%
\end{pgfscope}%
\begin{pgfscope}%
\definecolor{textcolor}{rgb}{0.000000,0.000000,0.000000}%
\pgfsetstrokecolor{textcolor}%
\pgfsetfillcolor{textcolor}%
\pgftext[x=1.671531in,y=0.555422in,,top]{\color{textcolor}\rmfamily\fontsize{10.000000}{12.000000}\selectfont \(\displaystyle {2014}\)}%
\end{pgfscope}%
\begin{pgfscope}%
\pgfsetbuttcap%
\pgfsetroundjoin%
\definecolor{currentfill}{rgb}{0.000000,0.000000,0.000000}%
\pgfsetfillcolor{currentfill}%
\pgfsetlinewidth{0.803000pt}%
\definecolor{currentstroke}{rgb}{0.000000,0.000000,0.000000}%
\pgfsetstrokecolor{currentstroke}%
\pgfsetdash{}{0pt}%
\pgfsys@defobject{currentmarker}{\pgfqpoint{0.000000in}{-0.048611in}}{\pgfqpoint{0.000000in}{0.000000in}}{%
\pgfpathmoveto{\pgfqpoint{0.000000in}{0.000000in}}%
\pgfpathlineto{\pgfqpoint{0.000000in}{-0.048611in}}%
\pgfusepath{stroke,fill}%
}%
\begin{pgfscope}%
\pgfsys@transformshift{2.118858in}{0.652645in}%
\pgfsys@useobject{currentmarker}{}%
\end{pgfscope}%
\end{pgfscope}%
\begin{pgfscope}%
\definecolor{textcolor}{rgb}{0.000000,0.000000,0.000000}%
\pgfsetstrokecolor{textcolor}%
\pgfsetfillcolor{textcolor}%
\pgftext[x=2.118858in,y=0.555422in,,top]{\color{textcolor}\rmfamily\fontsize{10.000000}{12.000000}\selectfont \(\displaystyle {2015}\)}%
\end{pgfscope}%
\begin{pgfscope}%
\pgfsetbuttcap%
\pgfsetroundjoin%
\definecolor{currentfill}{rgb}{0.000000,0.000000,0.000000}%
\pgfsetfillcolor{currentfill}%
\pgfsetlinewidth{0.803000pt}%
\definecolor{currentstroke}{rgb}{0.000000,0.000000,0.000000}%
\pgfsetstrokecolor{currentstroke}%
\pgfsetdash{}{0pt}%
\pgfsys@defobject{currentmarker}{\pgfqpoint{0.000000in}{-0.048611in}}{\pgfqpoint{0.000000in}{0.000000in}}{%
\pgfpathmoveto{\pgfqpoint{0.000000in}{0.000000in}}%
\pgfpathlineto{\pgfqpoint{0.000000in}{-0.048611in}}%
\pgfusepath{stroke,fill}%
}%
\begin{pgfscope}%
\pgfsys@transformshift{2.566185in}{0.652645in}%
\pgfsys@useobject{currentmarker}{}%
\end{pgfscope}%
\end{pgfscope}%
\begin{pgfscope}%
\definecolor{textcolor}{rgb}{0.000000,0.000000,0.000000}%
\pgfsetstrokecolor{textcolor}%
\pgfsetfillcolor{textcolor}%
\pgftext[x=2.566185in,y=0.555422in,,top]{\color{textcolor}\rmfamily\fontsize{10.000000}{12.000000}\selectfont \(\displaystyle {2016}\)}%
\end{pgfscope}%
\begin{pgfscope}%
\pgfsetbuttcap%
\pgfsetroundjoin%
\definecolor{currentfill}{rgb}{0.000000,0.000000,0.000000}%
\pgfsetfillcolor{currentfill}%
\pgfsetlinewidth{0.803000pt}%
\definecolor{currentstroke}{rgb}{0.000000,0.000000,0.000000}%
\pgfsetstrokecolor{currentstroke}%
\pgfsetdash{}{0pt}%
\pgfsys@defobject{currentmarker}{\pgfqpoint{0.000000in}{-0.048611in}}{\pgfqpoint{0.000000in}{0.000000in}}{%
\pgfpathmoveto{\pgfqpoint{0.000000in}{0.000000in}}%
\pgfpathlineto{\pgfqpoint{0.000000in}{-0.048611in}}%
\pgfusepath{stroke,fill}%
}%
\begin{pgfscope}%
\pgfsys@transformshift{3.014737in}{0.652645in}%
\pgfsys@useobject{currentmarker}{}%
\end{pgfscope}%
\end{pgfscope}%
\begin{pgfscope}%
\definecolor{textcolor}{rgb}{0.000000,0.000000,0.000000}%
\pgfsetstrokecolor{textcolor}%
\pgfsetfillcolor{textcolor}%
\pgftext[x=3.014737in,y=0.555422in,,top]{\color{textcolor}\rmfamily\fontsize{10.000000}{12.000000}\selectfont \(\displaystyle {2017}\)}%
\end{pgfscope}%
\begin{pgfscope}%
\pgfsetbuttcap%
\pgfsetroundjoin%
\definecolor{currentfill}{rgb}{0.000000,0.000000,0.000000}%
\pgfsetfillcolor{currentfill}%
\pgfsetlinewidth{0.803000pt}%
\definecolor{currentstroke}{rgb}{0.000000,0.000000,0.000000}%
\pgfsetstrokecolor{currentstroke}%
\pgfsetdash{}{0pt}%
\pgfsys@defobject{currentmarker}{\pgfqpoint{0.000000in}{-0.048611in}}{\pgfqpoint{0.000000in}{0.000000in}}{%
\pgfpathmoveto{\pgfqpoint{0.000000in}{0.000000in}}%
\pgfpathlineto{\pgfqpoint{0.000000in}{-0.048611in}}%
\pgfusepath{stroke,fill}%
}%
\begin{pgfscope}%
\pgfsys@transformshift{3.462064in}{0.652645in}%
\pgfsys@useobject{currentmarker}{}%
\end{pgfscope}%
\end{pgfscope}%
\begin{pgfscope}%
\definecolor{textcolor}{rgb}{0.000000,0.000000,0.000000}%
\pgfsetstrokecolor{textcolor}%
\pgfsetfillcolor{textcolor}%
\pgftext[x=3.462064in,y=0.555422in,,top]{\color{textcolor}\rmfamily\fontsize{10.000000}{12.000000}\selectfont \(\displaystyle {2018}\)}%
\end{pgfscope}%
\begin{pgfscope}%
\pgfsetbuttcap%
\pgfsetroundjoin%
\definecolor{currentfill}{rgb}{0.000000,0.000000,0.000000}%
\pgfsetfillcolor{currentfill}%
\pgfsetlinewidth{0.803000pt}%
\definecolor{currentstroke}{rgb}{0.000000,0.000000,0.000000}%
\pgfsetstrokecolor{currentstroke}%
\pgfsetdash{}{0pt}%
\pgfsys@defobject{currentmarker}{\pgfqpoint{0.000000in}{-0.048611in}}{\pgfqpoint{0.000000in}{0.000000in}}{%
\pgfpathmoveto{\pgfqpoint{0.000000in}{0.000000in}}%
\pgfpathlineto{\pgfqpoint{0.000000in}{-0.048611in}}%
\pgfusepath{stroke,fill}%
}%
\begin{pgfscope}%
\pgfsys@transformshift{3.909391in}{0.652645in}%
\pgfsys@useobject{currentmarker}{}%
\end{pgfscope}%
\end{pgfscope}%
\begin{pgfscope}%
\definecolor{textcolor}{rgb}{0.000000,0.000000,0.000000}%
\pgfsetstrokecolor{textcolor}%
\pgfsetfillcolor{textcolor}%
\pgftext[x=3.909391in,y=0.555422in,,top]{\color{textcolor}\rmfamily\fontsize{10.000000}{12.000000}\selectfont \(\displaystyle {2019}\)}%
\end{pgfscope}%
\begin{pgfscope}%
\pgfsetbuttcap%
\pgfsetroundjoin%
\definecolor{currentfill}{rgb}{0.000000,0.000000,0.000000}%
\pgfsetfillcolor{currentfill}%
\pgfsetlinewidth{0.803000pt}%
\definecolor{currentstroke}{rgb}{0.000000,0.000000,0.000000}%
\pgfsetstrokecolor{currentstroke}%
\pgfsetdash{}{0pt}%
\pgfsys@defobject{currentmarker}{\pgfqpoint{0.000000in}{-0.048611in}}{\pgfqpoint{0.000000in}{0.000000in}}{%
\pgfpathmoveto{\pgfqpoint{0.000000in}{0.000000in}}%
\pgfpathlineto{\pgfqpoint{0.000000in}{-0.048611in}}%
\pgfusepath{stroke,fill}%
}%
\begin{pgfscope}%
\pgfsys@transformshift{4.356718in}{0.652645in}%
\pgfsys@useobject{currentmarker}{}%
\end{pgfscope}%
\end{pgfscope}%
\begin{pgfscope}%
\definecolor{textcolor}{rgb}{0.000000,0.000000,0.000000}%
\pgfsetstrokecolor{textcolor}%
\pgfsetfillcolor{textcolor}%
\pgftext[x=4.356718in,y=0.555422in,,top]{\color{textcolor}\rmfamily\fontsize{10.000000}{12.000000}\selectfont \(\displaystyle {2020}\)}%
\end{pgfscope}%
\begin{pgfscope}%
\pgfsetbuttcap%
\pgfsetroundjoin%
\definecolor{currentfill}{rgb}{0.000000,0.000000,0.000000}%
\pgfsetfillcolor{currentfill}%
\pgfsetlinewidth{0.803000pt}%
\definecolor{currentstroke}{rgb}{0.000000,0.000000,0.000000}%
\pgfsetstrokecolor{currentstroke}%
\pgfsetdash{}{0pt}%
\pgfsys@defobject{currentmarker}{\pgfqpoint{0.000000in}{-0.048611in}}{\pgfqpoint{0.000000in}{0.000000in}}{%
\pgfpathmoveto{\pgfqpoint{0.000000in}{0.000000in}}%
\pgfpathlineto{\pgfqpoint{0.000000in}{-0.048611in}}%
\pgfusepath{stroke,fill}%
}%
\begin{pgfscope}%
\pgfsys@transformshift{4.805271in}{0.652645in}%
\pgfsys@useobject{currentmarker}{}%
\end{pgfscope}%
\end{pgfscope}%
\begin{pgfscope}%
\definecolor{textcolor}{rgb}{0.000000,0.000000,0.000000}%
\pgfsetstrokecolor{textcolor}%
\pgfsetfillcolor{textcolor}%
\pgftext[x=4.805271in,y=0.555422in,,top]{\color{textcolor}\rmfamily\fontsize{10.000000}{12.000000}\selectfont \(\displaystyle {2021}\)}%
\end{pgfscope}%
\begin{pgfscope}%
\pgfsetbuttcap%
\pgfsetroundjoin%
\definecolor{currentfill}{rgb}{0.000000,0.000000,0.000000}%
\pgfsetfillcolor{currentfill}%
\pgfsetlinewidth{0.803000pt}%
\definecolor{currentstroke}{rgb}{0.000000,0.000000,0.000000}%
\pgfsetstrokecolor{currentstroke}%
\pgfsetdash{}{0pt}%
\pgfsys@defobject{currentmarker}{\pgfqpoint{0.000000in}{-0.048611in}}{\pgfqpoint{0.000000in}{0.000000in}}{%
\pgfpathmoveto{\pgfqpoint{0.000000in}{0.000000in}}%
\pgfpathlineto{\pgfqpoint{0.000000in}{-0.048611in}}%
\pgfusepath{stroke,fill}%
}%
\begin{pgfscope}%
\pgfsys@transformshift{5.252597in}{0.652645in}%
\pgfsys@useobject{currentmarker}{}%
\end{pgfscope}%
\end{pgfscope}%
\begin{pgfscope}%
\definecolor{textcolor}{rgb}{0.000000,0.000000,0.000000}%
\pgfsetstrokecolor{textcolor}%
\pgfsetfillcolor{textcolor}%
\pgftext[x=5.252597in,y=0.555422in,,top]{\color{textcolor}\rmfamily\fontsize{10.000000}{12.000000}\selectfont \(\displaystyle {2022}\)}%
\end{pgfscope}%
\begin{pgfscope}%
\pgfsetbuttcap%
\pgfsetroundjoin%
\definecolor{currentfill}{rgb}{0.000000,0.000000,0.000000}%
\pgfsetfillcolor{currentfill}%
\pgfsetlinewidth{0.803000pt}%
\definecolor{currentstroke}{rgb}{0.000000,0.000000,0.000000}%
\pgfsetstrokecolor{currentstroke}%
\pgfsetdash{}{0pt}%
\pgfsys@defobject{currentmarker}{\pgfqpoint{-0.048611in}{0.000000in}}{\pgfqpoint{-0.000000in}{0.000000in}}{%
\pgfpathmoveto{\pgfqpoint{-0.000000in}{0.000000in}}%
\pgfpathlineto{\pgfqpoint{-0.048611in}{0.000000in}}%
\pgfusepath{stroke,fill}%
}%
\begin{pgfscope}%
\pgfsys@transformshift{0.553704in}{0.693205in}%
\pgfsys@useobject{currentmarker}{}%
\end{pgfscope}%
\end{pgfscope}%
\begin{pgfscope}%
\definecolor{textcolor}{rgb}{0.000000,0.000000,0.000000}%
\pgfsetstrokecolor{textcolor}%
\pgfsetfillcolor{textcolor}%
\pgftext[x=0.279012in, y=0.644980in, left, base]{\color{textcolor}\rmfamily\fontsize{10.000000}{12.000000}\selectfont \(\displaystyle {0.0}\)}%
\end{pgfscope}%
\begin{pgfscope}%
\pgfsetbuttcap%
\pgfsetroundjoin%
\definecolor{currentfill}{rgb}{0.000000,0.000000,0.000000}%
\pgfsetfillcolor{currentfill}%
\pgfsetlinewidth{0.803000pt}%
\definecolor{currentstroke}{rgb}{0.000000,0.000000,0.000000}%
\pgfsetstrokecolor{currentstroke}%
\pgfsetdash{}{0pt}%
\pgfsys@defobject{currentmarker}{\pgfqpoint{-0.048611in}{0.000000in}}{\pgfqpoint{-0.000000in}{0.000000in}}{%
\pgfpathmoveto{\pgfqpoint{-0.000000in}{0.000000in}}%
\pgfpathlineto{\pgfqpoint{-0.048611in}{0.000000in}}%
\pgfusepath{stroke,fill}%
}%
\begin{pgfscope}%
\pgfsys@transformshift{0.553704in}{1.153280in}%
\pgfsys@useobject{currentmarker}{}%
\end{pgfscope}%
\end{pgfscope}%
\begin{pgfscope}%
\definecolor{textcolor}{rgb}{0.000000,0.000000,0.000000}%
\pgfsetstrokecolor{textcolor}%
\pgfsetfillcolor{textcolor}%
\pgftext[x=0.279012in, y=1.105055in, left, base]{\color{textcolor}\rmfamily\fontsize{10.000000}{12.000000}\selectfont \(\displaystyle {0.5}\)}%
\end{pgfscope}%
\begin{pgfscope}%
\pgfsetbuttcap%
\pgfsetroundjoin%
\definecolor{currentfill}{rgb}{0.000000,0.000000,0.000000}%
\pgfsetfillcolor{currentfill}%
\pgfsetlinewidth{0.803000pt}%
\definecolor{currentstroke}{rgb}{0.000000,0.000000,0.000000}%
\pgfsetstrokecolor{currentstroke}%
\pgfsetdash{}{0pt}%
\pgfsys@defobject{currentmarker}{\pgfqpoint{-0.048611in}{0.000000in}}{\pgfqpoint{-0.000000in}{0.000000in}}{%
\pgfpathmoveto{\pgfqpoint{-0.000000in}{0.000000in}}%
\pgfpathlineto{\pgfqpoint{-0.048611in}{0.000000in}}%
\pgfusepath{stroke,fill}%
}%
\begin{pgfscope}%
\pgfsys@transformshift{0.553704in}{1.613355in}%
\pgfsys@useobject{currentmarker}{}%
\end{pgfscope}%
\end{pgfscope}%
\begin{pgfscope}%
\definecolor{textcolor}{rgb}{0.000000,0.000000,0.000000}%
\pgfsetstrokecolor{textcolor}%
\pgfsetfillcolor{textcolor}%
\pgftext[x=0.279012in, y=1.565130in, left, base]{\color{textcolor}\rmfamily\fontsize{10.000000}{12.000000}\selectfont \(\displaystyle {1.0}\)}%
\end{pgfscope}%
\begin{pgfscope}%
\definecolor{textcolor}{rgb}{0.000000,0.000000,0.000000}%
\pgfsetstrokecolor{textcolor}%
\pgfsetfillcolor{textcolor}%
\pgftext[x=0.223457in,y=1.247250in,,bottom,rotate=90.000000]{\color{textcolor}\rmfamily\fontsize{10.000000}{12.000000}\selectfont Forgalom}%
\end{pgfscope}%
\begin{pgfscope}%
\definecolor{textcolor}{rgb}{0.000000,0.000000,0.000000}%
\pgfsetstrokecolor{textcolor}%
\pgfsetfillcolor{textcolor}%
\pgftext[x=0.553704in,y=1.883521in,left,base]{\color{textcolor}\rmfamily\fontsize{10.000000}{12.000000}\selectfont \(\displaystyle \times{10^{9}}{}\)}%
\end{pgfscope}%
\begin{pgfscope}%
\pgfpathrectangle{\pgfqpoint{0.553704in}{0.652645in}}{\pgfqpoint{4.882849in}{1.189210in}}%
\pgfusepath{clip}%
\pgfsetrectcap%
\pgfsetroundjoin%
\pgfsetlinewidth{1.505625pt}%
\definecolor{currentstroke}{rgb}{0.121569,0.466667,0.705882}%
\pgfsetstrokecolor{currentstroke}%
\pgfsetdash{}{0pt}%
\pgfpathmoveto{\pgfqpoint{0.775652in}{0.754863in}}%
\pgfpathlineto{\pgfqpoint{0.813644in}{0.754735in}}%
\pgfpathlineto{\pgfqpoint{0.849185in}{0.752282in}}%
\pgfpathlineto{\pgfqpoint{0.887177in}{0.750375in}}%
\pgfpathlineto{\pgfqpoint{0.923944in}{0.756885in}}%
\pgfpathlineto{\pgfqpoint{0.961936in}{0.767237in}}%
\pgfpathlineto{\pgfqpoint{0.998703in}{0.763412in}}%
\pgfpathlineto{\pgfqpoint{1.036695in}{0.755155in}}%
\pgfpathlineto{\pgfqpoint{1.074687in}{0.761104in}}%
\pgfpathlineto{\pgfqpoint{1.111453in}{0.767771in}}%
\pgfpathlineto{\pgfqpoint{1.149446in}{0.754669in}}%
\pgfpathlineto{\pgfqpoint{1.186212in}{0.757427in}}%
\pgfpathlineto{\pgfqpoint{1.224204in}{0.755957in}}%
\pgfpathlineto{\pgfqpoint{1.262196in}{0.749638in}}%
\pgfpathlineto{\pgfqpoint{1.296512in}{0.749115in}}%
\pgfpathlineto{\pgfqpoint{1.334504in}{0.747531in}}%
\pgfpathlineto{\pgfqpoint{1.371271in}{0.742198in}}%
\pgfpathlineto{\pgfqpoint{1.409263in}{0.743293in}}%
\pgfpathlineto{\pgfqpoint{1.446029in}{0.742268in}}%
\pgfpathlineto{\pgfqpoint{1.484022in}{0.731570in}}%
\pgfpathlineto{\pgfqpoint{1.522014in}{0.732903in}}%
\pgfpathlineto{\pgfqpoint{1.558780in}{0.733667in}}%
\pgfpathlineto{\pgfqpoint{1.596772in}{0.795304in}}%
\pgfpathlineto{\pgfqpoint{1.633539in}{0.859523in}}%
\pgfpathlineto{\pgfqpoint{1.671531in}{0.858627in}}%
\pgfpathlineto{\pgfqpoint{1.709523in}{0.866218in}}%
\pgfpathlineto{\pgfqpoint{1.743839in}{0.825932in}}%
\pgfpathlineto{\pgfqpoint{1.781831in}{0.811108in}}%
\pgfpathlineto{\pgfqpoint{1.818598in}{0.809839in}}%
\pgfpathlineto{\pgfqpoint{1.856590in}{0.831598in}}%
\pgfpathlineto{\pgfqpoint{1.893356in}{0.825225in}}%
\pgfpathlineto{\pgfqpoint{1.931348in}{0.839093in}}%
\pgfpathlineto{\pgfqpoint{1.969341in}{0.843374in}}%
\pgfpathlineto{\pgfqpoint{2.006107in}{0.842522in}}%
\pgfpathlineto{\pgfqpoint{2.044099in}{0.871786in}}%
\pgfpathlineto{\pgfqpoint{2.080866in}{0.888584in}}%
\pgfpathlineto{\pgfqpoint{2.118858in}{0.889683in}}%
\pgfpathlineto{\pgfqpoint{2.156850in}{0.879736in}}%
\pgfpathlineto{\pgfqpoint{2.191166in}{0.880705in}}%
\pgfpathlineto{\pgfqpoint{2.229158in}{0.878057in}}%
\pgfpathlineto{\pgfqpoint{2.265924in}{0.879355in}}%
\pgfpathlineto{\pgfqpoint{2.303917in}{0.890873in}}%
\pgfpathlineto{\pgfqpoint{2.340683in}{0.897848in}}%
\pgfpathlineto{\pgfqpoint{2.378675in}{0.898130in}}%
\pgfpathlineto{\pgfqpoint{2.416667in}{0.898500in}}%
\pgfpathlineto{\pgfqpoint{2.453434in}{0.897931in}}%
\pgfpathlineto{\pgfqpoint{2.491426in}{0.910716in}}%
\pgfpathlineto{\pgfqpoint{2.528193in}{0.933425in}}%
\pgfpathlineto{\pgfqpoint{2.566185in}{0.940512in}}%
\pgfpathlineto{\pgfqpoint{2.604177in}{0.960250in}}%
\pgfpathlineto{\pgfqpoint{2.639718in}{0.960335in}}%
\pgfpathlineto{\pgfqpoint{2.677710in}{0.974063in}}%
\pgfpathlineto{\pgfqpoint{2.714477in}{0.973540in}}%
\pgfpathlineto{\pgfqpoint{2.752469in}{0.962463in}}%
\pgfpathlineto{\pgfqpoint{2.789236in}{0.950680in}}%
\pgfpathlineto{\pgfqpoint{2.827228in}{0.947572in}}%
\pgfpathlineto{\pgfqpoint{2.865220in}{0.946050in}}%
\pgfpathlineto{\pgfqpoint{2.901987in}{0.967726in}}%
\pgfpathlineto{\pgfqpoint{2.939979in}{0.970696in}}%
\pgfpathlineto{\pgfqpoint{2.976745in}{1.081037in}}%
\pgfpathlineto{\pgfqpoint{3.014737in}{1.087003in}}%
\pgfpathlineto{\pgfqpoint{3.052730in}{1.067590in}}%
\pgfpathlineto{\pgfqpoint{3.087045in}{1.044152in}}%
\pgfpathlineto{\pgfqpoint{3.125037in}{1.077192in}}%
\pgfpathlineto{\pgfqpoint{3.161804in}{1.098229in}}%
\pgfpathlineto{\pgfqpoint{3.199796in}{1.081795in}}%
\pgfpathlineto{\pgfqpoint{3.236563in}{1.063649in}}%
\pgfpathlineto{\pgfqpoint{3.274555in}{1.074944in}}%
\pgfpathlineto{\pgfqpoint{3.312547in}{1.072775in}}%
\pgfpathlineto{\pgfqpoint{3.349313in}{1.056093in}}%
\pgfpathlineto{\pgfqpoint{3.387306in}{1.077505in}}%
\pgfpathlineto{\pgfqpoint{3.424072in}{1.078156in}}%
\pgfpathlineto{\pgfqpoint{3.462064in}{1.170768in}}%
\pgfpathlineto{\pgfqpoint{3.500056in}{1.170332in}}%
\pgfpathlineto{\pgfqpoint{3.534372in}{1.257042in}}%
\pgfpathlineto{\pgfqpoint{3.572364in}{1.249500in}}%
\pgfpathlineto{\pgfqpoint{3.609131in}{1.204896in}}%
\pgfpathlineto{\pgfqpoint{3.647123in}{1.199209in}}%
\pgfpathlineto{\pgfqpoint{3.683889in}{1.188247in}}%
\pgfpathlineto{\pgfqpoint{3.721882in}{1.253398in}}%
\pgfpathlineto{\pgfqpoint{3.759874in}{1.253214in}}%
\pgfpathlineto{\pgfqpoint{3.796640in}{1.287783in}}%
\pgfpathlineto{\pgfqpoint{3.834632in}{1.272674in}}%
\pgfpathlineto{\pgfqpoint{3.871399in}{1.255173in}}%
\pgfpathlineto{\pgfqpoint{3.909391in}{1.281271in}}%
\pgfpathlineto{\pgfqpoint{3.947383in}{1.284192in}}%
\pgfpathlineto{\pgfqpoint{3.981699in}{1.233143in}}%
\pgfpathlineto{\pgfqpoint{4.019691in}{1.220539in}}%
\pgfpathlineto{\pgfqpoint{4.056458in}{1.225544in}}%
\pgfpathlineto{\pgfqpoint{4.094450in}{1.227445in}}%
\pgfpathlineto{\pgfqpoint{4.131216in}{1.203136in}}%
\pgfpathlineto{\pgfqpoint{4.169208in}{1.211888in}}%
\pgfpathlineto{\pgfqpoint{4.207201in}{1.236508in}}%
\pgfpathlineto{\pgfqpoint{4.243967in}{1.218780in}}%
\pgfpathlineto{\pgfqpoint{4.281959in}{1.271379in}}%
\pgfpathlineto{\pgfqpoint{4.318726in}{1.318882in}}%
\pgfpathlineto{\pgfqpoint{4.356718in}{1.371729in}}%
\pgfpathlineto{\pgfqpoint{4.394710in}{1.388593in}}%
\pgfpathlineto{\pgfqpoint{4.430251in}{1.415375in}}%
\pgfpathlineto{\pgfqpoint{4.468243in}{1.444437in}}%
\pgfpathlineto{\pgfqpoint{4.505010in}{1.568332in}}%
\pgfpathlineto{\pgfqpoint{4.543002in}{1.644576in}}%
\pgfpathlineto{\pgfqpoint{4.579769in}{1.683985in}}%
\pgfpathlineto{\pgfqpoint{4.617761in}{1.674230in}}%
\pgfpathlineto{\pgfqpoint{4.655753in}{1.713483in}}%
\pgfpathlineto{\pgfqpoint{4.692520in}{1.772818in}}%
\pgfpathlineto{\pgfqpoint{4.730512in}{1.764632in}}%
\pgfpathlineto{\pgfqpoint{4.767278in}{1.738619in}}%
\pgfpathlineto{\pgfqpoint{4.805271in}{1.722644in}}%
\pgfpathlineto{\pgfqpoint{4.843263in}{1.729485in}}%
\pgfpathlineto{\pgfqpoint{4.877578in}{1.706336in}}%
\pgfpathlineto{\pgfqpoint{4.915570in}{1.740163in}}%
\pgfpathlineto{\pgfqpoint{4.952337in}{1.725315in}}%
\pgfpathlineto{\pgfqpoint{4.990329in}{1.743171in}}%
\pgfpathlineto{\pgfqpoint{5.027096in}{1.701487in}}%
\pgfpathlineto{\pgfqpoint{5.065088in}{1.688185in}}%
\pgfpathlineto{\pgfqpoint{5.103080in}{1.724059in}}%
\pgfpathlineto{\pgfqpoint{5.139847in}{1.737707in}}%
\pgfpathlineto{\pgfqpoint{5.177839in}{1.757759in}}%
\pgfpathlineto{\pgfqpoint{5.214605in}{1.787799in}}%
\pgfusepath{stroke}%
\end{pgfscope}%
\begin{pgfscope}%
\pgfpathrectangle{\pgfqpoint{0.553704in}{0.652645in}}{\pgfqpoint{4.882849in}{1.189210in}}%
\pgfusepath{clip}%
\pgfsetrectcap%
\pgfsetroundjoin%
\pgfsetlinewidth{1.505625pt}%
\definecolor{currentstroke}{rgb}{1.000000,0.498039,0.054902}%
\pgfsetstrokecolor{currentstroke}%
\pgfsetdash{}{0pt}%
\pgfpathmoveto{\pgfqpoint{0.775652in}{0.715244in}}%
\pgfpathlineto{\pgfqpoint{0.813644in}{0.714549in}}%
\pgfpathlineto{\pgfqpoint{0.849185in}{0.715054in}}%
\pgfpathlineto{\pgfqpoint{0.887177in}{0.714187in}}%
\pgfpathlineto{\pgfqpoint{0.923944in}{0.716936in}}%
\pgfpathlineto{\pgfqpoint{0.961936in}{0.723366in}}%
\pgfpathlineto{\pgfqpoint{0.998703in}{0.722830in}}%
\pgfpathlineto{\pgfqpoint{1.036695in}{0.712324in}}%
\pgfpathlineto{\pgfqpoint{1.074687in}{0.719613in}}%
\pgfpathlineto{\pgfqpoint{1.111453in}{0.724924in}}%
\pgfpathlineto{\pgfqpoint{1.149446in}{0.719755in}}%
\pgfpathlineto{\pgfqpoint{1.186212in}{0.720512in}}%
\pgfpathlineto{\pgfqpoint{1.224204in}{0.720638in}}%
\pgfpathlineto{\pgfqpoint{1.262196in}{0.712699in}}%
\pgfpathlineto{\pgfqpoint{1.296512in}{0.712375in}}%
\pgfpathlineto{\pgfqpoint{1.334504in}{0.709871in}}%
\pgfpathlineto{\pgfqpoint{1.371271in}{0.711168in}}%
\pgfpathlineto{\pgfqpoint{1.409263in}{0.709658in}}%
\pgfpathlineto{\pgfqpoint{1.446029in}{0.712704in}}%
\pgfpathlineto{\pgfqpoint{1.484022in}{0.706700in}}%
\pgfpathlineto{\pgfqpoint{1.522014in}{0.706926in}}%
\pgfpathlineto{\pgfqpoint{1.558780in}{0.706845in}}%
\pgfpathlineto{\pgfqpoint{1.596772in}{0.718575in}}%
\pgfpathlineto{\pgfqpoint{1.633539in}{0.750173in}}%
\pgfpathlineto{\pgfqpoint{1.671531in}{0.752062in}}%
\pgfpathlineto{\pgfqpoint{1.709523in}{0.750796in}}%
\pgfpathlineto{\pgfqpoint{1.743839in}{0.752123in}}%
\pgfpathlineto{\pgfqpoint{1.781831in}{0.737426in}}%
\pgfpathlineto{\pgfqpoint{1.818598in}{0.738800in}}%
\pgfpathlineto{\pgfqpoint{1.856590in}{0.752072in}}%
\pgfpathlineto{\pgfqpoint{1.893356in}{0.754502in}}%
\pgfpathlineto{\pgfqpoint{1.931348in}{0.762954in}}%
\pgfpathlineto{\pgfqpoint{1.969341in}{0.765826in}}%
\pgfpathlineto{\pgfqpoint{2.006107in}{0.764780in}}%
\pgfpathlineto{\pgfqpoint{2.044099in}{0.791118in}}%
\pgfpathlineto{\pgfqpoint{2.080866in}{0.785843in}}%
\pgfpathlineto{\pgfqpoint{2.118858in}{0.788879in}}%
\pgfpathlineto{\pgfqpoint{2.156850in}{0.792678in}}%
\pgfpathlineto{\pgfqpoint{2.191166in}{0.793974in}}%
\pgfpathlineto{\pgfqpoint{2.229158in}{0.794440in}}%
\pgfpathlineto{\pgfqpoint{2.265924in}{0.800884in}}%
\pgfpathlineto{\pgfqpoint{2.303917in}{0.801776in}}%
\pgfpathlineto{\pgfqpoint{2.340683in}{0.804711in}}%
\pgfpathlineto{\pgfqpoint{2.378675in}{0.805572in}}%
\pgfpathlineto{\pgfqpoint{2.416667in}{0.806442in}}%
\pgfpathlineto{\pgfqpoint{2.453434in}{0.806750in}}%
\pgfpathlineto{\pgfqpoint{2.491426in}{0.816135in}}%
\pgfpathlineto{\pgfqpoint{2.528193in}{0.838278in}}%
\pgfpathlineto{\pgfqpoint{2.566185in}{0.840697in}}%
\pgfpathlineto{\pgfqpoint{2.604177in}{0.841438in}}%
\pgfpathlineto{\pgfqpoint{2.639718in}{0.839651in}}%
\pgfpathlineto{\pgfqpoint{2.677710in}{0.848178in}}%
\pgfpathlineto{\pgfqpoint{2.714477in}{0.847736in}}%
\pgfpathlineto{\pgfqpoint{2.752469in}{0.842259in}}%
\pgfpathlineto{\pgfqpoint{2.789236in}{0.838257in}}%
\pgfpathlineto{\pgfqpoint{2.827228in}{0.837212in}}%
\pgfpathlineto{\pgfqpoint{2.865220in}{0.838511in}}%
\pgfpathlineto{\pgfqpoint{2.901987in}{0.850203in}}%
\pgfpathlineto{\pgfqpoint{2.939979in}{0.852245in}}%
\pgfpathlineto{\pgfqpoint{2.976745in}{0.889754in}}%
\pgfpathlineto{\pgfqpoint{3.014737in}{0.892143in}}%
\pgfpathlineto{\pgfqpoint{3.052730in}{0.878065in}}%
\pgfpathlineto{\pgfqpoint{3.087045in}{0.861625in}}%
\pgfpathlineto{\pgfqpoint{3.125037in}{0.892282in}}%
\pgfpathlineto{\pgfqpoint{3.161804in}{0.894998in}}%
\pgfpathlineto{\pgfqpoint{3.199796in}{0.876744in}}%
\pgfpathlineto{\pgfqpoint{3.236563in}{0.866936in}}%
\pgfpathlineto{\pgfqpoint{3.274555in}{0.879488in}}%
\pgfpathlineto{\pgfqpoint{3.312547in}{0.876530in}}%
\pgfpathlineto{\pgfqpoint{3.349313in}{0.859486in}}%
\pgfpathlineto{\pgfqpoint{3.387306in}{0.860808in}}%
\pgfpathlineto{\pgfqpoint{3.424072in}{0.856156in}}%
\pgfpathlineto{\pgfqpoint{3.462064in}{0.893157in}}%
\pgfpathlineto{\pgfqpoint{3.500056in}{0.880186in}}%
\pgfpathlineto{\pgfqpoint{3.534372in}{0.899231in}}%
\pgfpathlineto{\pgfqpoint{3.572364in}{0.903348in}}%
\pgfpathlineto{\pgfqpoint{3.609131in}{0.882669in}}%
\pgfpathlineto{\pgfqpoint{3.647123in}{0.881086in}}%
\pgfpathlineto{\pgfqpoint{3.683889in}{0.876591in}}%
\pgfpathlineto{\pgfqpoint{3.721882in}{0.902658in}}%
\pgfpathlineto{\pgfqpoint{3.759874in}{0.903559in}}%
\pgfpathlineto{\pgfqpoint{3.796640in}{0.912494in}}%
\pgfpathlineto{\pgfqpoint{3.834632in}{0.911633in}}%
\pgfpathlineto{\pgfqpoint{3.871399in}{0.910489in}}%
\pgfpathlineto{\pgfqpoint{3.909391in}{0.915331in}}%
\pgfpathlineto{\pgfqpoint{3.947383in}{0.929673in}}%
\pgfpathlineto{\pgfqpoint{3.981699in}{0.907981in}}%
\pgfpathlineto{\pgfqpoint{4.019691in}{0.895548in}}%
\pgfpathlineto{\pgfqpoint{4.056458in}{0.895668in}}%
\pgfpathlineto{\pgfqpoint{4.094450in}{0.905336in}}%
\pgfpathlineto{\pgfqpoint{4.131216in}{0.889221in}}%
\pgfpathlineto{\pgfqpoint{4.169208in}{0.889966in}}%
\pgfpathlineto{\pgfqpoint{4.207201in}{0.901527in}}%
\pgfpathlineto{\pgfqpoint{4.243967in}{0.887733in}}%
\pgfpathlineto{\pgfqpoint{4.281959in}{0.895622in}}%
\pgfpathlineto{\pgfqpoint{4.318726in}{0.898867in}}%
\pgfpathlineto{\pgfqpoint{4.356718in}{0.916405in}}%
\pgfpathlineto{\pgfqpoint{4.394710in}{0.916545in}}%
\pgfpathlineto{\pgfqpoint{4.430251in}{0.928027in}}%
\pgfpathlineto{\pgfqpoint{4.468243in}{0.944945in}}%
\pgfpathlineto{\pgfqpoint{4.505010in}{0.972467in}}%
\pgfpathlineto{\pgfqpoint{4.543002in}{0.977800in}}%
\pgfpathlineto{\pgfqpoint{4.579769in}{0.998895in}}%
\pgfpathlineto{\pgfqpoint{4.617761in}{0.986043in}}%
\pgfpathlineto{\pgfqpoint{4.655753in}{0.998978in}}%
\pgfpathlineto{\pgfqpoint{4.692520in}{1.009338in}}%
\pgfpathlineto{\pgfqpoint{4.730512in}{1.004138in}}%
\pgfpathlineto{\pgfqpoint{4.767278in}{0.980107in}}%
\pgfpathlineto{\pgfqpoint{4.805271in}{0.973848in}}%
\pgfpathlineto{\pgfqpoint{4.843263in}{0.970243in}}%
\pgfpathlineto{\pgfqpoint{4.877578in}{0.966853in}}%
\pgfpathlineto{\pgfqpoint{4.915570in}{0.971490in}}%
\pgfpathlineto{\pgfqpoint{4.952337in}{0.946701in}}%
\pgfpathlineto{\pgfqpoint{4.990329in}{0.960137in}}%
\pgfpathlineto{\pgfqpoint{5.027096in}{0.949449in}}%
\pgfpathlineto{\pgfqpoint{5.065088in}{0.938112in}}%
\pgfpathlineto{\pgfqpoint{5.103080in}{0.937418in}}%
\pgfpathlineto{\pgfqpoint{5.139847in}{0.949273in}}%
\pgfpathlineto{\pgfqpoint{5.177839in}{0.962590in}}%
\pgfpathlineto{\pgfqpoint{5.214605in}{0.979048in}}%
\pgfusepath{stroke}%
\end{pgfscope}%
\begin{pgfscope}%
\pgfsetrectcap%
\pgfsetmiterjoin%
\pgfsetlinewidth{0.803000pt}%
\definecolor{currentstroke}{rgb}{0.000000,0.000000,0.000000}%
\pgfsetstrokecolor{currentstroke}%
\pgfsetdash{}{0pt}%
\pgfpathmoveto{\pgfqpoint{0.553704in}{0.652645in}}%
\pgfpathlineto{\pgfqpoint{0.553704in}{1.841854in}}%
\pgfusepath{stroke}%
\end{pgfscope}%
\begin{pgfscope}%
\pgfsetrectcap%
\pgfsetmiterjoin%
\pgfsetlinewidth{0.803000pt}%
\definecolor{currentstroke}{rgb}{0.000000,0.000000,0.000000}%
\pgfsetstrokecolor{currentstroke}%
\pgfsetdash{}{0pt}%
\pgfpathmoveto{\pgfqpoint{5.436553in}{0.652645in}}%
\pgfpathlineto{\pgfqpoint{5.436553in}{1.841854in}}%
\pgfusepath{stroke}%
\end{pgfscope}%
\begin{pgfscope}%
\pgfsetrectcap%
\pgfsetmiterjoin%
\pgfsetlinewidth{0.803000pt}%
\definecolor{currentstroke}{rgb}{0.000000,0.000000,0.000000}%
\pgfsetstrokecolor{currentstroke}%
\pgfsetdash{}{0pt}%
\pgfpathmoveto{\pgfqpoint{0.553704in}{0.652645in}}%
\pgfpathlineto{\pgfqpoint{5.436553in}{0.652645in}}%
\pgfusepath{stroke}%
\end{pgfscope}%
\begin{pgfscope}%
\pgfsetrectcap%
\pgfsetmiterjoin%
\pgfsetlinewidth{0.803000pt}%
\definecolor{currentstroke}{rgb}{0.000000,0.000000,0.000000}%
\pgfsetstrokecolor{currentstroke}%
\pgfsetdash{}{0pt}%
\pgfpathmoveto{\pgfqpoint{0.553704in}{1.841854in}}%
\pgfpathlineto{\pgfqpoint{5.436553in}{1.841854in}}%
\pgfusepath{stroke}%
\end{pgfscope}%
\begin{pgfscope}%
\pgfsetbuttcap%
\pgfsetmiterjoin%
\pgfsetlinewidth{0.000000pt}%
\definecolor{currentstroke}{rgb}{0.800000,0.800000,0.800000}%
\pgfsetstrokecolor{currentstroke}%
\pgfsetstrokeopacity{0.000000}%
\pgfsetdash{}{0pt}%
\pgfpathmoveto{\pgfqpoint{1.950709in}{0.100000in}}%
\pgfpathlineto{\pgfqpoint{3.882037in}{0.100000in}}%
\pgfpathquadraticcurveto{\pgfqpoint{3.909815in}{0.100000in}}{\pgfqpoint{3.909815in}{0.127778in}}%
\pgfpathlineto{\pgfqpoint{3.909815in}{0.311199in}}%
\pgfpathquadraticcurveto{\pgfqpoint{3.909815in}{0.338977in}}{\pgfqpoint{3.882037in}{0.338977in}}%
\pgfpathlineto{\pgfqpoint{1.950709in}{0.338977in}}%
\pgfpathquadraticcurveto{\pgfqpoint{1.922931in}{0.338977in}}{\pgfqpoint{1.922931in}{0.311199in}}%
\pgfpathlineto{\pgfqpoint{1.922931in}{0.127778in}}%
\pgfpathquadraticcurveto{\pgfqpoint{1.922931in}{0.100000in}}{\pgfqpoint{1.950709in}{0.100000in}}%
\pgfpathclose%
\pgfusepath{}%
\end{pgfscope}%
\begin{pgfscope}%
\pgfsetrectcap%
\pgfsetroundjoin%
\pgfsetlinewidth{1.505625pt}%
\definecolor{currentstroke}{rgb}{0.121569,0.466667,0.705882}%
\pgfsetstrokecolor{currentstroke}%
\pgfsetdash{}{0pt}%
\pgfpathmoveto{\pgfqpoint{1.978487in}{0.231173in}}%
\pgfpathlineto{\pgfqpoint{2.256264in}{0.231173in}}%
\pgfusepath{stroke}%
\end{pgfscope}%
\begin{pgfscope}%
\definecolor{textcolor}{rgb}{0.000000,0.000000,0.000000}%
\pgfsetstrokecolor{textcolor}%
\pgfsetfillcolor{textcolor}%
\pgftext[x=2.367375in,y=0.182562in,left,base]{\color{textcolor}\rmfamily\fontsize{10.000000}{12.000000}\selectfont Összesen}%
\end{pgfscope}%
\begin{pgfscope}%
\pgfsetrectcap%
\pgfsetroundjoin%
\pgfsetlinewidth{1.505625pt}%
\definecolor{currentstroke}{rgb}{1.000000,0.498039,0.054902}%
\pgfsetstrokecolor{currentstroke}%
\pgfsetdash{}{0pt}%
\pgfpathmoveto{\pgfqpoint{3.179876in}{0.231173in}}%
\pgfpathlineto{\pgfqpoint{3.457654in}{0.231173in}}%
\pgfusepath{stroke}%
\end{pgfscope}%
\begin{pgfscope}%
\definecolor{textcolor}{rgb}{0.000000,0.000000,0.000000}%
\pgfsetstrokecolor{textcolor}%
\pgfsetfillcolor{textcolor}%
\pgftext[x=3.568765in,y=0.182562in,left,base]{\color{textcolor}\rmfamily\fontsize{10.000000}{12.000000}\selectfont USA}%
\end{pgfscope}%
\end{pgfpicture}%
\makeatother%
\endgroup%

    \end{center}
\end{figure}

A társadalomra jellemző folyamatok megismeréséhez a General Social Survey adatait használtam fel \citep{gss_2021}. Ez egy olyan amerikai kérdőív, amely lehetővé teszi az országban megfigyelhető jelenségek mérését, valamint reprezentatív a teljes lakosságra nézve. A kérdőív adatait az elemzésben aggregált formában, nemenként külön-külön ábrázoltam, mivel így válik összehasonlíthatóvá a nézettségi adatokkal.

\section{Nézettség becslése}

Mivel a PornHub weboldalról származó nézettségi adatok a 2021 decemberi időpontra vonatkoznak, ezért az egyes időszakokban nem értelmezhetőek a videók nézettségi adatai. Emiatt az elemzésem során a forgalmi adatokból megbecsültem minden videó nézettségét minden egyes időpontra. Ebben a fejezetben ennek a folyamatnak a lépéseit fogom bemutatni.

A forgalmi adatok sajnos csak 2012-től voltak elérhetőek, ezért az első pornográf videó megjelenésétől, vagyis 2017-től kezdve az első adatpontig, tehát 2012-ig megbecsültem a forgalmi adatokat. Azt feltételeztem, hogy ebben az időszakban még nem volt annyira népszerű az oldal, ezért a hiányzó forgalmi adatokat egy számtani sorozattal egészítettem ki a következőképpen:

$$view_t = \frac{view_T}{missing + 1} (1 + t)$$
\\
\noindent ahol a $view_t$ a forgalmi adat a $t$ időpontban, a $view_T$ a forgalmi adat az első meglévő időpontban és a $missing$ a hiányzó adatpontok száma. Az így kapott imputált adatokban az első időpontban sem lesz nulla a forgalom mértéke, valamint az utolsó imputált adatpont nem lesz egyenlő az első valós adatponttal.

Ezt követően kiszámoltam minden egyes időpontra, hogy az adott időponttól kezdve a mai időpontig összesen mennyi forgalom volt megfigyelhető. Ezt a következőképp tettem meg:

$$cum\_traffic_t = \sum_{s = t}^{2021} traffic_s$$
\\
\noindent ahol a $cum\_traffic$ a kumulált forgalom $t$ időpontra vonatkozóan, a $traffic$ egy adott évben mért forgalom és a $t$ a vizsgált időpont.

Könnyen belátható, hogy ezzel a módszerrel egyszerűen megállapítható egy videó nézettsége egy adott évben, mivel a 2021-es nézettségi adat a videó kiadásának időpontjától 2021-ig eltelt forgalomból származik. Ennek megfelelően egy adott videó adott évi nézettsége a következőképp számolható ki:

$$view_{s,t} = \frac{view_{s, 2021}}{cum\_traffic_p} traffic_t$$
\\
ahol a $view$ az $s$ videó nézettsége $t$ időpontban, a $cum\_traffic_p$ a kumulált forgalom az $s$ videó kiadásának évében és a $traffic$ pedig egy adott évben mért forgalom.

Mivel a General Social Survey kérdőív az amerikai lakosságon lett felmérve, ezért, ha a fenti képletet a teljes forgalom ($traffic$) helyett az amerikai forgalommal szorzom fel, akkor megkapom egy adott videó adott évi amerikai nézettségének értékét. Ez a becsülés pedig már alkalmas lehet a pornográfia hatásának aggregált vizsgálatára.

\section{Elemzés}

Mivel a videó és a forgalom adatok aggregált formában és havonta állnak rendelkezésre, ezért ökonometriai modell és egyén szintű elemzés nem végezhető el rajtuk. Ugyanakkor egyszerű vizualizációs technikával érdekes összefüggésekre lehet rávilágítani, amelyek egy későbbi elemzés alapját is képezhetik. Fontos megemlítenem, hogy a szimpla vizualizációs forma nem alkalmas oksági kapcsolatok mérésére, csupán korreláció és együtt mozgás kimutatására alkalmas. 

Ebben a fejezetben a továbbiakban három különböző szempontból fogom vizsgálni a társadalom és a pornográf videók kapcsolatát.

\subsection{A pornográfia és az óvszerhasználat kapcsolata}

Ahogy a szakirodalom is rávilágít, a pornográfiának nagy hatása van az óvszerhasználatra. A \ref{condom.view}. ábrán az látható, hogy míg férfiak körében kevésbé csökkent az óvszerhasználat, addig a nők esetén gyenge csökkenés volt tapasztalható. A 2021-es évre azonban mindkét nemnél csökkenés volt megfigyelhető.

Ezzel egyúttal az ezer főre jutó pornográf videó nézettség folyamatosan növekedett, valamint 2021-ben ugrásszerű megnőtt. Elmondható, hogy a videók nézettsége fordítottan arányos volt az óvszerhasználattal, de ez leginkább csak a 2021-es évre volt igaz. Az ábrán látható eredmények megfelelnek a szakirodalomból vett elvárásokkal.

\begin{figure}[h]
    \caption[Nézettség és óvszerhasználat]{\footnotesize{Az óvszerhasználat és a PornHub videók nézettségének alakulása. Forrás: saját ábra}}
    \label{condom.view}
    \begin{center}
        %% Creator: Matplotlib, PGF backend
%%
%% To include the figure in your LaTeX document, write
%%   \input{<filename>.pgf}
%%
%% Make sure the required packages are loaded in your preamble
%%   \usepackage{pgf}
%%
%% Figures using additional raster images can only be included by \input if
%% they are in the same directory as the main LaTeX file. For loading figures
%% from other directories you can use the `import` package
%%   \usepackage{import}
%%
%% and then include the figures with
%%   \import{<path to file>}{<filename>.pgf}
%%
%% Matplotlib used the following preamble
%%
\begingroup%
\makeatletter%
\begin{pgfpicture}%
\pgfpathrectangle{\pgfpointorigin}{\pgfqpoint{6.317803in}{2.128799in}}%
\pgfusepath{use as bounding box, clip}%
\begin{pgfscope}%
\pgfsetbuttcap%
\pgfsetmiterjoin%
\pgfsetlinewidth{0.000000pt}%
\definecolor{currentstroke}{rgb}{1.000000,1.000000,1.000000}%
\pgfsetstrokecolor{currentstroke}%
\pgfsetstrokeopacity{0.000000}%
\pgfsetdash{}{0pt}%
\pgfpathmoveto{\pgfqpoint{0.000000in}{0.000000in}}%
\pgfpathlineto{\pgfqpoint{6.317803in}{0.000000in}}%
\pgfpathlineto{\pgfqpoint{6.317803in}{2.128799in}}%
\pgfpathlineto{\pgfqpoint{0.000000in}{2.128799in}}%
\pgfpathclose%
\pgfusepath{}%
\end{pgfscope}%
\begin{pgfscope}%
\pgfsetbuttcap%
\pgfsetmiterjoin%
\definecolor{currentfill}{rgb}{1.000000,1.000000,1.000000}%
\pgfsetfillcolor{currentfill}%
\pgfsetlinewidth{0.000000pt}%
\definecolor{currentstroke}{rgb}{0.000000,0.000000,0.000000}%
\pgfsetstrokecolor{currentstroke}%
\pgfsetstrokeopacity{0.000000}%
\pgfsetdash{}{0pt}%
\pgfpathmoveto{\pgfqpoint{0.721528in}{0.693354in}}%
\pgfpathlineto{\pgfqpoint{5.604377in}{0.693354in}}%
\pgfpathlineto{\pgfqpoint{5.604377in}{1.882564in}}%
\pgfpathlineto{\pgfqpoint{0.721528in}{1.882564in}}%
\pgfpathclose%
\pgfusepath{fill}%
\end{pgfscope}%
\begin{pgfscope}%
\pgfpathrectangle{\pgfqpoint{0.721528in}{0.693354in}}{\pgfqpoint{4.882849in}{1.189210in}}%
\pgfusepath{clip}%
\pgfsetbuttcap%
\pgfsetmiterjoin%
\definecolor{currentfill}{rgb}{0.121569,0.466667,0.705882}%
\pgfsetfillcolor{currentfill}%
\pgfsetlinewidth{0.000000pt}%
\definecolor{currentstroke}{rgb}{0.000000,0.000000,0.000000}%
\pgfsetstrokecolor{currentstroke}%
\pgfsetstrokeopacity{0.000000}%
\pgfsetdash{}{0pt}%
\pgfpathmoveto{\pgfqpoint{1.034067in}{0.693354in}}%
\pgfpathlineto{\pgfqpoint{1.245445in}{0.693354in}}%
\pgfpathlineto{\pgfqpoint{1.245445in}{1.783987in}}%
\pgfpathlineto{\pgfqpoint{1.034067in}{1.783987in}}%
\pgfpathclose%
\pgfusepath{fill}%
\end{pgfscope}%
\begin{pgfscope}%
\pgfpathrectangle{\pgfqpoint{0.721528in}{0.693354in}}{\pgfqpoint{4.882849in}{1.189210in}}%
\pgfusepath{clip}%
\pgfsetbuttcap%
\pgfsetmiterjoin%
\definecolor{currentfill}{rgb}{0.121569,0.466667,0.705882}%
\pgfsetfillcolor{currentfill}%
\pgfsetlinewidth{0.000000pt}%
\definecolor{currentstroke}{rgb}{0.000000,0.000000,0.000000}%
\pgfsetstrokecolor{currentstroke}%
\pgfsetstrokeopacity{0.000000}%
\pgfsetdash{}{0pt}%
\pgfpathmoveto{\pgfqpoint{1.638006in}{0.693354in}}%
\pgfpathlineto{\pgfqpoint{1.849385in}{0.693354in}}%
\pgfpathlineto{\pgfqpoint{1.849385in}{1.825935in}}%
\pgfpathlineto{\pgfqpoint{1.638006in}{1.825935in}}%
\pgfpathclose%
\pgfusepath{fill}%
\end{pgfscope}%
\begin{pgfscope}%
\pgfpathrectangle{\pgfqpoint{0.721528in}{0.693354in}}{\pgfqpoint{4.882849in}{1.189210in}}%
\pgfusepath{clip}%
\pgfsetbuttcap%
\pgfsetmiterjoin%
\definecolor{currentfill}{rgb}{0.121569,0.466667,0.705882}%
\pgfsetfillcolor{currentfill}%
\pgfsetlinewidth{0.000000pt}%
\definecolor{currentstroke}{rgb}{0.000000,0.000000,0.000000}%
\pgfsetstrokecolor{currentstroke}%
\pgfsetstrokeopacity{0.000000}%
\pgfsetdash{}{0pt}%
\pgfpathmoveto{\pgfqpoint{2.241945in}{0.693354in}}%
\pgfpathlineto{\pgfqpoint{2.453324in}{0.693354in}}%
\pgfpathlineto{\pgfqpoint{2.453324in}{1.783987in}}%
\pgfpathlineto{\pgfqpoint{2.241945in}{1.783987in}}%
\pgfpathclose%
\pgfusepath{fill}%
\end{pgfscope}%
\begin{pgfscope}%
\pgfpathrectangle{\pgfqpoint{0.721528in}{0.693354in}}{\pgfqpoint{4.882849in}{1.189210in}}%
\pgfusepath{clip}%
\pgfsetbuttcap%
\pgfsetmiterjoin%
\definecolor{currentfill}{rgb}{0.121569,0.466667,0.705882}%
\pgfsetfillcolor{currentfill}%
\pgfsetlinewidth{0.000000pt}%
\definecolor{currentstroke}{rgb}{0.000000,0.000000,0.000000}%
\pgfsetstrokecolor{currentstroke}%
\pgfsetstrokeopacity{0.000000}%
\pgfsetdash{}{0pt}%
\pgfpathmoveto{\pgfqpoint{2.845884in}{0.693354in}}%
\pgfpathlineto{\pgfqpoint{3.057263in}{0.693354in}}%
\pgfpathlineto{\pgfqpoint{3.057263in}{1.742040in}}%
\pgfpathlineto{\pgfqpoint{2.845884in}{1.742040in}}%
\pgfpathclose%
\pgfusepath{fill}%
\end{pgfscope}%
\begin{pgfscope}%
\pgfpathrectangle{\pgfqpoint{0.721528in}{0.693354in}}{\pgfqpoint{4.882849in}{1.189210in}}%
\pgfusepath{clip}%
\pgfsetbuttcap%
\pgfsetmiterjoin%
\definecolor{currentfill}{rgb}{0.121569,0.466667,0.705882}%
\pgfsetfillcolor{currentfill}%
\pgfsetlinewidth{0.000000pt}%
\definecolor{currentstroke}{rgb}{0.000000,0.000000,0.000000}%
\pgfsetstrokecolor{currentstroke}%
\pgfsetstrokeopacity{0.000000}%
\pgfsetdash{}{0pt}%
\pgfpathmoveto{\pgfqpoint{3.449824in}{0.693354in}}%
\pgfpathlineto{\pgfqpoint{3.661202in}{0.693354in}}%
\pgfpathlineto{\pgfqpoint{3.661202in}{1.742040in}}%
\pgfpathlineto{\pgfqpoint{3.449824in}{1.742040in}}%
\pgfpathclose%
\pgfusepath{fill}%
\end{pgfscope}%
\begin{pgfscope}%
\pgfpathrectangle{\pgfqpoint{0.721528in}{0.693354in}}{\pgfqpoint{4.882849in}{1.189210in}}%
\pgfusepath{clip}%
\pgfsetbuttcap%
\pgfsetmiterjoin%
\definecolor{currentfill}{rgb}{0.121569,0.466667,0.705882}%
\pgfsetfillcolor{currentfill}%
\pgfsetlinewidth{0.000000pt}%
\definecolor{currentstroke}{rgb}{0.000000,0.000000,0.000000}%
\pgfsetstrokecolor{currentstroke}%
\pgfsetstrokeopacity{0.000000}%
\pgfsetdash{}{0pt}%
\pgfpathmoveto{\pgfqpoint{4.053763in}{0.693354in}}%
\pgfpathlineto{\pgfqpoint{4.265142in}{0.693354in}}%
\pgfpathlineto{\pgfqpoint{4.265142in}{1.742040in}}%
\pgfpathlineto{\pgfqpoint{4.053763in}{1.742040in}}%
\pgfpathclose%
\pgfusepath{fill}%
\end{pgfscope}%
\begin{pgfscope}%
\pgfpathrectangle{\pgfqpoint{0.721528in}{0.693354in}}{\pgfqpoint{4.882849in}{1.189210in}}%
\pgfusepath{clip}%
\pgfsetbuttcap%
\pgfsetmiterjoin%
\definecolor{currentfill}{rgb}{0.121569,0.466667,0.705882}%
\pgfsetfillcolor{currentfill}%
\pgfsetlinewidth{0.000000pt}%
\definecolor{currentstroke}{rgb}{0.000000,0.000000,0.000000}%
\pgfsetstrokecolor{currentstroke}%
\pgfsetstrokeopacity{0.000000}%
\pgfsetdash{}{0pt}%
\pgfpathmoveto{\pgfqpoint{4.959672in}{0.693354in}}%
\pgfpathlineto{\pgfqpoint{5.171051in}{0.693354in}}%
\pgfpathlineto{\pgfqpoint{5.171051in}{1.406460in}}%
\pgfpathlineto{\pgfqpoint{4.959672in}{1.406460in}}%
\pgfpathclose%
\pgfusepath{fill}%
\end{pgfscope}%
\begin{pgfscope}%
\pgfpathrectangle{\pgfqpoint{0.721528in}{0.693354in}}{\pgfqpoint{4.882849in}{1.189210in}}%
\pgfusepath{clip}%
\pgfsetbuttcap%
\pgfsetmiterjoin%
\definecolor{currentfill}{rgb}{1.000000,0.498039,0.054902}%
\pgfsetfillcolor{currentfill}%
\pgfsetlinewidth{0.000000pt}%
\definecolor{currentstroke}{rgb}{0.000000,0.000000,0.000000}%
\pgfsetstrokecolor{currentstroke}%
\pgfsetstrokeopacity{0.000000}%
\pgfsetdash{}{0pt}%
\pgfpathmoveto{\pgfqpoint{1.245445in}{0.693354in}}%
\pgfpathlineto{\pgfqpoint{1.456824in}{0.693354in}}%
\pgfpathlineto{\pgfqpoint{1.456824in}{1.658145in}}%
\pgfpathlineto{\pgfqpoint{1.245445in}{1.658145in}}%
\pgfpathclose%
\pgfusepath{fill}%
\end{pgfscope}%
\begin{pgfscope}%
\pgfpathrectangle{\pgfqpoint{0.721528in}{0.693354in}}{\pgfqpoint{4.882849in}{1.189210in}}%
\pgfusepath{clip}%
\pgfsetbuttcap%
\pgfsetmiterjoin%
\definecolor{currentfill}{rgb}{1.000000,0.498039,0.054902}%
\pgfsetfillcolor{currentfill}%
\pgfsetlinewidth{0.000000pt}%
\definecolor{currentstroke}{rgb}{0.000000,0.000000,0.000000}%
\pgfsetstrokecolor{currentstroke}%
\pgfsetstrokeopacity{0.000000}%
\pgfsetdash{}{0pt}%
\pgfpathmoveto{\pgfqpoint{1.849385in}{0.693354in}}%
\pgfpathlineto{\pgfqpoint{2.060763in}{0.693354in}}%
\pgfpathlineto{\pgfqpoint{2.060763in}{1.574250in}}%
\pgfpathlineto{\pgfqpoint{1.849385in}{1.574250in}}%
\pgfpathclose%
\pgfusepath{fill}%
\end{pgfscope}%
\begin{pgfscope}%
\pgfpathrectangle{\pgfqpoint{0.721528in}{0.693354in}}{\pgfqpoint{4.882849in}{1.189210in}}%
\pgfusepath{clip}%
\pgfsetbuttcap%
\pgfsetmiterjoin%
\definecolor{currentfill}{rgb}{1.000000,0.498039,0.054902}%
\pgfsetfillcolor{currentfill}%
\pgfsetlinewidth{0.000000pt}%
\definecolor{currentstroke}{rgb}{0.000000,0.000000,0.000000}%
\pgfsetstrokecolor{currentstroke}%
\pgfsetstrokeopacity{0.000000}%
\pgfsetdash{}{0pt}%
\pgfpathmoveto{\pgfqpoint{2.453324in}{0.693354in}}%
\pgfpathlineto{\pgfqpoint{2.664703in}{0.693354in}}%
\pgfpathlineto{\pgfqpoint{2.664703in}{1.448408in}}%
\pgfpathlineto{\pgfqpoint{2.453324in}{1.448408in}}%
\pgfpathclose%
\pgfusepath{fill}%
\end{pgfscope}%
\begin{pgfscope}%
\pgfpathrectangle{\pgfqpoint{0.721528in}{0.693354in}}{\pgfqpoint{4.882849in}{1.189210in}}%
\pgfusepath{clip}%
\pgfsetbuttcap%
\pgfsetmiterjoin%
\definecolor{currentfill}{rgb}{1.000000,0.498039,0.054902}%
\pgfsetfillcolor{currentfill}%
\pgfsetlinewidth{0.000000pt}%
\definecolor{currentstroke}{rgb}{0.000000,0.000000,0.000000}%
\pgfsetstrokecolor{currentstroke}%
\pgfsetstrokeopacity{0.000000}%
\pgfsetdash{}{0pt}%
\pgfpathmoveto{\pgfqpoint{3.057263in}{0.693354in}}%
\pgfpathlineto{\pgfqpoint{3.268642in}{0.693354in}}%
\pgfpathlineto{\pgfqpoint{3.268642in}{1.574250in}}%
\pgfpathlineto{\pgfqpoint{3.057263in}{1.574250in}}%
\pgfpathclose%
\pgfusepath{fill}%
\end{pgfscope}%
\begin{pgfscope}%
\pgfpathrectangle{\pgfqpoint{0.721528in}{0.693354in}}{\pgfqpoint{4.882849in}{1.189210in}}%
\pgfusepath{clip}%
\pgfsetbuttcap%
\pgfsetmiterjoin%
\definecolor{currentfill}{rgb}{1.000000,0.498039,0.054902}%
\pgfsetfillcolor{currentfill}%
\pgfsetlinewidth{0.000000pt}%
\definecolor{currentstroke}{rgb}{0.000000,0.000000,0.000000}%
\pgfsetstrokecolor{currentstroke}%
\pgfsetstrokeopacity{0.000000}%
\pgfsetdash{}{0pt}%
\pgfpathmoveto{\pgfqpoint{3.661202in}{0.693354in}}%
\pgfpathlineto{\pgfqpoint{3.872581in}{0.693354in}}%
\pgfpathlineto{\pgfqpoint{3.872581in}{1.532303in}}%
\pgfpathlineto{\pgfqpoint{3.661202in}{1.532303in}}%
\pgfpathclose%
\pgfusepath{fill}%
\end{pgfscope}%
\begin{pgfscope}%
\pgfpathrectangle{\pgfqpoint{0.721528in}{0.693354in}}{\pgfqpoint{4.882849in}{1.189210in}}%
\pgfusepath{clip}%
\pgfsetbuttcap%
\pgfsetmiterjoin%
\definecolor{currentfill}{rgb}{1.000000,0.498039,0.054902}%
\pgfsetfillcolor{currentfill}%
\pgfsetlinewidth{0.000000pt}%
\definecolor{currentstroke}{rgb}{0.000000,0.000000,0.000000}%
\pgfsetstrokecolor{currentstroke}%
\pgfsetstrokeopacity{0.000000}%
\pgfsetdash{}{0pt}%
\pgfpathmoveto{\pgfqpoint{4.265142in}{0.693354in}}%
\pgfpathlineto{\pgfqpoint{4.476520in}{0.693354in}}%
\pgfpathlineto{\pgfqpoint{4.476520in}{1.406460in}}%
\pgfpathlineto{\pgfqpoint{4.265142in}{1.406460in}}%
\pgfpathclose%
\pgfusepath{fill}%
\end{pgfscope}%
\begin{pgfscope}%
\pgfpathrectangle{\pgfqpoint{0.721528in}{0.693354in}}{\pgfqpoint{4.882849in}{1.189210in}}%
\pgfusepath{clip}%
\pgfsetbuttcap%
\pgfsetmiterjoin%
\definecolor{currentfill}{rgb}{1.000000,0.498039,0.054902}%
\pgfsetfillcolor{currentfill}%
\pgfsetlinewidth{0.000000pt}%
\definecolor{currentstroke}{rgb}{0.000000,0.000000,0.000000}%
\pgfsetstrokecolor{currentstroke}%
\pgfsetstrokeopacity{0.000000}%
\pgfsetdash{}{0pt}%
\pgfpathmoveto{\pgfqpoint{5.171051in}{0.693354in}}%
\pgfpathlineto{\pgfqpoint{5.382429in}{0.693354in}}%
\pgfpathlineto{\pgfqpoint{5.382429in}{1.364513in}}%
\pgfpathlineto{\pgfqpoint{5.171051in}{1.364513in}}%
\pgfpathclose%
\pgfusepath{fill}%
\end{pgfscope}%
\begin{pgfscope}%
\pgfsetbuttcap%
\pgfsetroundjoin%
\definecolor{currentfill}{rgb}{0.000000,0.000000,0.000000}%
\pgfsetfillcolor{currentfill}%
\pgfsetlinewidth{0.803000pt}%
\definecolor{currentstroke}{rgb}{0.000000,0.000000,0.000000}%
\pgfsetstrokecolor{currentstroke}%
\pgfsetdash{}{0pt}%
\pgfsys@defobject{currentmarker}{\pgfqpoint{0.000000in}{-0.048611in}}{\pgfqpoint{0.000000in}{0.000000in}}{%
\pgfpathmoveto{\pgfqpoint{0.000000in}{0.000000in}}%
\pgfpathlineto{\pgfqpoint{0.000000in}{-0.048611in}}%
\pgfusepath{stroke,fill}%
}%
\begin{pgfscope}%
\pgfsys@transformshift{1.245445in}{0.693354in}%
\pgfsys@useobject{currentmarker}{}%
\end{pgfscope}%
\end{pgfscope}%
\begin{pgfscope}%
\definecolor{textcolor}{rgb}{0.000000,0.000000,0.000000}%
\pgfsetstrokecolor{textcolor}%
\pgfsetfillcolor{textcolor}%
\pgftext[x=1.245445in,y=0.596131in,,top]{\color{textcolor}\rmfamily\fontsize{10.000000}{12.000000}\selectfont \(\displaystyle {2008}\)}%
\end{pgfscope}%
\begin{pgfscope}%
\pgfsetbuttcap%
\pgfsetroundjoin%
\definecolor{currentfill}{rgb}{0.000000,0.000000,0.000000}%
\pgfsetfillcolor{currentfill}%
\pgfsetlinewidth{0.803000pt}%
\definecolor{currentstroke}{rgb}{0.000000,0.000000,0.000000}%
\pgfsetstrokecolor{currentstroke}%
\pgfsetdash{}{0pt}%
\pgfsys@defobject{currentmarker}{\pgfqpoint{0.000000in}{-0.048611in}}{\pgfqpoint{0.000000in}{0.000000in}}{%
\pgfpathmoveto{\pgfqpoint{0.000000in}{0.000000in}}%
\pgfpathlineto{\pgfqpoint{0.000000in}{-0.048611in}}%
\pgfusepath{stroke,fill}%
}%
\begin{pgfscope}%
\pgfsys@transformshift{1.849385in}{0.693354in}%
\pgfsys@useobject{currentmarker}{}%
\end{pgfscope}%
\end{pgfscope}%
\begin{pgfscope}%
\definecolor{textcolor}{rgb}{0.000000,0.000000,0.000000}%
\pgfsetstrokecolor{textcolor}%
\pgfsetfillcolor{textcolor}%
\pgftext[x=1.849385in,y=0.596131in,,top]{\color{textcolor}\rmfamily\fontsize{10.000000}{12.000000}\selectfont \(\displaystyle {2010}\)}%
\end{pgfscope}%
\begin{pgfscope}%
\pgfsetbuttcap%
\pgfsetroundjoin%
\definecolor{currentfill}{rgb}{0.000000,0.000000,0.000000}%
\pgfsetfillcolor{currentfill}%
\pgfsetlinewidth{0.803000pt}%
\definecolor{currentstroke}{rgb}{0.000000,0.000000,0.000000}%
\pgfsetstrokecolor{currentstroke}%
\pgfsetdash{}{0pt}%
\pgfsys@defobject{currentmarker}{\pgfqpoint{0.000000in}{-0.048611in}}{\pgfqpoint{0.000000in}{0.000000in}}{%
\pgfpathmoveto{\pgfqpoint{0.000000in}{0.000000in}}%
\pgfpathlineto{\pgfqpoint{0.000000in}{-0.048611in}}%
\pgfusepath{stroke,fill}%
}%
\begin{pgfscope}%
\pgfsys@transformshift{2.453324in}{0.693354in}%
\pgfsys@useobject{currentmarker}{}%
\end{pgfscope}%
\end{pgfscope}%
\begin{pgfscope}%
\definecolor{textcolor}{rgb}{0.000000,0.000000,0.000000}%
\pgfsetstrokecolor{textcolor}%
\pgfsetfillcolor{textcolor}%
\pgftext[x=2.453324in,y=0.596131in,,top]{\color{textcolor}\rmfamily\fontsize{10.000000}{12.000000}\selectfont \(\displaystyle {2012}\)}%
\end{pgfscope}%
\begin{pgfscope}%
\pgfsetbuttcap%
\pgfsetroundjoin%
\definecolor{currentfill}{rgb}{0.000000,0.000000,0.000000}%
\pgfsetfillcolor{currentfill}%
\pgfsetlinewidth{0.803000pt}%
\definecolor{currentstroke}{rgb}{0.000000,0.000000,0.000000}%
\pgfsetstrokecolor{currentstroke}%
\pgfsetdash{}{0pt}%
\pgfsys@defobject{currentmarker}{\pgfqpoint{0.000000in}{-0.048611in}}{\pgfqpoint{0.000000in}{0.000000in}}{%
\pgfpathmoveto{\pgfqpoint{0.000000in}{0.000000in}}%
\pgfpathlineto{\pgfqpoint{0.000000in}{-0.048611in}}%
\pgfusepath{stroke,fill}%
}%
\begin{pgfscope}%
\pgfsys@transformshift{3.057263in}{0.693354in}%
\pgfsys@useobject{currentmarker}{}%
\end{pgfscope}%
\end{pgfscope}%
\begin{pgfscope}%
\definecolor{textcolor}{rgb}{0.000000,0.000000,0.000000}%
\pgfsetstrokecolor{textcolor}%
\pgfsetfillcolor{textcolor}%
\pgftext[x=3.057263in,y=0.596131in,,top]{\color{textcolor}\rmfamily\fontsize{10.000000}{12.000000}\selectfont \(\displaystyle {2014}\)}%
\end{pgfscope}%
\begin{pgfscope}%
\pgfsetbuttcap%
\pgfsetroundjoin%
\definecolor{currentfill}{rgb}{0.000000,0.000000,0.000000}%
\pgfsetfillcolor{currentfill}%
\pgfsetlinewidth{0.803000pt}%
\definecolor{currentstroke}{rgb}{0.000000,0.000000,0.000000}%
\pgfsetstrokecolor{currentstroke}%
\pgfsetdash{}{0pt}%
\pgfsys@defobject{currentmarker}{\pgfqpoint{0.000000in}{-0.048611in}}{\pgfqpoint{0.000000in}{0.000000in}}{%
\pgfpathmoveto{\pgfqpoint{0.000000in}{0.000000in}}%
\pgfpathlineto{\pgfqpoint{0.000000in}{-0.048611in}}%
\pgfusepath{stroke,fill}%
}%
\begin{pgfscope}%
\pgfsys@transformshift{3.661202in}{0.693354in}%
\pgfsys@useobject{currentmarker}{}%
\end{pgfscope}%
\end{pgfscope}%
\begin{pgfscope}%
\definecolor{textcolor}{rgb}{0.000000,0.000000,0.000000}%
\pgfsetstrokecolor{textcolor}%
\pgfsetfillcolor{textcolor}%
\pgftext[x=3.661202in,y=0.596131in,,top]{\color{textcolor}\rmfamily\fontsize{10.000000}{12.000000}\selectfont \(\displaystyle {2016}\)}%
\end{pgfscope}%
\begin{pgfscope}%
\pgfsetbuttcap%
\pgfsetroundjoin%
\definecolor{currentfill}{rgb}{0.000000,0.000000,0.000000}%
\pgfsetfillcolor{currentfill}%
\pgfsetlinewidth{0.803000pt}%
\definecolor{currentstroke}{rgb}{0.000000,0.000000,0.000000}%
\pgfsetstrokecolor{currentstroke}%
\pgfsetdash{}{0pt}%
\pgfsys@defobject{currentmarker}{\pgfqpoint{0.000000in}{-0.048611in}}{\pgfqpoint{0.000000in}{0.000000in}}{%
\pgfpathmoveto{\pgfqpoint{0.000000in}{0.000000in}}%
\pgfpathlineto{\pgfqpoint{0.000000in}{-0.048611in}}%
\pgfusepath{stroke,fill}%
}%
\begin{pgfscope}%
\pgfsys@transformshift{4.265142in}{0.693354in}%
\pgfsys@useobject{currentmarker}{}%
\end{pgfscope}%
\end{pgfscope}%
\begin{pgfscope}%
\definecolor{textcolor}{rgb}{0.000000,0.000000,0.000000}%
\pgfsetstrokecolor{textcolor}%
\pgfsetfillcolor{textcolor}%
\pgftext[x=4.265142in,y=0.596131in,,top]{\color{textcolor}\rmfamily\fontsize{10.000000}{12.000000}\selectfont \(\displaystyle {2018}\)}%
\end{pgfscope}%
\begin{pgfscope}%
\pgfsetbuttcap%
\pgfsetroundjoin%
\definecolor{currentfill}{rgb}{0.000000,0.000000,0.000000}%
\pgfsetfillcolor{currentfill}%
\pgfsetlinewidth{0.803000pt}%
\definecolor{currentstroke}{rgb}{0.000000,0.000000,0.000000}%
\pgfsetstrokecolor{currentstroke}%
\pgfsetdash{}{0pt}%
\pgfsys@defobject{currentmarker}{\pgfqpoint{0.000000in}{-0.048611in}}{\pgfqpoint{0.000000in}{0.000000in}}{%
\pgfpathmoveto{\pgfqpoint{0.000000in}{0.000000in}}%
\pgfpathlineto{\pgfqpoint{0.000000in}{-0.048611in}}%
\pgfusepath{stroke,fill}%
}%
\begin{pgfscope}%
\pgfsys@transformshift{5.171051in}{0.693354in}%
\pgfsys@useobject{currentmarker}{}%
\end{pgfscope}%
\end{pgfscope}%
\begin{pgfscope}%
\definecolor{textcolor}{rgb}{0.000000,0.000000,0.000000}%
\pgfsetstrokecolor{textcolor}%
\pgfsetfillcolor{textcolor}%
\pgftext[x=5.171051in,y=0.596131in,,top]{\color{textcolor}\rmfamily\fontsize{10.000000}{12.000000}\selectfont \(\displaystyle {2021}\)}%
\end{pgfscope}%
\begin{pgfscope}%
\pgfsetbuttcap%
\pgfsetroundjoin%
\definecolor{currentfill}{rgb}{0.000000,0.000000,0.000000}%
\pgfsetfillcolor{currentfill}%
\pgfsetlinewidth{0.803000pt}%
\definecolor{currentstroke}{rgb}{0.000000,0.000000,0.000000}%
\pgfsetstrokecolor{currentstroke}%
\pgfsetdash{}{0pt}%
\pgfsys@defobject{currentmarker}{\pgfqpoint{-0.048611in}{0.000000in}}{\pgfqpoint{-0.000000in}{0.000000in}}{%
\pgfpathmoveto{\pgfqpoint{-0.000000in}{0.000000in}}%
\pgfpathlineto{\pgfqpoint{-0.048611in}{0.000000in}}%
\pgfusepath{stroke,fill}%
}%
\begin{pgfscope}%
\pgfsys@transformshift{0.721528in}{0.693354in}%
\pgfsys@useobject{currentmarker}{}%
\end{pgfscope}%
\end{pgfscope}%
\begin{pgfscope}%
\definecolor{textcolor}{rgb}{0.000000,0.000000,0.000000}%
\pgfsetstrokecolor{textcolor}%
\pgfsetfillcolor{textcolor}%
\pgftext[x=0.446836in, y=0.645128in, left, base]{\color{textcolor}\rmfamily\fontsize{10.000000}{12.000000}\selectfont \(\displaystyle {0.0}\)}%
\end{pgfscope}%
\begin{pgfscope}%
\pgfsetbuttcap%
\pgfsetroundjoin%
\definecolor{currentfill}{rgb}{0.000000,0.000000,0.000000}%
\pgfsetfillcolor{currentfill}%
\pgfsetlinewidth{0.803000pt}%
\definecolor{currentstroke}{rgb}{0.000000,0.000000,0.000000}%
\pgfsetstrokecolor{currentstroke}%
\pgfsetdash{}{0pt}%
\pgfsys@defobject{currentmarker}{\pgfqpoint{-0.048611in}{0.000000in}}{\pgfqpoint{-0.000000in}{0.000000in}}{%
\pgfpathmoveto{\pgfqpoint{-0.000000in}{0.000000in}}%
\pgfpathlineto{\pgfqpoint{-0.048611in}{0.000000in}}%
\pgfusepath{stroke,fill}%
}%
\begin{pgfscope}%
\pgfsys@transformshift{0.721528in}{1.112828in}%
\pgfsys@useobject{currentmarker}{}%
\end{pgfscope}%
\end{pgfscope}%
\begin{pgfscope}%
\definecolor{textcolor}{rgb}{0.000000,0.000000,0.000000}%
\pgfsetstrokecolor{textcolor}%
\pgfsetfillcolor{textcolor}%
\pgftext[x=0.446836in, y=1.064603in, left, base]{\color{textcolor}\rmfamily\fontsize{10.000000}{12.000000}\selectfont \(\displaystyle {0.1}\)}%
\end{pgfscope}%
\begin{pgfscope}%
\pgfsetbuttcap%
\pgfsetroundjoin%
\definecolor{currentfill}{rgb}{0.000000,0.000000,0.000000}%
\pgfsetfillcolor{currentfill}%
\pgfsetlinewidth{0.803000pt}%
\definecolor{currentstroke}{rgb}{0.000000,0.000000,0.000000}%
\pgfsetstrokecolor{currentstroke}%
\pgfsetdash{}{0pt}%
\pgfsys@defobject{currentmarker}{\pgfqpoint{-0.048611in}{0.000000in}}{\pgfqpoint{-0.000000in}{0.000000in}}{%
\pgfpathmoveto{\pgfqpoint{-0.000000in}{0.000000in}}%
\pgfpathlineto{\pgfqpoint{-0.048611in}{0.000000in}}%
\pgfusepath{stroke,fill}%
}%
\begin{pgfscope}%
\pgfsys@transformshift{0.721528in}{1.532303in}%
\pgfsys@useobject{currentmarker}{}%
\end{pgfscope}%
\end{pgfscope}%
\begin{pgfscope}%
\definecolor{textcolor}{rgb}{0.000000,0.000000,0.000000}%
\pgfsetstrokecolor{textcolor}%
\pgfsetfillcolor{textcolor}%
\pgftext[x=0.446836in, y=1.484077in, left, base]{\color{textcolor}\rmfamily\fontsize{10.000000}{12.000000}\selectfont \(\displaystyle {0.2}\)}%
\end{pgfscope}%
\begin{pgfscope}%
\definecolor{textcolor}{rgb}{0.000000,0.000000,0.000000}%
\pgfsetstrokecolor{textcolor}%
\pgfsetfillcolor{textcolor}%
\pgftext[x=0.204552in, y=0.771947in, left, base,rotate=90.000000]{\color{textcolor}\rmfamily\fontsize{10.000000}{12.000000}\selectfont Óvszerhasználat }%
\end{pgfscope}%
\begin{pgfscope}%
\definecolor{textcolor}{rgb}{0.000000,0.000000,0.000000}%
\pgfsetstrokecolor{textcolor}%
\pgfsetfillcolor{textcolor}%
\pgftext[x=0.356558in, y=1.176076in, left, base,rotate=90.000000]{\color{textcolor}\rmfamily\fontsize{10.000000}{12.000000}\selectfont (\%)}%
\end{pgfscope}%
\begin{pgfscope}%
\pgfsetrectcap%
\pgfsetmiterjoin%
\pgfsetlinewidth{0.803000pt}%
\definecolor{currentstroke}{rgb}{0.000000,0.000000,0.000000}%
\pgfsetstrokecolor{currentstroke}%
\pgfsetdash{}{0pt}%
\pgfpathmoveto{\pgfqpoint{0.721528in}{0.693354in}}%
\pgfpathlineto{\pgfqpoint{0.721528in}{1.882564in}}%
\pgfusepath{stroke}%
\end{pgfscope}%
\begin{pgfscope}%
\pgfsetrectcap%
\pgfsetmiterjoin%
\pgfsetlinewidth{0.803000pt}%
\definecolor{currentstroke}{rgb}{0.000000,0.000000,0.000000}%
\pgfsetstrokecolor{currentstroke}%
\pgfsetdash{}{0pt}%
\pgfpathmoveto{\pgfqpoint{5.604377in}{0.693354in}}%
\pgfpathlineto{\pgfqpoint{5.604377in}{1.882564in}}%
\pgfusepath{stroke}%
\end{pgfscope}%
\begin{pgfscope}%
\pgfsetrectcap%
\pgfsetmiterjoin%
\pgfsetlinewidth{0.803000pt}%
\definecolor{currentstroke}{rgb}{0.000000,0.000000,0.000000}%
\pgfsetstrokecolor{currentstroke}%
\pgfsetdash{}{0pt}%
\pgfpathmoveto{\pgfqpoint{0.721528in}{0.693354in}}%
\pgfpathlineto{\pgfqpoint{5.604377in}{0.693354in}}%
\pgfusepath{stroke}%
\end{pgfscope}%
\begin{pgfscope}%
\pgfsetrectcap%
\pgfsetmiterjoin%
\pgfsetlinewidth{0.803000pt}%
\definecolor{currentstroke}{rgb}{0.000000,0.000000,0.000000}%
\pgfsetstrokecolor{currentstroke}%
\pgfsetdash{}{0pt}%
\pgfpathmoveto{\pgfqpoint{0.721528in}{1.882564in}}%
\pgfpathlineto{\pgfqpoint{5.604377in}{1.882564in}}%
\pgfusepath{stroke}%
\end{pgfscope}%
\begin{pgfscope}%
\definecolor{textcolor}{rgb}{1.000000,1.000000,1.000000}%
\pgfsetstrokecolor{textcolor}%
\pgfsetfillcolor{textcolor}%
\pgftext[x=1.167534in, y=1.084540in, left, base,rotate=90.000000]{\color{textcolor}\rmfamily\fontsize{8.000000}{9.600000}\selectfont 0.26\%}%
\end{pgfscope}%
\begin{pgfscope}%
\definecolor{textcolor}{rgb}{1.000000,1.000000,1.000000}%
\pgfsetstrokecolor{textcolor}%
\pgfsetfillcolor{textcolor}%
\pgftext[x=1.771473in, y=1.105514in, left, base,rotate=90.000000]{\color{textcolor}\rmfamily\fontsize{8.000000}{9.600000}\selectfont 0.27\%}%
\end{pgfscope}%
\begin{pgfscope}%
\definecolor{textcolor}{rgb}{1.000000,1.000000,1.000000}%
\pgfsetstrokecolor{textcolor}%
\pgfsetfillcolor{textcolor}%
\pgftext[x=2.375412in, y=1.084540in, left, base,rotate=90.000000]{\color{textcolor}\rmfamily\fontsize{8.000000}{9.600000}\selectfont 0.26\%}%
\end{pgfscope}%
\begin{pgfscope}%
\definecolor{textcolor}{rgb}{1.000000,1.000000,1.000000}%
\pgfsetstrokecolor{textcolor}%
\pgfsetfillcolor{textcolor}%
\pgftext[x=2.979352in, y=1.063566in, left, base,rotate=90.000000]{\color{textcolor}\rmfamily\fontsize{8.000000}{9.600000}\selectfont 0.25\%}%
\end{pgfscope}%
\begin{pgfscope}%
\definecolor{textcolor}{rgb}{1.000000,1.000000,1.000000}%
\pgfsetstrokecolor{textcolor}%
\pgfsetfillcolor{textcolor}%
\pgftext[x=3.583291in, y=1.063566in, left, base,rotate=90.000000]{\color{textcolor}\rmfamily\fontsize{8.000000}{9.600000}\selectfont 0.25\%}%
\end{pgfscope}%
\begin{pgfscope}%
\definecolor{textcolor}{rgb}{1.000000,1.000000,1.000000}%
\pgfsetstrokecolor{textcolor}%
\pgfsetfillcolor{textcolor}%
\pgftext[x=4.187230in, y=1.063566in, left, base,rotate=90.000000]{\color{textcolor}\rmfamily\fontsize{8.000000}{9.600000}\selectfont 0.25\%}%
\end{pgfscope}%
\begin{pgfscope}%
\definecolor{textcolor}{rgb}{1.000000,1.000000,1.000000}%
\pgfsetstrokecolor{textcolor}%
\pgfsetfillcolor{textcolor}%
\pgftext[x=5.093139in, y=0.895777in, left, base,rotate=90.000000]{\color{textcolor}\rmfamily\fontsize{8.000000}{9.600000}\selectfont 0.17\%}%
\end{pgfscope}%
\begin{pgfscope}%
\definecolor{textcolor}{rgb}{1.000000,1.000000,1.000000}%
\pgfsetstrokecolor{textcolor}%
\pgfsetfillcolor{textcolor}%
\pgftext[x=1.378913in, y=1.021619in, left, base,rotate=90.000000]{\color{textcolor}\rmfamily\fontsize{8.000000}{9.600000}\selectfont 0.23\%}%
\end{pgfscope}%
\begin{pgfscope}%
\definecolor{textcolor}{rgb}{1.000000,1.000000,1.000000}%
\pgfsetstrokecolor{textcolor}%
\pgfsetfillcolor{textcolor}%
\pgftext[x=1.982852in, y=0.979672in, left, base,rotate=90.000000]{\color{textcolor}\rmfamily\fontsize{8.000000}{9.600000}\selectfont 0.21\%}%
\end{pgfscope}%
\begin{pgfscope}%
\definecolor{textcolor}{rgb}{1.000000,1.000000,1.000000}%
\pgfsetstrokecolor{textcolor}%
\pgfsetfillcolor{textcolor}%
\pgftext[x=2.586791in, y=0.916750in, left, base,rotate=90.000000]{\color{textcolor}\rmfamily\fontsize{8.000000}{9.600000}\selectfont 0.18\%}%
\end{pgfscope}%
\begin{pgfscope}%
\definecolor{textcolor}{rgb}{1.000000,1.000000,1.000000}%
\pgfsetstrokecolor{textcolor}%
\pgfsetfillcolor{textcolor}%
\pgftext[x=3.190730in, y=0.979672in, left, base,rotate=90.000000]{\color{textcolor}\rmfamily\fontsize{8.000000}{9.600000}\selectfont 0.21\%}%
\end{pgfscope}%
\begin{pgfscope}%
\definecolor{textcolor}{rgb}{1.000000,1.000000,1.000000}%
\pgfsetstrokecolor{textcolor}%
\pgfsetfillcolor{textcolor}%
\pgftext[x=3.794670in, y=0.958698in, left, base,rotate=90.000000]{\color{textcolor}\rmfamily\fontsize{8.000000}{9.600000}\selectfont 0.20\%}%
\end{pgfscope}%
\begin{pgfscope}%
\definecolor{textcolor}{rgb}{1.000000,1.000000,1.000000}%
\pgfsetstrokecolor{textcolor}%
\pgfsetfillcolor{textcolor}%
\pgftext[x=4.398609in, y=0.895777in, left, base,rotate=90.000000]{\color{textcolor}\rmfamily\fontsize{8.000000}{9.600000}\selectfont 0.17\%}%
\end{pgfscope}%
\begin{pgfscope}%
\definecolor{textcolor}{rgb}{1.000000,1.000000,1.000000}%
\pgfsetstrokecolor{textcolor}%
\pgfsetfillcolor{textcolor}%
\pgftext[x=5.304518in, y=0.874803in, left, base,rotate=90.000000]{\color{textcolor}\rmfamily\fontsize{8.000000}{9.600000}\selectfont 0.16\%}%
\end{pgfscope}%
\begin{pgfscope}%
\pgfsetbuttcap%
\pgfsetmiterjoin%
\pgfsetlinewidth{0.000000pt}%
\definecolor{currentstroke}{rgb}{0.800000,0.800000,0.800000}%
\pgfsetstrokecolor{currentstroke}%
\pgfsetstrokeopacity{0.000000}%
\pgfsetdash{}{0pt}%
\pgfpathmoveto{\pgfqpoint{2.378616in}{0.100000in}}%
\pgfpathlineto{\pgfqpoint{3.947290in}{0.100000in}}%
\pgfpathquadraticcurveto{\pgfqpoint{3.975067in}{0.100000in}}{\pgfqpoint{3.975067in}{0.127778in}}%
\pgfpathlineto{\pgfqpoint{3.975067in}{0.307562in}}%
\pgfpathquadraticcurveto{\pgfqpoint{3.975067in}{0.335339in}}{\pgfqpoint{3.947290in}{0.335339in}}%
\pgfpathlineto{\pgfqpoint{2.378616in}{0.335339in}}%
\pgfpathquadraticcurveto{\pgfqpoint{2.350838in}{0.335339in}}{\pgfqpoint{2.350838in}{0.307562in}}%
\pgfpathlineto{\pgfqpoint{2.350838in}{0.127778in}}%
\pgfpathquadraticcurveto{\pgfqpoint{2.350838in}{0.100000in}}{\pgfqpoint{2.378616in}{0.100000in}}%
\pgfpathclose%
\pgfusepath{}%
\end{pgfscope}%
\begin{pgfscope}%
\pgfsetbuttcap%
\pgfsetmiterjoin%
\definecolor{currentfill}{rgb}{0.121569,0.466667,0.705882}%
\pgfsetfillcolor{currentfill}%
\pgfsetlinewidth{0.000000pt}%
\definecolor{currentstroke}{rgb}{0.000000,0.000000,0.000000}%
\pgfsetstrokecolor{currentstroke}%
\pgfsetstrokeopacity{0.000000}%
\pgfsetdash{}{0pt}%
\pgfpathmoveto{\pgfqpoint{2.406393in}{0.182562in}}%
\pgfpathlineto{\pgfqpoint{2.684171in}{0.182562in}}%
\pgfpathlineto{\pgfqpoint{2.684171in}{0.279784in}}%
\pgfpathlineto{\pgfqpoint{2.406393in}{0.279784in}}%
\pgfpathclose%
\pgfusepath{fill}%
\end{pgfscope}%
\begin{pgfscope}%
\definecolor{textcolor}{rgb}{0.000000,0.000000,0.000000}%
\pgfsetstrokecolor{textcolor}%
\pgfsetfillcolor{textcolor}%
\pgftext[x=2.795282in,y=0.182562in,left,base]{\color{textcolor}\rmfamily\fontsize{10.000000}{12.000000}\selectfont Férfi}%
\end{pgfscope}%
\begin{pgfscope}%
\pgfsetbuttcap%
\pgfsetmiterjoin%
\definecolor{currentfill}{rgb}{1.000000,0.498039,0.054902}%
\pgfsetfillcolor{currentfill}%
\pgfsetlinewidth{0.000000pt}%
\definecolor{currentstroke}{rgb}{0.000000,0.000000,0.000000}%
\pgfsetstrokecolor{currentstroke}%
\pgfsetstrokeopacity{0.000000}%
\pgfsetdash{}{0pt}%
\pgfpathmoveto{\pgfqpoint{3.357011in}{0.182562in}}%
\pgfpathlineto{\pgfqpoint{3.634789in}{0.182562in}}%
\pgfpathlineto{\pgfqpoint{3.634789in}{0.279784in}}%
\pgfpathlineto{\pgfqpoint{3.357011in}{0.279784in}}%
\pgfpathclose%
\pgfusepath{fill}%
\end{pgfscope}%
\begin{pgfscope}%
\definecolor{textcolor}{rgb}{0.000000,0.000000,0.000000}%
\pgfsetstrokecolor{textcolor}%
\pgfsetfillcolor{textcolor}%
\pgftext[x=3.745900in,y=0.182562in,left,base]{\color{textcolor}\rmfamily\fontsize{10.000000}{12.000000}\selectfont Nő}%
\end{pgfscope}%
\begin{pgfscope}%
\pgfsetbuttcap%
\pgfsetroundjoin%
\definecolor{currentfill}{rgb}{0.000000,0.000000,0.000000}%
\pgfsetfillcolor{currentfill}%
\pgfsetlinewidth{0.803000pt}%
\definecolor{currentstroke}{rgb}{0.000000,0.000000,0.000000}%
\pgfsetstrokecolor{currentstroke}%
\pgfsetdash{}{0pt}%
\pgfsys@defobject{currentmarker}{\pgfqpoint{0.000000in}{0.000000in}}{\pgfqpoint{0.048611in}{0.000000in}}{%
\pgfpathmoveto{\pgfqpoint{0.000000in}{0.000000in}}%
\pgfpathlineto{\pgfqpoint{0.048611in}{0.000000in}}%
\pgfusepath{stroke,fill}%
}%
\begin{pgfscope}%
\pgfsys@transformshift{5.604377in}{0.747409in}%
\pgfsys@useobject{currentmarker}{}%
\end{pgfscope}%
\end{pgfscope}%
\begin{pgfscope}%
\definecolor{textcolor}{rgb}{0.000000,0.000000,0.000000}%
\pgfsetstrokecolor{textcolor}%
\pgfsetfillcolor{textcolor}%
\pgftext[x=5.701599in, y=0.699183in, left, base]{\color{textcolor}\rmfamily\fontsize{10.000000}{12.000000}\selectfont \(\displaystyle {0.0}\)}%
\end{pgfscope}%
\begin{pgfscope}%
\pgfsetbuttcap%
\pgfsetroundjoin%
\definecolor{currentfill}{rgb}{0.000000,0.000000,0.000000}%
\pgfsetfillcolor{currentfill}%
\pgfsetlinewidth{0.803000pt}%
\definecolor{currentstroke}{rgb}{0.000000,0.000000,0.000000}%
\pgfsetstrokecolor{currentstroke}%
\pgfsetdash{}{0pt}%
\pgfsys@defobject{currentmarker}{\pgfqpoint{0.000000in}{0.000000in}}{\pgfqpoint{0.048611in}{0.000000in}}{%
\pgfpathmoveto{\pgfqpoint{0.000000in}{0.000000in}}%
\pgfpathlineto{\pgfqpoint{0.048611in}{0.000000in}}%
\pgfusepath{stroke,fill}%
}%
\begin{pgfscope}%
\pgfsys@transformshift{5.604377in}{1.091549in}%
\pgfsys@useobject{currentmarker}{}%
\end{pgfscope}%
\end{pgfscope}%
\begin{pgfscope}%
\definecolor{textcolor}{rgb}{0.000000,0.000000,0.000000}%
\pgfsetstrokecolor{textcolor}%
\pgfsetfillcolor{textcolor}%
\pgftext[x=5.701599in, y=1.043323in, left, base]{\color{textcolor}\rmfamily\fontsize{10.000000}{12.000000}\selectfont \(\displaystyle {2.5}\)}%
\end{pgfscope}%
\begin{pgfscope}%
\pgfsetbuttcap%
\pgfsetroundjoin%
\definecolor{currentfill}{rgb}{0.000000,0.000000,0.000000}%
\pgfsetfillcolor{currentfill}%
\pgfsetlinewidth{0.803000pt}%
\definecolor{currentstroke}{rgb}{0.000000,0.000000,0.000000}%
\pgfsetstrokecolor{currentstroke}%
\pgfsetdash{}{0pt}%
\pgfsys@defobject{currentmarker}{\pgfqpoint{0.000000in}{0.000000in}}{\pgfqpoint{0.048611in}{0.000000in}}{%
\pgfpathmoveto{\pgfqpoint{0.000000in}{0.000000in}}%
\pgfpathlineto{\pgfqpoint{0.048611in}{0.000000in}}%
\pgfusepath{stroke,fill}%
}%
\begin{pgfscope}%
\pgfsys@transformshift{5.604377in}{1.435689in}%
\pgfsys@useobject{currentmarker}{}%
\end{pgfscope}%
\end{pgfscope}%
\begin{pgfscope}%
\definecolor{textcolor}{rgb}{0.000000,0.000000,0.000000}%
\pgfsetstrokecolor{textcolor}%
\pgfsetfillcolor{textcolor}%
\pgftext[x=5.701599in, y=1.387463in, left, base]{\color{textcolor}\rmfamily\fontsize{10.000000}{12.000000}\selectfont \(\displaystyle {5.0}\)}%
\end{pgfscope}%
\begin{pgfscope}%
\pgfsetbuttcap%
\pgfsetroundjoin%
\definecolor{currentfill}{rgb}{0.000000,0.000000,0.000000}%
\pgfsetfillcolor{currentfill}%
\pgfsetlinewidth{0.803000pt}%
\definecolor{currentstroke}{rgb}{0.000000,0.000000,0.000000}%
\pgfsetstrokecolor{currentstroke}%
\pgfsetdash{}{0pt}%
\pgfsys@defobject{currentmarker}{\pgfqpoint{0.000000in}{0.000000in}}{\pgfqpoint{0.048611in}{0.000000in}}{%
\pgfpathmoveto{\pgfqpoint{0.000000in}{0.000000in}}%
\pgfpathlineto{\pgfqpoint{0.048611in}{0.000000in}}%
\pgfusepath{stroke,fill}%
}%
\begin{pgfscope}%
\pgfsys@transformshift{5.604377in}{1.779829in}%
\pgfsys@useobject{currentmarker}{}%
\end{pgfscope}%
\end{pgfscope}%
\begin{pgfscope}%
\definecolor{textcolor}{rgb}{0.000000,0.000000,0.000000}%
\pgfsetstrokecolor{textcolor}%
\pgfsetfillcolor{textcolor}%
\pgftext[x=5.701599in, y=1.731603in, left, base]{\color{textcolor}\rmfamily\fontsize{10.000000}{12.000000}\selectfont \(\displaystyle {7.5}\)}%
\end{pgfscope}%
\begin{pgfscope}%
\definecolor{textcolor}{rgb}{0.000000,0.000000,0.000000}%
\pgfsetstrokecolor{textcolor}%
\pgfsetfillcolor{textcolor}%
\pgftext[x=6.031075in, y=0.973144in, left, base,rotate=90.000000]{\color{textcolor}\rmfamily\fontsize{10.000000}{12.000000}\selectfont Nézettség }%
\end{pgfscope}%
\begin{pgfscope}%
\definecolor{textcolor}{rgb}{0.000000,0.000000,0.000000}%
\pgfsetstrokecolor{textcolor}%
\pgfsetfillcolor{textcolor}%
\pgftext[x=6.183081in, y=1.015967in, left, base,rotate=90.000000]{\color{textcolor}\rmfamily\fontsize{10.000000}{12.000000}\selectfont (1000 fő)}%
\end{pgfscope}%
\begin{pgfscope}%
\definecolor{textcolor}{rgb}{0.000000,0.000000,0.000000}%
\pgfsetstrokecolor{textcolor}%
\pgfsetfillcolor{textcolor}%
\pgftext[x=5.604377in,y=1.924230in,right,base]{\color{textcolor}\rmfamily\fontsize{10.000000}{12.000000}\selectfont \(\displaystyle \times{10^{4}}{}\)}%
\end{pgfscope}%
\begin{pgfscope}%
\pgfpathrectangle{\pgfqpoint{0.721528in}{0.693354in}}{\pgfqpoint{4.882849in}{1.189210in}}%
\pgfusepath{clip}%
\pgfsetrectcap%
\pgfsetroundjoin%
\pgfsetlinewidth{1.505625pt}%
\definecolor{currentstroke}{rgb}{0.000000,0.000000,0.000000}%
\pgfsetstrokecolor{currentstroke}%
\pgfsetdash{}{0pt}%
\pgfpathmoveto{\pgfqpoint{0.943476in}{0.747409in}}%
\pgfpathlineto{\pgfqpoint{1.245445in}{0.747413in}}%
\pgfpathlineto{\pgfqpoint{1.547415in}{0.747449in}}%
\pgfpathlineto{\pgfqpoint{1.849385in}{0.747713in}}%
\pgfpathlineto{\pgfqpoint{2.151354in}{0.748685in}}%
\pgfpathlineto{\pgfqpoint{2.453324in}{0.750731in}}%
\pgfpathlineto{\pgfqpoint{2.755294in}{0.752387in}}%
\pgfpathlineto{\pgfqpoint{3.057263in}{0.769587in}}%
\pgfpathlineto{\pgfqpoint{3.359233in}{0.802215in}}%
\pgfpathlineto{\pgfqpoint{3.661202in}{0.855611in}}%
\pgfpathlineto{\pgfqpoint{3.963172in}{0.918555in}}%
\pgfpathlineto{\pgfqpoint{4.265142in}{1.013737in}}%
\pgfpathlineto{\pgfqpoint{4.567111in}{1.146256in}}%
\pgfpathlineto{\pgfqpoint{4.869081in}{1.615453in}}%
\pgfpathlineto{\pgfqpoint{5.171051in}{1.828509in}}%
\pgfusepath{stroke}%
\end{pgfscope}%
\begin{pgfscope}%
\pgfsetrectcap%
\pgfsetmiterjoin%
\pgfsetlinewidth{0.803000pt}%
\definecolor{currentstroke}{rgb}{0.000000,0.000000,0.000000}%
\pgfsetstrokecolor{currentstroke}%
\pgfsetdash{}{0pt}%
\pgfpathmoveto{\pgfqpoint{0.721528in}{0.693354in}}%
\pgfpathlineto{\pgfqpoint{0.721528in}{1.882564in}}%
\pgfusepath{stroke}%
\end{pgfscope}%
\begin{pgfscope}%
\pgfsetrectcap%
\pgfsetmiterjoin%
\pgfsetlinewidth{0.803000pt}%
\definecolor{currentstroke}{rgb}{0.000000,0.000000,0.000000}%
\pgfsetstrokecolor{currentstroke}%
\pgfsetdash{}{0pt}%
\pgfpathmoveto{\pgfqpoint{5.604377in}{0.693354in}}%
\pgfpathlineto{\pgfqpoint{5.604377in}{1.882564in}}%
\pgfusepath{stroke}%
\end{pgfscope}%
\begin{pgfscope}%
\pgfsetrectcap%
\pgfsetmiterjoin%
\pgfsetlinewidth{0.803000pt}%
\definecolor{currentstroke}{rgb}{0.000000,0.000000,0.000000}%
\pgfsetstrokecolor{currentstroke}%
\pgfsetdash{}{0pt}%
\pgfpathmoveto{\pgfqpoint{0.721528in}{0.693354in}}%
\pgfpathlineto{\pgfqpoint{5.604377in}{0.693354in}}%
\pgfusepath{stroke}%
\end{pgfscope}%
\begin{pgfscope}%
\pgfsetrectcap%
\pgfsetmiterjoin%
\pgfsetlinewidth{0.803000pt}%
\definecolor{currentstroke}{rgb}{0.000000,0.000000,0.000000}%
\pgfsetstrokecolor{currentstroke}%
\pgfsetdash{}{0pt}%
\pgfpathmoveto{\pgfqpoint{0.721528in}{1.882564in}}%
\pgfpathlineto{\pgfqpoint{5.604377in}{1.882564in}}%
\pgfusepath{stroke}%
\end{pgfscope}%
\end{pgfpicture}%
\makeatother%
\endgroup%

    \end{center}
\end{figure}

A kutatásomban azt feltételeztem, hogy az amatőr videók jobban hasonlítanak az emberek valódi szexuális viselkedésére, mivel előre nem megszervezett jeleneteket forgatnak le bennük. Ebből fakadóan arra lehetne számítani, hogy ahol magas volt az amatőr videók aránya, ott magasabb volt az óvszerhasználat is.

Az \ref{condom.amateur}. ábra alapján ez az összefüggés azonban nem áll fent, sőt pont az ellenkezője tapasztalható. Ahol magasabb volt az amatőr videók nézettségének aránya, ott alacsonyabb volt az óvszerhasználat. 

A jelenség egy lehetséges magyarázata lehet, hogy az amatőr videók valójában pont a szokatlan és egyedi tapasztalatokat jelenítik meg, amelyekben eleve kevesebben használtak óvszert, de az is lehetséges, hogy az amatőr videók aránya nincs kapcsolatban az óvszerhasználattal és csupán véletlenszerű a megfigyelt összefüggés.

\begin{figure}[h]
    \caption[Amatőr videók és óvszerhasználat]{\footnotesize{Az óvszerhasználat és az amatőr videók nézettségi arányának alakulása. Forrás: saját ábra}}
    \label{condom.amateur}
    \begin{center}
        %% Creator: Matplotlib, PGF backend
%%
%% To include the figure in your LaTeX document, write
%%   \input{<filename>.pgf}
%%
%% Make sure the required packages are loaded in your preamble
%%   \usepackage{pgf}
%%
%% Figures using additional raster images can only be included by \input if
%% they are in the same directory as the main LaTeX file. For loading figures
%% from other directories you can use the `import` package
%%   \usepackage{import}
%%
%% and then include the figures with
%%   \import{<path to file>}{<filename>.pgf}
%%
%% Matplotlib used the following preamble
%%
\begingroup%
\makeatletter%
\begin{pgfpicture}%
\pgfpathrectangle{\pgfpointorigin}{\pgfqpoint{6.192803in}{1.992705in}}%
\pgfusepath{use as bounding box, clip}%
\begin{pgfscope}%
\pgfsetbuttcap%
\pgfsetmiterjoin%
\pgfsetlinewidth{0.000000pt}%
\definecolor{currentstroke}{rgb}{1.000000,1.000000,1.000000}%
\pgfsetstrokecolor{currentstroke}%
\pgfsetstrokeopacity{0.000000}%
\pgfsetdash{}{0pt}%
\pgfpathmoveto{\pgfqpoint{0.000000in}{0.000000in}}%
\pgfpathlineto{\pgfqpoint{6.192803in}{0.000000in}}%
\pgfpathlineto{\pgfqpoint{6.192803in}{1.992705in}}%
\pgfpathlineto{\pgfqpoint{0.000000in}{1.992705in}}%
\pgfpathclose%
\pgfusepath{}%
\end{pgfscope}%
\begin{pgfscope}%
\pgfsetbuttcap%
\pgfsetmiterjoin%
\definecolor{currentfill}{rgb}{1.000000,1.000000,1.000000}%
\pgfsetfillcolor{currentfill}%
\pgfsetlinewidth{0.000000pt}%
\definecolor{currentstroke}{rgb}{0.000000,0.000000,0.000000}%
\pgfsetstrokecolor{currentstroke}%
\pgfsetstrokeopacity{0.000000}%
\pgfsetdash{}{0pt}%
\pgfpathmoveto{\pgfqpoint{0.596528in}{0.693354in}}%
\pgfpathlineto{\pgfqpoint{5.479377in}{0.693354in}}%
\pgfpathlineto{\pgfqpoint{5.479377in}{1.882564in}}%
\pgfpathlineto{\pgfqpoint{0.596528in}{1.882564in}}%
\pgfpathclose%
\pgfusepath{fill}%
\end{pgfscope}%
\begin{pgfscope}%
\pgfpathrectangle{\pgfqpoint{0.596528in}{0.693354in}}{\pgfqpoint{4.882849in}{1.189210in}}%
\pgfusepath{clip}%
\pgfsetbuttcap%
\pgfsetmiterjoin%
\definecolor{currentfill}{rgb}{0.121569,0.466667,0.705882}%
\pgfsetfillcolor{currentfill}%
\pgfsetlinewidth{0.000000pt}%
\definecolor{currentstroke}{rgb}{0.000000,0.000000,0.000000}%
\pgfsetstrokecolor{currentstroke}%
\pgfsetstrokeopacity{0.000000}%
\pgfsetdash{}{0pt}%
\pgfpathmoveto{\pgfqpoint{0.909067in}{0.693354in}}%
\pgfpathlineto{\pgfqpoint{1.120445in}{0.693354in}}%
\pgfpathlineto{\pgfqpoint{1.120445in}{1.783987in}}%
\pgfpathlineto{\pgfqpoint{0.909067in}{1.783987in}}%
\pgfpathclose%
\pgfusepath{fill}%
\end{pgfscope}%
\begin{pgfscope}%
\pgfpathrectangle{\pgfqpoint{0.596528in}{0.693354in}}{\pgfqpoint{4.882849in}{1.189210in}}%
\pgfusepath{clip}%
\pgfsetbuttcap%
\pgfsetmiterjoin%
\definecolor{currentfill}{rgb}{0.121569,0.466667,0.705882}%
\pgfsetfillcolor{currentfill}%
\pgfsetlinewidth{0.000000pt}%
\definecolor{currentstroke}{rgb}{0.000000,0.000000,0.000000}%
\pgfsetstrokecolor{currentstroke}%
\pgfsetstrokeopacity{0.000000}%
\pgfsetdash{}{0pt}%
\pgfpathmoveto{\pgfqpoint{1.513006in}{0.693354in}}%
\pgfpathlineto{\pgfqpoint{1.724385in}{0.693354in}}%
\pgfpathlineto{\pgfqpoint{1.724385in}{1.825935in}}%
\pgfpathlineto{\pgfqpoint{1.513006in}{1.825935in}}%
\pgfpathclose%
\pgfusepath{fill}%
\end{pgfscope}%
\begin{pgfscope}%
\pgfpathrectangle{\pgfqpoint{0.596528in}{0.693354in}}{\pgfqpoint{4.882849in}{1.189210in}}%
\pgfusepath{clip}%
\pgfsetbuttcap%
\pgfsetmiterjoin%
\definecolor{currentfill}{rgb}{0.121569,0.466667,0.705882}%
\pgfsetfillcolor{currentfill}%
\pgfsetlinewidth{0.000000pt}%
\definecolor{currentstroke}{rgb}{0.000000,0.000000,0.000000}%
\pgfsetstrokecolor{currentstroke}%
\pgfsetstrokeopacity{0.000000}%
\pgfsetdash{}{0pt}%
\pgfpathmoveto{\pgfqpoint{2.116945in}{0.693354in}}%
\pgfpathlineto{\pgfqpoint{2.328324in}{0.693354in}}%
\pgfpathlineto{\pgfqpoint{2.328324in}{1.783987in}}%
\pgfpathlineto{\pgfqpoint{2.116945in}{1.783987in}}%
\pgfpathclose%
\pgfusepath{fill}%
\end{pgfscope}%
\begin{pgfscope}%
\pgfpathrectangle{\pgfqpoint{0.596528in}{0.693354in}}{\pgfqpoint{4.882849in}{1.189210in}}%
\pgfusepath{clip}%
\pgfsetbuttcap%
\pgfsetmiterjoin%
\definecolor{currentfill}{rgb}{0.121569,0.466667,0.705882}%
\pgfsetfillcolor{currentfill}%
\pgfsetlinewidth{0.000000pt}%
\definecolor{currentstroke}{rgb}{0.000000,0.000000,0.000000}%
\pgfsetstrokecolor{currentstroke}%
\pgfsetstrokeopacity{0.000000}%
\pgfsetdash{}{0pt}%
\pgfpathmoveto{\pgfqpoint{2.720884in}{0.693354in}}%
\pgfpathlineto{\pgfqpoint{2.932263in}{0.693354in}}%
\pgfpathlineto{\pgfqpoint{2.932263in}{1.742040in}}%
\pgfpathlineto{\pgfqpoint{2.720884in}{1.742040in}}%
\pgfpathclose%
\pgfusepath{fill}%
\end{pgfscope}%
\begin{pgfscope}%
\pgfpathrectangle{\pgfqpoint{0.596528in}{0.693354in}}{\pgfqpoint{4.882849in}{1.189210in}}%
\pgfusepath{clip}%
\pgfsetbuttcap%
\pgfsetmiterjoin%
\definecolor{currentfill}{rgb}{0.121569,0.466667,0.705882}%
\pgfsetfillcolor{currentfill}%
\pgfsetlinewidth{0.000000pt}%
\definecolor{currentstroke}{rgb}{0.000000,0.000000,0.000000}%
\pgfsetstrokecolor{currentstroke}%
\pgfsetstrokeopacity{0.000000}%
\pgfsetdash{}{0pt}%
\pgfpathmoveto{\pgfqpoint{3.324824in}{0.693354in}}%
\pgfpathlineto{\pgfqpoint{3.536202in}{0.693354in}}%
\pgfpathlineto{\pgfqpoint{3.536202in}{1.742040in}}%
\pgfpathlineto{\pgfqpoint{3.324824in}{1.742040in}}%
\pgfpathclose%
\pgfusepath{fill}%
\end{pgfscope}%
\begin{pgfscope}%
\pgfpathrectangle{\pgfqpoint{0.596528in}{0.693354in}}{\pgfqpoint{4.882849in}{1.189210in}}%
\pgfusepath{clip}%
\pgfsetbuttcap%
\pgfsetmiterjoin%
\definecolor{currentfill}{rgb}{0.121569,0.466667,0.705882}%
\pgfsetfillcolor{currentfill}%
\pgfsetlinewidth{0.000000pt}%
\definecolor{currentstroke}{rgb}{0.000000,0.000000,0.000000}%
\pgfsetstrokecolor{currentstroke}%
\pgfsetstrokeopacity{0.000000}%
\pgfsetdash{}{0pt}%
\pgfpathmoveto{\pgfqpoint{3.928763in}{0.693354in}}%
\pgfpathlineto{\pgfqpoint{4.140142in}{0.693354in}}%
\pgfpathlineto{\pgfqpoint{4.140142in}{1.742040in}}%
\pgfpathlineto{\pgfqpoint{3.928763in}{1.742040in}}%
\pgfpathclose%
\pgfusepath{fill}%
\end{pgfscope}%
\begin{pgfscope}%
\pgfpathrectangle{\pgfqpoint{0.596528in}{0.693354in}}{\pgfqpoint{4.882849in}{1.189210in}}%
\pgfusepath{clip}%
\pgfsetbuttcap%
\pgfsetmiterjoin%
\definecolor{currentfill}{rgb}{0.121569,0.466667,0.705882}%
\pgfsetfillcolor{currentfill}%
\pgfsetlinewidth{0.000000pt}%
\definecolor{currentstroke}{rgb}{0.000000,0.000000,0.000000}%
\pgfsetstrokecolor{currentstroke}%
\pgfsetstrokeopacity{0.000000}%
\pgfsetdash{}{0pt}%
\pgfpathmoveto{\pgfqpoint{4.834672in}{0.693354in}}%
\pgfpathlineto{\pgfqpoint{5.046051in}{0.693354in}}%
\pgfpathlineto{\pgfqpoint{5.046051in}{1.406460in}}%
\pgfpathlineto{\pgfqpoint{4.834672in}{1.406460in}}%
\pgfpathclose%
\pgfusepath{fill}%
\end{pgfscope}%
\begin{pgfscope}%
\pgfpathrectangle{\pgfqpoint{0.596528in}{0.693354in}}{\pgfqpoint{4.882849in}{1.189210in}}%
\pgfusepath{clip}%
\pgfsetbuttcap%
\pgfsetmiterjoin%
\definecolor{currentfill}{rgb}{1.000000,0.498039,0.054902}%
\pgfsetfillcolor{currentfill}%
\pgfsetlinewidth{0.000000pt}%
\definecolor{currentstroke}{rgb}{0.000000,0.000000,0.000000}%
\pgfsetstrokecolor{currentstroke}%
\pgfsetstrokeopacity{0.000000}%
\pgfsetdash{}{0pt}%
\pgfpathmoveto{\pgfqpoint{1.120445in}{0.693354in}}%
\pgfpathlineto{\pgfqpoint{1.331824in}{0.693354in}}%
\pgfpathlineto{\pgfqpoint{1.331824in}{1.658145in}}%
\pgfpathlineto{\pgfqpoint{1.120445in}{1.658145in}}%
\pgfpathclose%
\pgfusepath{fill}%
\end{pgfscope}%
\begin{pgfscope}%
\pgfpathrectangle{\pgfqpoint{0.596528in}{0.693354in}}{\pgfqpoint{4.882849in}{1.189210in}}%
\pgfusepath{clip}%
\pgfsetbuttcap%
\pgfsetmiterjoin%
\definecolor{currentfill}{rgb}{1.000000,0.498039,0.054902}%
\pgfsetfillcolor{currentfill}%
\pgfsetlinewidth{0.000000pt}%
\definecolor{currentstroke}{rgb}{0.000000,0.000000,0.000000}%
\pgfsetstrokecolor{currentstroke}%
\pgfsetstrokeopacity{0.000000}%
\pgfsetdash{}{0pt}%
\pgfpathmoveto{\pgfqpoint{1.724385in}{0.693354in}}%
\pgfpathlineto{\pgfqpoint{1.935763in}{0.693354in}}%
\pgfpathlineto{\pgfqpoint{1.935763in}{1.574250in}}%
\pgfpathlineto{\pgfqpoint{1.724385in}{1.574250in}}%
\pgfpathclose%
\pgfusepath{fill}%
\end{pgfscope}%
\begin{pgfscope}%
\pgfpathrectangle{\pgfqpoint{0.596528in}{0.693354in}}{\pgfqpoint{4.882849in}{1.189210in}}%
\pgfusepath{clip}%
\pgfsetbuttcap%
\pgfsetmiterjoin%
\definecolor{currentfill}{rgb}{1.000000,0.498039,0.054902}%
\pgfsetfillcolor{currentfill}%
\pgfsetlinewidth{0.000000pt}%
\definecolor{currentstroke}{rgb}{0.000000,0.000000,0.000000}%
\pgfsetstrokecolor{currentstroke}%
\pgfsetstrokeopacity{0.000000}%
\pgfsetdash{}{0pt}%
\pgfpathmoveto{\pgfqpoint{2.328324in}{0.693354in}}%
\pgfpathlineto{\pgfqpoint{2.539703in}{0.693354in}}%
\pgfpathlineto{\pgfqpoint{2.539703in}{1.448408in}}%
\pgfpathlineto{\pgfqpoint{2.328324in}{1.448408in}}%
\pgfpathclose%
\pgfusepath{fill}%
\end{pgfscope}%
\begin{pgfscope}%
\pgfpathrectangle{\pgfqpoint{0.596528in}{0.693354in}}{\pgfqpoint{4.882849in}{1.189210in}}%
\pgfusepath{clip}%
\pgfsetbuttcap%
\pgfsetmiterjoin%
\definecolor{currentfill}{rgb}{1.000000,0.498039,0.054902}%
\pgfsetfillcolor{currentfill}%
\pgfsetlinewidth{0.000000pt}%
\definecolor{currentstroke}{rgb}{0.000000,0.000000,0.000000}%
\pgfsetstrokecolor{currentstroke}%
\pgfsetstrokeopacity{0.000000}%
\pgfsetdash{}{0pt}%
\pgfpathmoveto{\pgfqpoint{2.932263in}{0.693354in}}%
\pgfpathlineto{\pgfqpoint{3.143642in}{0.693354in}}%
\pgfpathlineto{\pgfqpoint{3.143642in}{1.574250in}}%
\pgfpathlineto{\pgfqpoint{2.932263in}{1.574250in}}%
\pgfpathclose%
\pgfusepath{fill}%
\end{pgfscope}%
\begin{pgfscope}%
\pgfpathrectangle{\pgfqpoint{0.596528in}{0.693354in}}{\pgfqpoint{4.882849in}{1.189210in}}%
\pgfusepath{clip}%
\pgfsetbuttcap%
\pgfsetmiterjoin%
\definecolor{currentfill}{rgb}{1.000000,0.498039,0.054902}%
\pgfsetfillcolor{currentfill}%
\pgfsetlinewidth{0.000000pt}%
\definecolor{currentstroke}{rgb}{0.000000,0.000000,0.000000}%
\pgfsetstrokecolor{currentstroke}%
\pgfsetstrokeopacity{0.000000}%
\pgfsetdash{}{0pt}%
\pgfpathmoveto{\pgfqpoint{3.536202in}{0.693354in}}%
\pgfpathlineto{\pgfqpoint{3.747581in}{0.693354in}}%
\pgfpathlineto{\pgfqpoint{3.747581in}{1.532303in}}%
\pgfpathlineto{\pgfqpoint{3.536202in}{1.532303in}}%
\pgfpathclose%
\pgfusepath{fill}%
\end{pgfscope}%
\begin{pgfscope}%
\pgfpathrectangle{\pgfqpoint{0.596528in}{0.693354in}}{\pgfqpoint{4.882849in}{1.189210in}}%
\pgfusepath{clip}%
\pgfsetbuttcap%
\pgfsetmiterjoin%
\definecolor{currentfill}{rgb}{1.000000,0.498039,0.054902}%
\pgfsetfillcolor{currentfill}%
\pgfsetlinewidth{0.000000pt}%
\definecolor{currentstroke}{rgb}{0.000000,0.000000,0.000000}%
\pgfsetstrokecolor{currentstroke}%
\pgfsetstrokeopacity{0.000000}%
\pgfsetdash{}{0pt}%
\pgfpathmoveto{\pgfqpoint{4.140142in}{0.693354in}}%
\pgfpathlineto{\pgfqpoint{4.351520in}{0.693354in}}%
\pgfpathlineto{\pgfqpoint{4.351520in}{1.406460in}}%
\pgfpathlineto{\pgfqpoint{4.140142in}{1.406460in}}%
\pgfpathclose%
\pgfusepath{fill}%
\end{pgfscope}%
\begin{pgfscope}%
\pgfpathrectangle{\pgfqpoint{0.596528in}{0.693354in}}{\pgfqpoint{4.882849in}{1.189210in}}%
\pgfusepath{clip}%
\pgfsetbuttcap%
\pgfsetmiterjoin%
\definecolor{currentfill}{rgb}{1.000000,0.498039,0.054902}%
\pgfsetfillcolor{currentfill}%
\pgfsetlinewidth{0.000000pt}%
\definecolor{currentstroke}{rgb}{0.000000,0.000000,0.000000}%
\pgfsetstrokecolor{currentstroke}%
\pgfsetstrokeopacity{0.000000}%
\pgfsetdash{}{0pt}%
\pgfpathmoveto{\pgfqpoint{5.046051in}{0.693354in}}%
\pgfpathlineto{\pgfqpoint{5.257429in}{0.693354in}}%
\pgfpathlineto{\pgfqpoint{5.257429in}{1.364513in}}%
\pgfpathlineto{\pgfqpoint{5.046051in}{1.364513in}}%
\pgfpathclose%
\pgfusepath{fill}%
\end{pgfscope}%
\begin{pgfscope}%
\pgfsetbuttcap%
\pgfsetroundjoin%
\definecolor{currentfill}{rgb}{0.000000,0.000000,0.000000}%
\pgfsetfillcolor{currentfill}%
\pgfsetlinewidth{0.803000pt}%
\definecolor{currentstroke}{rgb}{0.000000,0.000000,0.000000}%
\pgfsetstrokecolor{currentstroke}%
\pgfsetdash{}{0pt}%
\pgfsys@defobject{currentmarker}{\pgfqpoint{0.000000in}{-0.048611in}}{\pgfqpoint{0.000000in}{0.000000in}}{%
\pgfpathmoveto{\pgfqpoint{0.000000in}{0.000000in}}%
\pgfpathlineto{\pgfqpoint{0.000000in}{-0.048611in}}%
\pgfusepath{stroke,fill}%
}%
\begin{pgfscope}%
\pgfsys@transformshift{1.120445in}{0.693354in}%
\pgfsys@useobject{currentmarker}{}%
\end{pgfscope}%
\end{pgfscope}%
\begin{pgfscope}%
\definecolor{textcolor}{rgb}{0.000000,0.000000,0.000000}%
\pgfsetstrokecolor{textcolor}%
\pgfsetfillcolor{textcolor}%
\pgftext[x=1.120445in,y=0.596131in,,top]{\color{textcolor}\rmfamily\fontsize{10.000000}{12.000000}\selectfont \(\displaystyle {2008}\)}%
\end{pgfscope}%
\begin{pgfscope}%
\pgfsetbuttcap%
\pgfsetroundjoin%
\definecolor{currentfill}{rgb}{0.000000,0.000000,0.000000}%
\pgfsetfillcolor{currentfill}%
\pgfsetlinewidth{0.803000pt}%
\definecolor{currentstroke}{rgb}{0.000000,0.000000,0.000000}%
\pgfsetstrokecolor{currentstroke}%
\pgfsetdash{}{0pt}%
\pgfsys@defobject{currentmarker}{\pgfqpoint{0.000000in}{-0.048611in}}{\pgfqpoint{0.000000in}{0.000000in}}{%
\pgfpathmoveto{\pgfqpoint{0.000000in}{0.000000in}}%
\pgfpathlineto{\pgfqpoint{0.000000in}{-0.048611in}}%
\pgfusepath{stroke,fill}%
}%
\begin{pgfscope}%
\pgfsys@transformshift{1.724385in}{0.693354in}%
\pgfsys@useobject{currentmarker}{}%
\end{pgfscope}%
\end{pgfscope}%
\begin{pgfscope}%
\definecolor{textcolor}{rgb}{0.000000,0.000000,0.000000}%
\pgfsetstrokecolor{textcolor}%
\pgfsetfillcolor{textcolor}%
\pgftext[x=1.724385in,y=0.596131in,,top]{\color{textcolor}\rmfamily\fontsize{10.000000}{12.000000}\selectfont \(\displaystyle {2010}\)}%
\end{pgfscope}%
\begin{pgfscope}%
\pgfsetbuttcap%
\pgfsetroundjoin%
\definecolor{currentfill}{rgb}{0.000000,0.000000,0.000000}%
\pgfsetfillcolor{currentfill}%
\pgfsetlinewidth{0.803000pt}%
\definecolor{currentstroke}{rgb}{0.000000,0.000000,0.000000}%
\pgfsetstrokecolor{currentstroke}%
\pgfsetdash{}{0pt}%
\pgfsys@defobject{currentmarker}{\pgfqpoint{0.000000in}{-0.048611in}}{\pgfqpoint{0.000000in}{0.000000in}}{%
\pgfpathmoveto{\pgfqpoint{0.000000in}{0.000000in}}%
\pgfpathlineto{\pgfqpoint{0.000000in}{-0.048611in}}%
\pgfusepath{stroke,fill}%
}%
\begin{pgfscope}%
\pgfsys@transformshift{2.328324in}{0.693354in}%
\pgfsys@useobject{currentmarker}{}%
\end{pgfscope}%
\end{pgfscope}%
\begin{pgfscope}%
\definecolor{textcolor}{rgb}{0.000000,0.000000,0.000000}%
\pgfsetstrokecolor{textcolor}%
\pgfsetfillcolor{textcolor}%
\pgftext[x=2.328324in,y=0.596131in,,top]{\color{textcolor}\rmfamily\fontsize{10.000000}{12.000000}\selectfont \(\displaystyle {2012}\)}%
\end{pgfscope}%
\begin{pgfscope}%
\pgfsetbuttcap%
\pgfsetroundjoin%
\definecolor{currentfill}{rgb}{0.000000,0.000000,0.000000}%
\pgfsetfillcolor{currentfill}%
\pgfsetlinewidth{0.803000pt}%
\definecolor{currentstroke}{rgb}{0.000000,0.000000,0.000000}%
\pgfsetstrokecolor{currentstroke}%
\pgfsetdash{}{0pt}%
\pgfsys@defobject{currentmarker}{\pgfqpoint{0.000000in}{-0.048611in}}{\pgfqpoint{0.000000in}{0.000000in}}{%
\pgfpathmoveto{\pgfqpoint{0.000000in}{0.000000in}}%
\pgfpathlineto{\pgfqpoint{0.000000in}{-0.048611in}}%
\pgfusepath{stroke,fill}%
}%
\begin{pgfscope}%
\pgfsys@transformshift{2.932263in}{0.693354in}%
\pgfsys@useobject{currentmarker}{}%
\end{pgfscope}%
\end{pgfscope}%
\begin{pgfscope}%
\definecolor{textcolor}{rgb}{0.000000,0.000000,0.000000}%
\pgfsetstrokecolor{textcolor}%
\pgfsetfillcolor{textcolor}%
\pgftext[x=2.932263in,y=0.596131in,,top]{\color{textcolor}\rmfamily\fontsize{10.000000}{12.000000}\selectfont \(\displaystyle {2014}\)}%
\end{pgfscope}%
\begin{pgfscope}%
\pgfsetbuttcap%
\pgfsetroundjoin%
\definecolor{currentfill}{rgb}{0.000000,0.000000,0.000000}%
\pgfsetfillcolor{currentfill}%
\pgfsetlinewidth{0.803000pt}%
\definecolor{currentstroke}{rgb}{0.000000,0.000000,0.000000}%
\pgfsetstrokecolor{currentstroke}%
\pgfsetdash{}{0pt}%
\pgfsys@defobject{currentmarker}{\pgfqpoint{0.000000in}{-0.048611in}}{\pgfqpoint{0.000000in}{0.000000in}}{%
\pgfpathmoveto{\pgfqpoint{0.000000in}{0.000000in}}%
\pgfpathlineto{\pgfqpoint{0.000000in}{-0.048611in}}%
\pgfusepath{stroke,fill}%
}%
\begin{pgfscope}%
\pgfsys@transformshift{3.536202in}{0.693354in}%
\pgfsys@useobject{currentmarker}{}%
\end{pgfscope}%
\end{pgfscope}%
\begin{pgfscope}%
\definecolor{textcolor}{rgb}{0.000000,0.000000,0.000000}%
\pgfsetstrokecolor{textcolor}%
\pgfsetfillcolor{textcolor}%
\pgftext[x=3.536202in,y=0.596131in,,top]{\color{textcolor}\rmfamily\fontsize{10.000000}{12.000000}\selectfont \(\displaystyle {2016}\)}%
\end{pgfscope}%
\begin{pgfscope}%
\pgfsetbuttcap%
\pgfsetroundjoin%
\definecolor{currentfill}{rgb}{0.000000,0.000000,0.000000}%
\pgfsetfillcolor{currentfill}%
\pgfsetlinewidth{0.803000pt}%
\definecolor{currentstroke}{rgb}{0.000000,0.000000,0.000000}%
\pgfsetstrokecolor{currentstroke}%
\pgfsetdash{}{0pt}%
\pgfsys@defobject{currentmarker}{\pgfqpoint{0.000000in}{-0.048611in}}{\pgfqpoint{0.000000in}{0.000000in}}{%
\pgfpathmoveto{\pgfqpoint{0.000000in}{0.000000in}}%
\pgfpathlineto{\pgfqpoint{0.000000in}{-0.048611in}}%
\pgfusepath{stroke,fill}%
}%
\begin{pgfscope}%
\pgfsys@transformshift{4.140142in}{0.693354in}%
\pgfsys@useobject{currentmarker}{}%
\end{pgfscope}%
\end{pgfscope}%
\begin{pgfscope}%
\definecolor{textcolor}{rgb}{0.000000,0.000000,0.000000}%
\pgfsetstrokecolor{textcolor}%
\pgfsetfillcolor{textcolor}%
\pgftext[x=4.140142in,y=0.596131in,,top]{\color{textcolor}\rmfamily\fontsize{10.000000}{12.000000}\selectfont \(\displaystyle {2018}\)}%
\end{pgfscope}%
\begin{pgfscope}%
\pgfsetbuttcap%
\pgfsetroundjoin%
\definecolor{currentfill}{rgb}{0.000000,0.000000,0.000000}%
\pgfsetfillcolor{currentfill}%
\pgfsetlinewidth{0.803000pt}%
\definecolor{currentstroke}{rgb}{0.000000,0.000000,0.000000}%
\pgfsetstrokecolor{currentstroke}%
\pgfsetdash{}{0pt}%
\pgfsys@defobject{currentmarker}{\pgfqpoint{0.000000in}{-0.048611in}}{\pgfqpoint{0.000000in}{0.000000in}}{%
\pgfpathmoveto{\pgfqpoint{0.000000in}{0.000000in}}%
\pgfpathlineto{\pgfqpoint{0.000000in}{-0.048611in}}%
\pgfusepath{stroke,fill}%
}%
\begin{pgfscope}%
\pgfsys@transformshift{5.046051in}{0.693354in}%
\pgfsys@useobject{currentmarker}{}%
\end{pgfscope}%
\end{pgfscope}%
\begin{pgfscope}%
\definecolor{textcolor}{rgb}{0.000000,0.000000,0.000000}%
\pgfsetstrokecolor{textcolor}%
\pgfsetfillcolor{textcolor}%
\pgftext[x=5.046051in,y=0.596131in,,top]{\color{textcolor}\rmfamily\fontsize{10.000000}{12.000000}\selectfont \(\displaystyle {2021}\)}%
\end{pgfscope}%
\begin{pgfscope}%
\pgfsetbuttcap%
\pgfsetroundjoin%
\definecolor{currentfill}{rgb}{0.000000,0.000000,0.000000}%
\pgfsetfillcolor{currentfill}%
\pgfsetlinewidth{0.803000pt}%
\definecolor{currentstroke}{rgb}{0.000000,0.000000,0.000000}%
\pgfsetstrokecolor{currentstroke}%
\pgfsetdash{}{0pt}%
\pgfsys@defobject{currentmarker}{\pgfqpoint{-0.048611in}{0.000000in}}{\pgfqpoint{-0.000000in}{0.000000in}}{%
\pgfpathmoveto{\pgfqpoint{-0.000000in}{0.000000in}}%
\pgfpathlineto{\pgfqpoint{-0.048611in}{0.000000in}}%
\pgfusepath{stroke,fill}%
}%
\begin{pgfscope}%
\pgfsys@transformshift{0.596528in}{0.693354in}%
\pgfsys@useobject{currentmarker}{}%
\end{pgfscope}%
\end{pgfscope}%
\begin{pgfscope}%
\definecolor{textcolor}{rgb}{0.000000,0.000000,0.000000}%
\pgfsetstrokecolor{textcolor}%
\pgfsetfillcolor{textcolor}%
\pgftext[x=0.321836in, y=0.645128in, left, base]{\color{textcolor}\rmfamily\fontsize{10.000000}{12.000000}\selectfont \(\displaystyle {0.0}\)}%
\end{pgfscope}%
\begin{pgfscope}%
\pgfsetbuttcap%
\pgfsetroundjoin%
\definecolor{currentfill}{rgb}{0.000000,0.000000,0.000000}%
\pgfsetfillcolor{currentfill}%
\pgfsetlinewidth{0.803000pt}%
\definecolor{currentstroke}{rgb}{0.000000,0.000000,0.000000}%
\pgfsetstrokecolor{currentstroke}%
\pgfsetdash{}{0pt}%
\pgfsys@defobject{currentmarker}{\pgfqpoint{-0.048611in}{0.000000in}}{\pgfqpoint{-0.000000in}{0.000000in}}{%
\pgfpathmoveto{\pgfqpoint{-0.000000in}{0.000000in}}%
\pgfpathlineto{\pgfqpoint{-0.048611in}{0.000000in}}%
\pgfusepath{stroke,fill}%
}%
\begin{pgfscope}%
\pgfsys@transformshift{0.596528in}{1.112828in}%
\pgfsys@useobject{currentmarker}{}%
\end{pgfscope}%
\end{pgfscope}%
\begin{pgfscope}%
\definecolor{textcolor}{rgb}{0.000000,0.000000,0.000000}%
\pgfsetstrokecolor{textcolor}%
\pgfsetfillcolor{textcolor}%
\pgftext[x=0.321836in, y=1.064603in, left, base]{\color{textcolor}\rmfamily\fontsize{10.000000}{12.000000}\selectfont \(\displaystyle {0.1}\)}%
\end{pgfscope}%
\begin{pgfscope}%
\pgfsetbuttcap%
\pgfsetroundjoin%
\definecolor{currentfill}{rgb}{0.000000,0.000000,0.000000}%
\pgfsetfillcolor{currentfill}%
\pgfsetlinewidth{0.803000pt}%
\definecolor{currentstroke}{rgb}{0.000000,0.000000,0.000000}%
\pgfsetstrokecolor{currentstroke}%
\pgfsetdash{}{0pt}%
\pgfsys@defobject{currentmarker}{\pgfqpoint{-0.048611in}{0.000000in}}{\pgfqpoint{-0.000000in}{0.000000in}}{%
\pgfpathmoveto{\pgfqpoint{-0.000000in}{0.000000in}}%
\pgfpathlineto{\pgfqpoint{-0.048611in}{0.000000in}}%
\pgfusepath{stroke,fill}%
}%
\begin{pgfscope}%
\pgfsys@transformshift{0.596528in}{1.532303in}%
\pgfsys@useobject{currentmarker}{}%
\end{pgfscope}%
\end{pgfscope}%
\begin{pgfscope}%
\definecolor{textcolor}{rgb}{0.000000,0.000000,0.000000}%
\pgfsetstrokecolor{textcolor}%
\pgfsetfillcolor{textcolor}%
\pgftext[x=0.321836in, y=1.484077in, left, base]{\color{textcolor}\rmfamily\fontsize{10.000000}{12.000000}\selectfont \(\displaystyle {0.2}\)}%
\end{pgfscope}%
\begin{pgfscope}%
\definecolor{textcolor}{rgb}{0.000000,0.000000,0.000000}%
\pgfsetstrokecolor{textcolor}%
\pgfsetfillcolor{textcolor}%
\pgftext[x=0.266281in,y=1.287959in,,bottom,rotate=90.000000]{\color{textcolor}\rmfamily\fontsize{10.000000}{12.000000}\selectfont Óvszerhasználat(\%)}%
\end{pgfscope}%
\begin{pgfscope}%
\pgfsetrectcap%
\pgfsetmiterjoin%
\pgfsetlinewidth{0.803000pt}%
\definecolor{currentstroke}{rgb}{0.000000,0.000000,0.000000}%
\pgfsetstrokecolor{currentstroke}%
\pgfsetdash{}{0pt}%
\pgfpathmoveto{\pgfqpoint{0.596528in}{0.693354in}}%
\pgfpathlineto{\pgfqpoint{0.596528in}{1.882564in}}%
\pgfusepath{stroke}%
\end{pgfscope}%
\begin{pgfscope}%
\pgfsetrectcap%
\pgfsetmiterjoin%
\pgfsetlinewidth{0.803000pt}%
\definecolor{currentstroke}{rgb}{0.000000,0.000000,0.000000}%
\pgfsetstrokecolor{currentstroke}%
\pgfsetdash{}{0pt}%
\pgfpathmoveto{\pgfqpoint{5.479377in}{0.693354in}}%
\pgfpathlineto{\pgfqpoint{5.479377in}{1.882564in}}%
\pgfusepath{stroke}%
\end{pgfscope}%
\begin{pgfscope}%
\pgfsetrectcap%
\pgfsetmiterjoin%
\pgfsetlinewidth{0.803000pt}%
\definecolor{currentstroke}{rgb}{0.000000,0.000000,0.000000}%
\pgfsetstrokecolor{currentstroke}%
\pgfsetdash{}{0pt}%
\pgfpathmoveto{\pgfqpoint{0.596528in}{0.693354in}}%
\pgfpathlineto{\pgfqpoint{5.479377in}{0.693354in}}%
\pgfusepath{stroke}%
\end{pgfscope}%
\begin{pgfscope}%
\pgfsetrectcap%
\pgfsetmiterjoin%
\pgfsetlinewidth{0.803000pt}%
\definecolor{currentstroke}{rgb}{0.000000,0.000000,0.000000}%
\pgfsetstrokecolor{currentstroke}%
\pgfsetdash{}{0pt}%
\pgfpathmoveto{\pgfqpoint{0.596528in}{1.882564in}}%
\pgfpathlineto{\pgfqpoint{5.479377in}{1.882564in}}%
\pgfusepath{stroke}%
\end{pgfscope}%
\begin{pgfscope}%
\definecolor{textcolor}{rgb}{1.000000,1.000000,1.000000}%
\pgfsetstrokecolor{textcolor}%
\pgfsetfillcolor{textcolor}%
\pgftext[x=1.042534in, y=1.133731in, left, base,rotate=90.000000]{\color{textcolor}\rmfamily\fontsize{8.000000}{9.600000}\selectfont 0.26}%
\end{pgfscope}%
\begin{pgfscope}%
\definecolor{textcolor}{rgb}{1.000000,1.000000,1.000000}%
\pgfsetstrokecolor{textcolor}%
\pgfsetfillcolor{textcolor}%
\pgftext[x=1.646473in, y=1.154704in, left, base,rotate=90.000000]{\color{textcolor}\rmfamily\fontsize{8.000000}{9.600000}\selectfont 0.27}%
\end{pgfscope}%
\begin{pgfscope}%
\definecolor{textcolor}{rgb}{1.000000,1.000000,1.000000}%
\pgfsetstrokecolor{textcolor}%
\pgfsetfillcolor{textcolor}%
\pgftext[x=2.250412in, y=1.133731in, left, base,rotate=90.000000]{\color{textcolor}\rmfamily\fontsize{8.000000}{9.600000}\selectfont 0.26}%
\end{pgfscope}%
\begin{pgfscope}%
\definecolor{textcolor}{rgb}{1.000000,1.000000,1.000000}%
\pgfsetstrokecolor{textcolor}%
\pgfsetfillcolor{textcolor}%
\pgftext[x=2.854352in, y=1.112757in, left, base,rotate=90.000000]{\color{textcolor}\rmfamily\fontsize{8.000000}{9.600000}\selectfont 0.25}%
\end{pgfscope}%
\begin{pgfscope}%
\definecolor{textcolor}{rgb}{1.000000,1.000000,1.000000}%
\pgfsetstrokecolor{textcolor}%
\pgfsetfillcolor{textcolor}%
\pgftext[x=3.458291in, y=1.112757in, left, base,rotate=90.000000]{\color{textcolor}\rmfamily\fontsize{8.000000}{9.600000}\selectfont 0.25}%
\end{pgfscope}%
\begin{pgfscope}%
\definecolor{textcolor}{rgb}{1.000000,1.000000,1.000000}%
\pgfsetstrokecolor{textcolor}%
\pgfsetfillcolor{textcolor}%
\pgftext[x=4.062230in, y=1.112757in, left, base,rotate=90.000000]{\color{textcolor}\rmfamily\fontsize{8.000000}{9.600000}\selectfont 0.25}%
\end{pgfscope}%
\begin{pgfscope}%
\definecolor{textcolor}{rgb}{1.000000,1.000000,1.000000}%
\pgfsetstrokecolor{textcolor}%
\pgfsetfillcolor{textcolor}%
\pgftext[x=4.968139in, y=0.944967in, left, base,rotate=90.000000]{\color{textcolor}\rmfamily\fontsize{8.000000}{9.600000}\selectfont 0.17}%
\end{pgfscope}%
\begin{pgfscope}%
\definecolor{textcolor}{rgb}{1.000000,1.000000,1.000000}%
\pgfsetstrokecolor{textcolor}%
\pgfsetfillcolor{textcolor}%
\pgftext[x=1.253913in, y=1.070809in, left, base,rotate=90.000000]{\color{textcolor}\rmfamily\fontsize{8.000000}{9.600000}\selectfont 0.23}%
\end{pgfscope}%
\begin{pgfscope}%
\definecolor{textcolor}{rgb}{1.000000,1.000000,1.000000}%
\pgfsetstrokecolor{textcolor}%
\pgfsetfillcolor{textcolor}%
\pgftext[x=1.857852in, y=1.028862in, left, base,rotate=90.000000]{\color{textcolor}\rmfamily\fontsize{8.000000}{9.600000}\selectfont 0.21}%
\end{pgfscope}%
\begin{pgfscope}%
\definecolor{textcolor}{rgb}{1.000000,1.000000,1.000000}%
\pgfsetstrokecolor{textcolor}%
\pgfsetfillcolor{textcolor}%
\pgftext[x=2.461791in, y=0.965941in, left, base,rotate=90.000000]{\color{textcolor}\rmfamily\fontsize{8.000000}{9.600000}\selectfont 0.18}%
\end{pgfscope}%
\begin{pgfscope}%
\definecolor{textcolor}{rgb}{1.000000,1.000000,1.000000}%
\pgfsetstrokecolor{textcolor}%
\pgfsetfillcolor{textcolor}%
\pgftext[x=3.065730in, y=1.028862in, left, base,rotate=90.000000]{\color{textcolor}\rmfamily\fontsize{8.000000}{9.600000}\selectfont 0.21}%
\end{pgfscope}%
\begin{pgfscope}%
\definecolor{textcolor}{rgb}{1.000000,1.000000,1.000000}%
\pgfsetstrokecolor{textcolor}%
\pgfsetfillcolor{textcolor}%
\pgftext[x=3.669670in, y=1.037403in, left, base,rotate=90.000000]{\color{textcolor}\rmfamily\fontsize{8.000000}{9.600000}\selectfont 0.2}%
\end{pgfscope}%
\begin{pgfscope}%
\definecolor{textcolor}{rgb}{1.000000,1.000000,1.000000}%
\pgfsetstrokecolor{textcolor}%
\pgfsetfillcolor{textcolor}%
\pgftext[x=4.273609in, y=0.944967in, left, base,rotate=90.000000]{\color{textcolor}\rmfamily\fontsize{8.000000}{9.600000}\selectfont 0.17}%
\end{pgfscope}%
\begin{pgfscope}%
\definecolor{textcolor}{rgb}{1.000000,1.000000,1.000000}%
\pgfsetstrokecolor{textcolor}%
\pgfsetfillcolor{textcolor}%
\pgftext[x=5.179518in, y=0.923993in, left, base,rotate=90.000000]{\color{textcolor}\rmfamily\fontsize{8.000000}{9.600000}\selectfont 0.16}%
\end{pgfscope}%
\begin{pgfscope}%
\pgfsetbuttcap%
\pgfsetmiterjoin%
\pgfsetlinewidth{0.000000pt}%
\definecolor{currentstroke}{rgb}{0.800000,0.800000,0.800000}%
\pgfsetstrokecolor{currentstroke}%
\pgfsetstrokeopacity{0.000000}%
\pgfsetdash{}{0pt}%
\pgfpathmoveto{\pgfqpoint{2.253616in}{0.100000in}}%
\pgfpathlineto{\pgfqpoint{3.822290in}{0.100000in}}%
\pgfpathquadraticcurveto{\pgfqpoint{3.850067in}{0.100000in}}{\pgfqpoint{3.850067in}{0.127778in}}%
\pgfpathlineto{\pgfqpoint{3.850067in}{0.307562in}}%
\pgfpathquadraticcurveto{\pgfqpoint{3.850067in}{0.335339in}}{\pgfqpoint{3.822290in}{0.335339in}}%
\pgfpathlineto{\pgfqpoint{2.253616in}{0.335339in}}%
\pgfpathquadraticcurveto{\pgfqpoint{2.225838in}{0.335339in}}{\pgfqpoint{2.225838in}{0.307562in}}%
\pgfpathlineto{\pgfqpoint{2.225838in}{0.127778in}}%
\pgfpathquadraticcurveto{\pgfqpoint{2.225838in}{0.100000in}}{\pgfqpoint{2.253616in}{0.100000in}}%
\pgfpathclose%
\pgfusepath{}%
\end{pgfscope}%
\begin{pgfscope}%
\pgfsetbuttcap%
\pgfsetmiterjoin%
\definecolor{currentfill}{rgb}{0.121569,0.466667,0.705882}%
\pgfsetfillcolor{currentfill}%
\pgfsetlinewidth{0.000000pt}%
\definecolor{currentstroke}{rgb}{0.000000,0.000000,0.000000}%
\pgfsetstrokecolor{currentstroke}%
\pgfsetstrokeopacity{0.000000}%
\pgfsetdash{}{0pt}%
\pgfpathmoveto{\pgfqpoint{2.281393in}{0.182562in}}%
\pgfpathlineto{\pgfqpoint{2.559171in}{0.182562in}}%
\pgfpathlineto{\pgfqpoint{2.559171in}{0.279784in}}%
\pgfpathlineto{\pgfqpoint{2.281393in}{0.279784in}}%
\pgfpathclose%
\pgfusepath{fill}%
\end{pgfscope}%
\begin{pgfscope}%
\definecolor{textcolor}{rgb}{0.000000,0.000000,0.000000}%
\pgfsetstrokecolor{textcolor}%
\pgfsetfillcolor{textcolor}%
\pgftext[x=2.670282in,y=0.182562in,left,base]{\color{textcolor}\rmfamily\fontsize{10.000000}{12.000000}\selectfont Férfi}%
\end{pgfscope}%
\begin{pgfscope}%
\pgfsetbuttcap%
\pgfsetmiterjoin%
\definecolor{currentfill}{rgb}{1.000000,0.498039,0.054902}%
\pgfsetfillcolor{currentfill}%
\pgfsetlinewidth{0.000000pt}%
\definecolor{currentstroke}{rgb}{0.000000,0.000000,0.000000}%
\pgfsetstrokecolor{currentstroke}%
\pgfsetstrokeopacity{0.000000}%
\pgfsetdash{}{0pt}%
\pgfpathmoveto{\pgfqpoint{3.232011in}{0.182562in}}%
\pgfpathlineto{\pgfqpoint{3.509789in}{0.182562in}}%
\pgfpathlineto{\pgfqpoint{3.509789in}{0.279784in}}%
\pgfpathlineto{\pgfqpoint{3.232011in}{0.279784in}}%
\pgfpathclose%
\pgfusepath{fill}%
\end{pgfscope}%
\begin{pgfscope}%
\definecolor{textcolor}{rgb}{0.000000,0.000000,0.000000}%
\pgfsetstrokecolor{textcolor}%
\pgfsetfillcolor{textcolor}%
\pgftext[x=3.620900in,y=0.182562in,left,base]{\color{textcolor}\rmfamily\fontsize{10.000000}{12.000000}\selectfont Nő}%
\end{pgfscope}%
\begin{pgfscope}%
\pgfsetbuttcap%
\pgfsetroundjoin%
\definecolor{currentfill}{rgb}{0.000000,0.000000,0.000000}%
\pgfsetfillcolor{currentfill}%
\pgfsetlinewidth{0.803000pt}%
\definecolor{currentstroke}{rgb}{0.000000,0.000000,0.000000}%
\pgfsetstrokecolor{currentstroke}%
\pgfsetdash{}{0pt}%
\pgfsys@defobject{currentmarker}{\pgfqpoint{0.000000in}{0.000000in}}{\pgfqpoint{0.048611in}{0.000000in}}{%
\pgfpathmoveto{\pgfqpoint{0.000000in}{0.000000in}}%
\pgfpathlineto{\pgfqpoint{0.048611in}{0.000000in}}%
\pgfusepath{stroke,fill}%
}%
\begin{pgfscope}%
\pgfsys@transformshift{5.479377in}{0.737523in}%
\pgfsys@useobject{currentmarker}{}%
\end{pgfscope}%
\end{pgfscope}%
\begin{pgfscope}%
\definecolor{textcolor}{rgb}{0.000000,0.000000,0.000000}%
\pgfsetstrokecolor{textcolor}%
\pgfsetfillcolor{textcolor}%
\pgftext[x=5.576599in, y=0.689298in, left, base]{\color{textcolor}\rmfamily\fontsize{10.000000}{12.000000}\selectfont \(\displaystyle {0.0}\)}%
\end{pgfscope}%
\begin{pgfscope}%
\pgfsetbuttcap%
\pgfsetroundjoin%
\definecolor{currentfill}{rgb}{0.000000,0.000000,0.000000}%
\pgfsetfillcolor{currentfill}%
\pgfsetlinewidth{0.803000pt}%
\definecolor{currentstroke}{rgb}{0.000000,0.000000,0.000000}%
\pgfsetstrokecolor{currentstroke}%
\pgfsetdash{}{0pt}%
\pgfsys@defobject{currentmarker}{\pgfqpoint{0.000000in}{0.000000in}}{\pgfqpoint{0.048611in}{0.000000in}}{%
\pgfpathmoveto{\pgfqpoint{0.000000in}{0.000000in}}%
\pgfpathlineto{\pgfqpoint{0.048611in}{0.000000in}}%
\pgfusepath{stroke,fill}%
}%
\begin{pgfscope}%
\pgfsys@transformshift{5.479377in}{1.119337in}%
\pgfsys@useobject{currentmarker}{}%
\end{pgfscope}%
\end{pgfscope}%
\begin{pgfscope}%
\definecolor{textcolor}{rgb}{0.000000,0.000000,0.000000}%
\pgfsetstrokecolor{textcolor}%
\pgfsetfillcolor{textcolor}%
\pgftext[x=5.576599in, y=1.071112in, left, base]{\color{textcolor}\rmfamily\fontsize{10.000000}{12.000000}\selectfont \(\displaystyle {0.2}\)}%
\end{pgfscope}%
\begin{pgfscope}%
\pgfsetbuttcap%
\pgfsetroundjoin%
\definecolor{currentfill}{rgb}{0.000000,0.000000,0.000000}%
\pgfsetfillcolor{currentfill}%
\pgfsetlinewidth{0.803000pt}%
\definecolor{currentstroke}{rgb}{0.000000,0.000000,0.000000}%
\pgfsetstrokecolor{currentstroke}%
\pgfsetdash{}{0pt}%
\pgfsys@defobject{currentmarker}{\pgfqpoint{0.000000in}{0.000000in}}{\pgfqpoint{0.048611in}{0.000000in}}{%
\pgfpathmoveto{\pgfqpoint{0.000000in}{0.000000in}}%
\pgfpathlineto{\pgfqpoint{0.048611in}{0.000000in}}%
\pgfusepath{stroke,fill}%
}%
\begin{pgfscope}%
\pgfsys@transformshift{5.479377in}{1.501152in}%
\pgfsys@useobject{currentmarker}{}%
\end{pgfscope}%
\end{pgfscope}%
\begin{pgfscope}%
\definecolor{textcolor}{rgb}{0.000000,0.000000,0.000000}%
\pgfsetstrokecolor{textcolor}%
\pgfsetfillcolor{textcolor}%
\pgftext[x=5.576599in, y=1.452927in, left, base]{\color{textcolor}\rmfamily\fontsize{10.000000}{12.000000}\selectfont \(\displaystyle {0.4}\)}%
\end{pgfscope}%
\begin{pgfscope}%
\definecolor{textcolor}{rgb}{0.000000,0.000000,0.000000}%
\pgfsetstrokecolor{textcolor}%
\pgfsetfillcolor{textcolor}%
\pgftext[x=5.906075in, y=1.031207in, left, base,rotate=90.000000]{\color{textcolor}\rmfamily\fontsize{10.000000}{12.000000}\selectfont Amatőr }%
\end{pgfscope}%
\begin{pgfscope}%
\definecolor{textcolor}{rgb}{0.000000,0.000000,0.000000}%
\pgfsetstrokecolor{textcolor}%
\pgfsetfillcolor{textcolor}%
\pgftext[x=6.058081in, y=0.992819in, left, base,rotate=90.000000]{\color{textcolor}\rmfamily\fontsize{10.000000}{12.000000}\selectfont videó (\%)}%
\end{pgfscope}%
\begin{pgfscope}%
\pgfpathrectangle{\pgfqpoint{0.596528in}{0.693354in}}{\pgfqpoint{4.882849in}{1.189210in}}%
\pgfusepath{clip}%
\pgfsetrectcap%
\pgfsetroundjoin%
\pgfsetlinewidth{1.505625pt}%
\definecolor{currentstroke}{rgb}{0.000000,0.000000,0.000000}%
\pgfsetstrokecolor{currentstroke}%
\pgfsetdash{}{0pt}%
\pgfpathmoveto{\pgfqpoint{0.818476in}{0.747409in}}%
\pgfpathlineto{\pgfqpoint{1.120445in}{0.747801in}}%
\pgfpathlineto{\pgfqpoint{1.422415in}{0.979150in}}%
\pgfpathlineto{\pgfqpoint{1.724385in}{1.072134in}}%
\pgfpathlineto{\pgfqpoint{2.026354in}{1.171115in}}%
\pgfpathlineto{\pgfqpoint{2.328324in}{1.197879in}}%
\pgfpathlineto{\pgfqpoint{2.630294in}{1.153665in}}%
\pgfpathlineto{\pgfqpoint{2.932263in}{1.129987in}}%
\pgfpathlineto{\pgfqpoint{3.234233in}{1.147212in}}%
\pgfpathlineto{\pgfqpoint{3.536202in}{1.172566in}}%
\pgfpathlineto{\pgfqpoint{3.838172in}{1.220377in}}%
\pgfpathlineto{\pgfqpoint{4.140142in}{1.322560in}}%
\pgfpathlineto{\pgfqpoint{4.442111in}{1.464480in}}%
\pgfpathlineto{\pgfqpoint{4.744081in}{1.672969in}}%
\pgfpathlineto{\pgfqpoint{5.046051in}{1.828509in}}%
\pgfusepath{stroke}%
\end{pgfscope}%
\begin{pgfscope}%
\pgfsetrectcap%
\pgfsetmiterjoin%
\pgfsetlinewidth{0.803000pt}%
\definecolor{currentstroke}{rgb}{0.000000,0.000000,0.000000}%
\pgfsetstrokecolor{currentstroke}%
\pgfsetdash{}{0pt}%
\pgfpathmoveto{\pgfqpoint{0.596528in}{0.693354in}}%
\pgfpathlineto{\pgfqpoint{0.596528in}{1.882564in}}%
\pgfusepath{stroke}%
\end{pgfscope}%
\begin{pgfscope}%
\pgfsetrectcap%
\pgfsetmiterjoin%
\pgfsetlinewidth{0.803000pt}%
\definecolor{currentstroke}{rgb}{0.000000,0.000000,0.000000}%
\pgfsetstrokecolor{currentstroke}%
\pgfsetdash{}{0pt}%
\pgfpathmoveto{\pgfqpoint{5.479377in}{0.693354in}}%
\pgfpathlineto{\pgfqpoint{5.479377in}{1.882564in}}%
\pgfusepath{stroke}%
\end{pgfscope}%
\begin{pgfscope}%
\pgfsetrectcap%
\pgfsetmiterjoin%
\pgfsetlinewidth{0.803000pt}%
\definecolor{currentstroke}{rgb}{0.000000,0.000000,0.000000}%
\pgfsetstrokecolor{currentstroke}%
\pgfsetdash{}{0pt}%
\pgfpathmoveto{\pgfqpoint{0.596528in}{0.693354in}}%
\pgfpathlineto{\pgfqpoint{5.479377in}{0.693354in}}%
\pgfusepath{stroke}%
\end{pgfscope}%
\begin{pgfscope}%
\pgfsetrectcap%
\pgfsetmiterjoin%
\pgfsetlinewidth{0.803000pt}%
\definecolor{currentstroke}{rgb}{0.000000,0.000000,0.000000}%
\pgfsetstrokecolor{currentstroke}%
\pgfsetdash{}{0pt}%
\pgfpathmoveto{\pgfqpoint{0.596528in}{1.882564in}}%
\pgfpathlineto{\pgfqpoint{5.479377in}{1.882564in}}%
\pgfusepath{stroke}%
\end{pgfscope}%
\end{pgfpicture}%
\makeatother%
\endgroup%

    \end{center}
\end{figure}

\subsection{A pornográfia és a szexuális erőszak kapcsolata}

Számos kutatás rávilágított, hogy a pornográf tartalmak növelhetik az szexuális agressziót és ebből fakadóan a szexuális zaklatásokra is nagy hatással lehetnek. Az elemzésben azt feltételeztem, hogy a hardcore pornográf videók erre különösen nagy erővel hatnak, mivel ezekben a tartalmakban kifejezetten erőszakos jelenetek szerepelnek.

A \ref{harrasment.hardcore}. ábrán az látható, hogy a munkahelyi szexuális zaklatás aránya habár a nőknél csökkent 5\%-ról 4\%-ra, de a férfiak esetén 2018-ban ugrásszerűen megnövekedett 1\%-ról 03\%-ra. Ezzel egyidőben a hardcore tartalmak nézettségének aránya a PornHub weboldal működésének kezdetén, valamint a végén volt nagyobb mértékű.

Az ábra alapján a két változó kapcsolata fordított, ami megfelel az előzetes várakozásoknak, valamint a szakirodalom álláspontjának is.

\begin{figure}[h]
    \caption[Hardcore videók és szexuális zaklatás]{\footnotesize{A szexuális zaklatás és a hardcore videók nézettségi arányának alakulása. Forrás: saját ábra}}
    \label{harrasment.hardcore}
    \begin{center}
        %% Creator: Matplotlib, PGF backend
%%
%% To include the figure in your LaTeX document, write
%%   \input{<filename>.pgf}
%%
%% Make sure the required packages are loaded in your preamble
%%   \usepackage{pgf}
%%
%% Figures using additional raster images can only be included by \input if
%% they are in the same directory as the main LaTeX file. For loading figures
%% from other directories you can use the `import` package
%%   \usepackage{import}
%%
%% and then include the figures with
%%   \import{<path to file>}{<filename>.pgf}
%%
%% Matplotlib used the following preamble
%%
\begingroup%
\makeatletter%
\begin{pgfpicture}%
\pgfpathrectangle{\pgfpointorigin}{\pgfqpoint{6.093652in}{2.128799in}}%
\pgfusepath{use as bounding box, clip}%
\begin{pgfscope}%
\pgfsetbuttcap%
\pgfsetmiterjoin%
\pgfsetlinewidth{0.000000pt}%
\definecolor{currentstroke}{rgb}{1.000000,1.000000,1.000000}%
\pgfsetstrokecolor{currentstroke}%
\pgfsetstrokeopacity{0.000000}%
\pgfsetdash{}{0pt}%
\pgfpathmoveto{\pgfqpoint{0.000000in}{0.000000in}}%
\pgfpathlineto{\pgfqpoint{6.093652in}{0.000000in}}%
\pgfpathlineto{\pgfqpoint{6.093652in}{2.128799in}}%
\pgfpathlineto{\pgfqpoint{0.000000in}{2.128799in}}%
\pgfpathclose%
\pgfusepath{}%
\end{pgfscope}%
\begin{pgfscope}%
\pgfsetbuttcap%
\pgfsetmiterjoin%
\definecolor{currentfill}{rgb}{1.000000,1.000000,1.000000}%
\pgfsetfillcolor{currentfill}%
\pgfsetlinewidth{0.000000pt}%
\definecolor{currentstroke}{rgb}{0.000000,0.000000,0.000000}%
\pgfsetstrokecolor{currentstroke}%
\pgfsetstrokeopacity{0.000000}%
\pgfsetdash{}{0pt}%
\pgfpathmoveto{\pgfqpoint{0.605401in}{0.693354in}}%
\pgfpathlineto{\pgfqpoint{5.488250in}{0.693354in}}%
\pgfpathlineto{\pgfqpoint{5.488250in}{1.882564in}}%
\pgfpathlineto{\pgfqpoint{0.605401in}{1.882564in}}%
\pgfpathclose%
\pgfusepath{fill}%
\end{pgfscope}%
\begin{pgfscope}%
\pgfpathrectangle{\pgfqpoint{0.605401in}{0.693354in}}{\pgfqpoint{4.882849in}{1.189210in}}%
\pgfusepath{clip}%
\pgfsetbuttcap%
\pgfsetmiterjoin%
\definecolor{currentfill}{rgb}{0.121569,0.466667,0.705882}%
\pgfsetfillcolor{currentfill}%
\pgfsetlinewidth{0.000000pt}%
\definecolor{currentstroke}{rgb}{0.000000,0.000000,0.000000}%
\pgfsetstrokecolor{currentstroke}%
\pgfsetstrokeopacity{0.000000}%
\pgfsetdash{}{0pt}%
\pgfpathmoveto{\pgfqpoint{1.556606in}{0.693354in}}%
\pgfpathlineto{\pgfqpoint{1.778553in}{0.693354in}}%
\pgfpathlineto{\pgfqpoint{1.778553in}{0.919870in}}%
\pgfpathlineto{\pgfqpoint{1.556606in}{0.919870in}}%
\pgfpathclose%
\pgfusepath{fill}%
\end{pgfscope}%
\begin{pgfscope}%
\pgfpathrectangle{\pgfqpoint{0.605401in}{0.693354in}}{\pgfqpoint{4.882849in}{1.189210in}}%
\pgfusepath{clip}%
\pgfsetbuttcap%
\pgfsetmiterjoin%
\definecolor{currentfill}{rgb}{0.121569,0.466667,0.705882}%
\pgfsetfillcolor{currentfill}%
\pgfsetlinewidth{0.000000pt}%
\definecolor{currentstroke}{rgb}{0.000000,0.000000,0.000000}%
\pgfsetstrokecolor{currentstroke}%
\pgfsetstrokeopacity{0.000000}%
\pgfsetdash{}{0pt}%
\pgfpathmoveto{\pgfqpoint{2.824878in}{0.693354in}}%
\pgfpathlineto{\pgfqpoint{3.046826in}{0.693354in}}%
\pgfpathlineto{\pgfqpoint{3.046826in}{0.919870in}}%
\pgfpathlineto{\pgfqpoint{2.824878in}{0.919870in}}%
\pgfpathclose%
\pgfusepath{fill}%
\end{pgfscope}%
\begin{pgfscope}%
\pgfpathrectangle{\pgfqpoint{0.605401in}{0.693354in}}{\pgfqpoint{4.882849in}{1.189210in}}%
\pgfusepath{clip}%
\pgfsetbuttcap%
\pgfsetmiterjoin%
\definecolor{currentfill}{rgb}{0.121569,0.466667,0.705882}%
\pgfsetfillcolor{currentfill}%
\pgfsetlinewidth{0.000000pt}%
\definecolor{currentstroke}{rgb}{0.000000,0.000000,0.000000}%
\pgfsetstrokecolor{currentstroke}%
\pgfsetstrokeopacity{0.000000}%
\pgfsetdash{}{0pt}%
\pgfpathmoveto{\pgfqpoint{4.093150in}{0.693354in}}%
\pgfpathlineto{\pgfqpoint{4.315098in}{0.693354in}}%
\pgfpathlineto{\pgfqpoint{4.315098in}{1.372902in}}%
\pgfpathlineto{\pgfqpoint{4.093150in}{1.372902in}}%
\pgfpathclose%
\pgfusepath{fill}%
\end{pgfscope}%
\begin{pgfscope}%
\pgfpathrectangle{\pgfqpoint{0.605401in}{0.693354in}}{\pgfqpoint{4.882849in}{1.189210in}}%
\pgfusepath{clip}%
\pgfsetbuttcap%
\pgfsetmiterjoin%
\definecolor{currentfill}{rgb}{1.000000,0.498039,0.054902}%
\pgfsetfillcolor{currentfill}%
\pgfsetlinewidth{0.000000pt}%
\definecolor{currentstroke}{rgb}{0.000000,0.000000,0.000000}%
\pgfsetstrokecolor{currentstroke}%
\pgfsetstrokeopacity{0.000000}%
\pgfsetdash{}{0pt}%
\pgfpathmoveto{\pgfqpoint{1.778553in}{0.693354in}}%
\pgfpathlineto{\pgfqpoint{2.000501in}{0.693354in}}%
\pgfpathlineto{\pgfqpoint{2.000501in}{1.825935in}}%
\pgfpathlineto{\pgfqpoint{1.778553in}{1.825935in}}%
\pgfpathclose%
\pgfusepath{fill}%
\end{pgfscope}%
\begin{pgfscope}%
\pgfpathrectangle{\pgfqpoint{0.605401in}{0.693354in}}{\pgfqpoint{4.882849in}{1.189210in}}%
\pgfusepath{clip}%
\pgfsetbuttcap%
\pgfsetmiterjoin%
\definecolor{currentfill}{rgb}{1.000000,0.498039,0.054902}%
\pgfsetfillcolor{currentfill}%
\pgfsetlinewidth{0.000000pt}%
\definecolor{currentstroke}{rgb}{0.000000,0.000000,0.000000}%
\pgfsetstrokecolor{currentstroke}%
\pgfsetstrokeopacity{0.000000}%
\pgfsetdash{}{0pt}%
\pgfpathmoveto{\pgfqpoint{3.046826in}{0.693354in}}%
\pgfpathlineto{\pgfqpoint{3.268773in}{0.693354in}}%
\pgfpathlineto{\pgfqpoint{3.268773in}{1.599418in}}%
\pgfpathlineto{\pgfqpoint{3.046826in}{1.599418in}}%
\pgfpathclose%
\pgfusepath{fill}%
\end{pgfscope}%
\begin{pgfscope}%
\pgfpathrectangle{\pgfqpoint{0.605401in}{0.693354in}}{\pgfqpoint{4.882849in}{1.189210in}}%
\pgfusepath{clip}%
\pgfsetbuttcap%
\pgfsetmiterjoin%
\definecolor{currentfill}{rgb}{1.000000,0.498039,0.054902}%
\pgfsetfillcolor{currentfill}%
\pgfsetlinewidth{0.000000pt}%
\definecolor{currentstroke}{rgb}{0.000000,0.000000,0.000000}%
\pgfsetstrokecolor{currentstroke}%
\pgfsetstrokeopacity{0.000000}%
\pgfsetdash{}{0pt}%
\pgfpathmoveto{\pgfqpoint{4.315098in}{0.693354in}}%
\pgfpathlineto{\pgfqpoint{4.537046in}{0.693354in}}%
\pgfpathlineto{\pgfqpoint{4.537046in}{1.599418in}}%
\pgfpathlineto{\pgfqpoint{4.315098in}{1.599418in}}%
\pgfpathclose%
\pgfusepath{fill}%
\end{pgfscope}%
\begin{pgfscope}%
\pgfsetbuttcap%
\pgfsetroundjoin%
\definecolor{currentfill}{rgb}{0.000000,0.000000,0.000000}%
\pgfsetfillcolor{currentfill}%
\pgfsetlinewidth{0.803000pt}%
\definecolor{currentstroke}{rgb}{0.000000,0.000000,0.000000}%
\pgfsetstrokecolor{currentstroke}%
\pgfsetdash{}{0pt}%
\pgfsys@defobject{currentmarker}{\pgfqpoint{0.000000in}{-0.048611in}}{\pgfqpoint{0.000000in}{0.000000in}}{%
\pgfpathmoveto{\pgfqpoint{0.000000in}{0.000000in}}%
\pgfpathlineto{\pgfqpoint{0.000000in}{-0.048611in}}%
\pgfusepath{stroke,fill}%
}%
\begin{pgfscope}%
\pgfsys@transformshift{1.778553in}{0.693354in}%
\pgfsys@useobject{currentmarker}{}%
\end{pgfscope}%
\end{pgfscope}%
\begin{pgfscope}%
\definecolor{textcolor}{rgb}{0.000000,0.000000,0.000000}%
\pgfsetstrokecolor{textcolor}%
\pgfsetfillcolor{textcolor}%
\pgftext[x=1.778553in,y=0.596131in,,top]{\color{textcolor}\rmfamily\fontsize{10.000000}{12.000000}\selectfont \(\displaystyle {2010}\)}%
\end{pgfscope}%
\begin{pgfscope}%
\pgfsetbuttcap%
\pgfsetroundjoin%
\definecolor{currentfill}{rgb}{0.000000,0.000000,0.000000}%
\pgfsetfillcolor{currentfill}%
\pgfsetlinewidth{0.803000pt}%
\definecolor{currentstroke}{rgb}{0.000000,0.000000,0.000000}%
\pgfsetstrokecolor{currentstroke}%
\pgfsetdash{}{0pt}%
\pgfsys@defobject{currentmarker}{\pgfqpoint{0.000000in}{-0.048611in}}{\pgfqpoint{0.000000in}{0.000000in}}{%
\pgfpathmoveto{\pgfqpoint{0.000000in}{0.000000in}}%
\pgfpathlineto{\pgfqpoint{0.000000in}{-0.048611in}}%
\pgfusepath{stroke,fill}%
}%
\begin{pgfscope}%
\pgfsys@transformshift{3.046826in}{0.693354in}%
\pgfsys@useobject{currentmarker}{}%
\end{pgfscope}%
\end{pgfscope}%
\begin{pgfscope}%
\definecolor{textcolor}{rgb}{0.000000,0.000000,0.000000}%
\pgfsetstrokecolor{textcolor}%
\pgfsetfillcolor{textcolor}%
\pgftext[x=3.046826in,y=0.596131in,,top]{\color{textcolor}\rmfamily\fontsize{10.000000}{12.000000}\selectfont \(\displaystyle {2014}\)}%
\end{pgfscope}%
\begin{pgfscope}%
\pgfsetbuttcap%
\pgfsetroundjoin%
\definecolor{currentfill}{rgb}{0.000000,0.000000,0.000000}%
\pgfsetfillcolor{currentfill}%
\pgfsetlinewidth{0.803000pt}%
\definecolor{currentstroke}{rgb}{0.000000,0.000000,0.000000}%
\pgfsetstrokecolor{currentstroke}%
\pgfsetdash{}{0pt}%
\pgfsys@defobject{currentmarker}{\pgfqpoint{0.000000in}{-0.048611in}}{\pgfqpoint{0.000000in}{0.000000in}}{%
\pgfpathmoveto{\pgfqpoint{0.000000in}{0.000000in}}%
\pgfpathlineto{\pgfqpoint{0.000000in}{-0.048611in}}%
\pgfusepath{stroke,fill}%
}%
\begin{pgfscope}%
\pgfsys@transformshift{4.315098in}{0.693354in}%
\pgfsys@useobject{currentmarker}{}%
\end{pgfscope}%
\end{pgfscope}%
\begin{pgfscope}%
\definecolor{textcolor}{rgb}{0.000000,0.000000,0.000000}%
\pgfsetstrokecolor{textcolor}%
\pgfsetfillcolor{textcolor}%
\pgftext[x=4.315098in,y=0.596131in,,top]{\color{textcolor}\rmfamily\fontsize{10.000000}{12.000000}\selectfont \(\displaystyle {2018}\)}%
\end{pgfscope}%
\begin{pgfscope}%
\pgfsetbuttcap%
\pgfsetroundjoin%
\definecolor{currentfill}{rgb}{0.000000,0.000000,0.000000}%
\pgfsetfillcolor{currentfill}%
\pgfsetlinewidth{0.803000pt}%
\definecolor{currentstroke}{rgb}{0.000000,0.000000,0.000000}%
\pgfsetstrokecolor{currentstroke}%
\pgfsetdash{}{0pt}%
\pgfsys@defobject{currentmarker}{\pgfqpoint{-0.048611in}{0.000000in}}{\pgfqpoint{-0.000000in}{0.000000in}}{%
\pgfpathmoveto{\pgfqpoint{-0.000000in}{0.000000in}}%
\pgfpathlineto{\pgfqpoint{-0.048611in}{0.000000in}}%
\pgfusepath{stroke,fill}%
}%
\begin{pgfscope}%
\pgfsys@transformshift{0.605401in}{0.693354in}%
\pgfsys@useobject{currentmarker}{}%
\end{pgfscope}%
\end{pgfscope}%
\begin{pgfscope}%
\definecolor{textcolor}{rgb}{0.000000,0.000000,0.000000}%
\pgfsetstrokecolor{textcolor}%
\pgfsetfillcolor{textcolor}%
\pgftext[x=0.438734in, y=0.645128in, left, base]{\color{textcolor}\rmfamily\fontsize{10.000000}{12.000000}\selectfont \(\displaystyle {0}\)}%
\end{pgfscope}%
\begin{pgfscope}%
\pgfsetbuttcap%
\pgfsetroundjoin%
\definecolor{currentfill}{rgb}{0.000000,0.000000,0.000000}%
\pgfsetfillcolor{currentfill}%
\pgfsetlinewidth{0.803000pt}%
\definecolor{currentstroke}{rgb}{0.000000,0.000000,0.000000}%
\pgfsetstrokecolor{currentstroke}%
\pgfsetdash{}{0pt}%
\pgfsys@defobject{currentmarker}{\pgfqpoint{-0.048611in}{0.000000in}}{\pgfqpoint{-0.000000in}{0.000000in}}{%
\pgfpathmoveto{\pgfqpoint{-0.000000in}{0.000000in}}%
\pgfpathlineto{\pgfqpoint{-0.048611in}{0.000000in}}%
\pgfusepath{stroke,fill}%
}%
\begin{pgfscope}%
\pgfsys@transformshift{0.605401in}{1.146386in}%
\pgfsys@useobject{currentmarker}{}%
\end{pgfscope}%
\end{pgfscope}%
\begin{pgfscope}%
\definecolor{textcolor}{rgb}{0.000000,0.000000,0.000000}%
\pgfsetstrokecolor{textcolor}%
\pgfsetfillcolor{textcolor}%
\pgftext[x=0.438734in, y=1.098161in, left, base]{\color{textcolor}\rmfamily\fontsize{10.000000}{12.000000}\selectfont \(\displaystyle {2}\)}%
\end{pgfscope}%
\begin{pgfscope}%
\pgfsetbuttcap%
\pgfsetroundjoin%
\definecolor{currentfill}{rgb}{0.000000,0.000000,0.000000}%
\pgfsetfillcolor{currentfill}%
\pgfsetlinewidth{0.803000pt}%
\definecolor{currentstroke}{rgb}{0.000000,0.000000,0.000000}%
\pgfsetstrokecolor{currentstroke}%
\pgfsetdash{}{0pt}%
\pgfsys@defobject{currentmarker}{\pgfqpoint{-0.048611in}{0.000000in}}{\pgfqpoint{-0.000000in}{0.000000in}}{%
\pgfpathmoveto{\pgfqpoint{-0.000000in}{0.000000in}}%
\pgfpathlineto{\pgfqpoint{-0.048611in}{0.000000in}}%
\pgfusepath{stroke,fill}%
}%
\begin{pgfscope}%
\pgfsys@transformshift{0.605401in}{1.599418in}%
\pgfsys@useobject{currentmarker}{}%
\end{pgfscope}%
\end{pgfscope}%
\begin{pgfscope}%
\definecolor{textcolor}{rgb}{0.000000,0.000000,0.000000}%
\pgfsetstrokecolor{textcolor}%
\pgfsetfillcolor{textcolor}%
\pgftext[x=0.438734in, y=1.551193in, left, base]{\color{textcolor}\rmfamily\fontsize{10.000000}{12.000000}\selectfont \(\displaystyle {4}\)}%
\end{pgfscope}%
\begin{pgfscope}%
\definecolor{textcolor}{rgb}{0.000000,0.000000,0.000000}%
\pgfsetstrokecolor{textcolor}%
\pgfsetfillcolor{textcolor}%
\pgftext[x=0.196451in, y=0.911800in, left, base,rotate=90.000000]{\color{textcolor}\rmfamily\fontsize{10.000000}{12.000000}\selectfont Munkahelyi }%
\end{pgfscope}%
\begin{pgfscope}%
\definecolor{textcolor}{rgb}{0.000000,0.000000,0.000000}%
\pgfsetstrokecolor{textcolor}%
\pgfsetfillcolor{textcolor}%
\pgftext[x=0.348457in, y=0.907557in, left, base,rotate=90.000000]{\color{textcolor}\rmfamily\fontsize{10.000000}{12.000000}\selectfont zaklatás (\%)}%
\end{pgfscope}%
\begin{pgfscope}%
\definecolor{textcolor}{rgb}{0.000000,0.000000,0.000000}%
\pgfsetstrokecolor{textcolor}%
\pgfsetfillcolor{textcolor}%
\pgftext[x=0.605401in,y=1.924230in,left,base]{\color{textcolor}\rmfamily\fontsize{10.000000}{12.000000}\selectfont \(\displaystyle \times{10^{\ensuremath{-}2}}{}\)}%
\end{pgfscope}%
\begin{pgfscope}%
\pgfsetrectcap%
\pgfsetmiterjoin%
\pgfsetlinewidth{0.803000pt}%
\definecolor{currentstroke}{rgb}{0.000000,0.000000,0.000000}%
\pgfsetstrokecolor{currentstroke}%
\pgfsetdash{}{0pt}%
\pgfpathmoveto{\pgfqpoint{0.605401in}{0.693354in}}%
\pgfpathlineto{\pgfqpoint{0.605401in}{1.882564in}}%
\pgfusepath{stroke}%
\end{pgfscope}%
\begin{pgfscope}%
\pgfsetrectcap%
\pgfsetmiterjoin%
\pgfsetlinewidth{0.803000pt}%
\definecolor{currentstroke}{rgb}{0.000000,0.000000,0.000000}%
\pgfsetstrokecolor{currentstroke}%
\pgfsetdash{}{0pt}%
\pgfpathmoveto{\pgfqpoint{5.488250in}{0.693354in}}%
\pgfpathlineto{\pgfqpoint{5.488250in}{1.882564in}}%
\pgfusepath{stroke}%
\end{pgfscope}%
\begin{pgfscope}%
\pgfsetrectcap%
\pgfsetmiterjoin%
\pgfsetlinewidth{0.803000pt}%
\definecolor{currentstroke}{rgb}{0.000000,0.000000,0.000000}%
\pgfsetstrokecolor{currentstroke}%
\pgfsetdash{}{0pt}%
\pgfpathmoveto{\pgfqpoint{0.605401in}{0.693354in}}%
\pgfpathlineto{\pgfqpoint{5.488250in}{0.693354in}}%
\pgfusepath{stroke}%
\end{pgfscope}%
\begin{pgfscope}%
\pgfsetrectcap%
\pgfsetmiterjoin%
\pgfsetlinewidth{0.803000pt}%
\definecolor{currentstroke}{rgb}{0.000000,0.000000,0.000000}%
\pgfsetstrokecolor{currentstroke}%
\pgfsetdash{}{0pt}%
\pgfpathmoveto{\pgfqpoint{0.605401in}{1.882564in}}%
\pgfpathlineto{\pgfqpoint{5.488250in}{1.882564in}}%
\pgfusepath{stroke}%
\end{pgfscope}%
\begin{pgfscope}%
\definecolor{textcolor}{rgb}{1.000000,1.000000,1.000000}%
\pgfsetstrokecolor{textcolor}%
\pgfsetfillcolor{textcolor}%
\pgftext[x=1.695357in, y=0.701672in, left, base,rotate=90.000000]{\color{textcolor}\rmfamily\fontsize{8.000000}{9.600000}\selectfont 0.01}%
\end{pgfscope}%
\begin{pgfscope}%
\definecolor{textcolor}{rgb}{1.000000,1.000000,1.000000}%
\pgfsetstrokecolor{textcolor}%
\pgfsetfillcolor{textcolor}%
\pgftext[x=2.963630in, y=0.701672in, left, base,rotate=90.000000]{\color{textcolor}\rmfamily\fontsize{8.000000}{9.600000}\selectfont 0.01}%
\end{pgfscope}%
\begin{pgfscope}%
\definecolor{textcolor}{rgb}{1.000000,1.000000,1.000000}%
\pgfsetstrokecolor{textcolor}%
\pgfsetfillcolor{textcolor}%
\pgftext[x=4.231902in, y=0.928188in, left, base,rotate=90.000000]{\color{textcolor}\rmfamily\fontsize{8.000000}{9.600000}\selectfont 0.03}%
\end{pgfscope}%
\begin{pgfscope}%
\definecolor{textcolor}{rgb}{1.000000,1.000000,1.000000}%
\pgfsetstrokecolor{textcolor}%
\pgfsetfillcolor{textcolor}%
\pgftext[x=1.917305in, y=1.154704in, left, base,rotate=90.000000]{\color{textcolor}\rmfamily\fontsize{8.000000}{9.600000}\selectfont 0.05}%
\end{pgfscope}%
\begin{pgfscope}%
\definecolor{textcolor}{rgb}{1.000000,1.000000,1.000000}%
\pgfsetstrokecolor{textcolor}%
\pgfsetfillcolor{textcolor}%
\pgftext[x=3.185577in, y=1.041446in, left, base,rotate=90.000000]{\color{textcolor}\rmfamily\fontsize{8.000000}{9.600000}\selectfont 0.04}%
\end{pgfscope}%
\begin{pgfscope}%
\definecolor{textcolor}{rgb}{1.000000,1.000000,1.000000}%
\pgfsetstrokecolor{textcolor}%
\pgfsetfillcolor{textcolor}%
\pgftext[x=4.453850in, y=1.041446in, left, base,rotate=90.000000]{\color{textcolor}\rmfamily\fontsize{8.000000}{9.600000}\selectfont 0.04}%
\end{pgfscope}%
\begin{pgfscope}%
\pgfsetbuttcap%
\pgfsetmiterjoin%
\pgfsetlinewidth{0.000000pt}%
\definecolor{currentstroke}{rgb}{0.800000,0.800000,0.800000}%
\pgfsetstrokecolor{currentstroke}%
\pgfsetstrokeopacity{0.000000}%
\pgfsetdash{}{0pt}%
\pgfpathmoveto{\pgfqpoint{2.262489in}{0.100000in}}%
\pgfpathlineto{\pgfqpoint{3.831163in}{0.100000in}}%
\pgfpathquadraticcurveto{\pgfqpoint{3.858940in}{0.100000in}}{\pgfqpoint{3.858940in}{0.127778in}}%
\pgfpathlineto{\pgfqpoint{3.858940in}{0.307562in}}%
\pgfpathquadraticcurveto{\pgfqpoint{3.858940in}{0.335339in}}{\pgfqpoint{3.831163in}{0.335339in}}%
\pgfpathlineto{\pgfqpoint{2.262489in}{0.335339in}}%
\pgfpathquadraticcurveto{\pgfqpoint{2.234711in}{0.335339in}}{\pgfqpoint{2.234711in}{0.307562in}}%
\pgfpathlineto{\pgfqpoint{2.234711in}{0.127778in}}%
\pgfpathquadraticcurveto{\pgfqpoint{2.234711in}{0.100000in}}{\pgfqpoint{2.262489in}{0.100000in}}%
\pgfpathclose%
\pgfusepath{}%
\end{pgfscope}%
\begin{pgfscope}%
\pgfsetbuttcap%
\pgfsetmiterjoin%
\definecolor{currentfill}{rgb}{0.121569,0.466667,0.705882}%
\pgfsetfillcolor{currentfill}%
\pgfsetlinewidth{0.000000pt}%
\definecolor{currentstroke}{rgb}{0.000000,0.000000,0.000000}%
\pgfsetstrokecolor{currentstroke}%
\pgfsetstrokeopacity{0.000000}%
\pgfsetdash{}{0pt}%
\pgfpathmoveto{\pgfqpoint{2.290267in}{0.182562in}}%
\pgfpathlineto{\pgfqpoint{2.568044in}{0.182562in}}%
\pgfpathlineto{\pgfqpoint{2.568044in}{0.279784in}}%
\pgfpathlineto{\pgfqpoint{2.290267in}{0.279784in}}%
\pgfpathclose%
\pgfusepath{fill}%
\end{pgfscope}%
\begin{pgfscope}%
\definecolor{textcolor}{rgb}{0.000000,0.000000,0.000000}%
\pgfsetstrokecolor{textcolor}%
\pgfsetfillcolor{textcolor}%
\pgftext[x=2.679155in,y=0.182562in,left,base]{\color{textcolor}\rmfamily\fontsize{10.000000}{12.000000}\selectfont Férfi}%
\end{pgfscope}%
\begin{pgfscope}%
\pgfsetbuttcap%
\pgfsetmiterjoin%
\definecolor{currentfill}{rgb}{1.000000,0.498039,0.054902}%
\pgfsetfillcolor{currentfill}%
\pgfsetlinewidth{0.000000pt}%
\definecolor{currentstroke}{rgb}{0.000000,0.000000,0.000000}%
\pgfsetstrokecolor{currentstroke}%
\pgfsetstrokeopacity{0.000000}%
\pgfsetdash{}{0pt}%
\pgfpathmoveto{\pgfqpoint{3.240885in}{0.182562in}}%
\pgfpathlineto{\pgfqpoint{3.518662in}{0.182562in}}%
\pgfpathlineto{\pgfqpoint{3.518662in}{0.279784in}}%
\pgfpathlineto{\pgfqpoint{3.240885in}{0.279784in}}%
\pgfpathclose%
\pgfusepath{fill}%
\end{pgfscope}%
\begin{pgfscope}%
\definecolor{textcolor}{rgb}{0.000000,0.000000,0.000000}%
\pgfsetstrokecolor{textcolor}%
\pgfsetfillcolor{textcolor}%
\pgftext[x=3.629773in,y=0.182562in,left,base]{\color{textcolor}\rmfamily\fontsize{10.000000}{12.000000}\selectfont Nő}%
\end{pgfscope}%
\begin{pgfscope}%
\pgfsetbuttcap%
\pgfsetroundjoin%
\definecolor{currentfill}{rgb}{0.000000,0.000000,0.000000}%
\pgfsetfillcolor{currentfill}%
\pgfsetlinewidth{0.803000pt}%
\definecolor{currentstroke}{rgb}{0.000000,0.000000,0.000000}%
\pgfsetstrokecolor{currentstroke}%
\pgfsetdash{}{0pt}%
\pgfsys@defobject{currentmarker}{\pgfqpoint{0.000000in}{0.000000in}}{\pgfqpoint{0.048611in}{0.000000in}}{%
\pgfpathmoveto{\pgfqpoint{0.000000in}{0.000000in}}%
\pgfpathlineto{\pgfqpoint{0.048611in}{0.000000in}}%
\pgfusepath{stroke,fill}%
}%
\begin{pgfscope}%
\pgfsys@transformshift{5.488250in}{0.722521in}%
\pgfsys@useobject{currentmarker}{}%
\end{pgfscope}%
\end{pgfscope}%
\begin{pgfscope}%
\definecolor{textcolor}{rgb}{0.000000,0.000000,0.000000}%
\pgfsetstrokecolor{textcolor}%
\pgfsetfillcolor{textcolor}%
\pgftext[x=5.585472in, y=0.674295in, left, base]{\color{textcolor}\rmfamily\fontsize{10.000000}{12.000000}\selectfont \(\displaystyle {0}\)}%
\end{pgfscope}%
\begin{pgfscope}%
\pgfsetbuttcap%
\pgfsetroundjoin%
\definecolor{currentfill}{rgb}{0.000000,0.000000,0.000000}%
\pgfsetfillcolor{currentfill}%
\pgfsetlinewidth{0.803000pt}%
\definecolor{currentstroke}{rgb}{0.000000,0.000000,0.000000}%
\pgfsetstrokecolor{currentstroke}%
\pgfsetdash{}{0pt}%
\pgfsys@defobject{currentmarker}{\pgfqpoint{0.000000in}{0.000000in}}{\pgfqpoint{0.048611in}{0.000000in}}{%
\pgfpathmoveto{\pgfqpoint{0.000000in}{0.000000in}}%
\pgfpathlineto{\pgfqpoint{0.048611in}{0.000000in}}%
\pgfusepath{stroke,fill}%
}%
\begin{pgfscope}%
\pgfsys@transformshift{5.488250in}{1.112692in}%
\pgfsys@useobject{currentmarker}{}%
\end{pgfscope}%
\end{pgfscope}%
\begin{pgfscope}%
\definecolor{textcolor}{rgb}{0.000000,0.000000,0.000000}%
\pgfsetstrokecolor{textcolor}%
\pgfsetfillcolor{textcolor}%
\pgftext[x=5.585472in, y=1.064467in, left, base]{\color{textcolor}\rmfamily\fontsize{10.000000}{12.000000}\selectfont \(\displaystyle {1}\)}%
\end{pgfscope}%
\begin{pgfscope}%
\pgfsetbuttcap%
\pgfsetroundjoin%
\definecolor{currentfill}{rgb}{0.000000,0.000000,0.000000}%
\pgfsetfillcolor{currentfill}%
\pgfsetlinewidth{0.803000pt}%
\definecolor{currentstroke}{rgb}{0.000000,0.000000,0.000000}%
\pgfsetstrokecolor{currentstroke}%
\pgfsetdash{}{0pt}%
\pgfsys@defobject{currentmarker}{\pgfqpoint{0.000000in}{0.000000in}}{\pgfqpoint{0.048611in}{0.000000in}}{%
\pgfpathmoveto{\pgfqpoint{0.000000in}{0.000000in}}%
\pgfpathlineto{\pgfqpoint{0.048611in}{0.000000in}}%
\pgfusepath{stroke,fill}%
}%
\begin{pgfscope}%
\pgfsys@transformshift{5.488250in}{1.502863in}%
\pgfsys@useobject{currentmarker}{}%
\end{pgfscope}%
\end{pgfscope}%
\begin{pgfscope}%
\definecolor{textcolor}{rgb}{0.000000,0.000000,0.000000}%
\pgfsetstrokecolor{textcolor}%
\pgfsetfillcolor{textcolor}%
\pgftext[x=5.585472in, y=1.454638in, left, base]{\color{textcolor}\rmfamily\fontsize{10.000000}{12.000000}\selectfont \(\displaystyle {2}\)}%
\end{pgfscope}%
\begin{pgfscope}%
\definecolor{textcolor}{rgb}{0.000000,0.000000,0.000000}%
\pgfsetstrokecolor{textcolor}%
\pgfsetfillcolor{textcolor}%
\pgftext[x=5.806923in, y=1.011724in, left, base,rotate=90.000000]{\color{textcolor}\rmfamily\fontsize{10.000000}{12.000000}\selectfont Hardcore}%
\end{pgfscope}%
\begin{pgfscope}%
\definecolor{textcolor}{rgb}{0.000000,0.000000,0.000000}%
\pgfsetstrokecolor{textcolor}%
\pgfsetfillcolor{textcolor}%
\pgftext[x=5.958929in, y=0.992819in, left, base,rotate=90.000000]{\color{textcolor}\rmfamily\fontsize{10.000000}{12.000000}\selectfont videó (\%)}%
\end{pgfscope}%
\begin{pgfscope}%
\definecolor{textcolor}{rgb}{0.000000,0.000000,0.000000}%
\pgfsetstrokecolor{textcolor}%
\pgfsetfillcolor{textcolor}%
\pgftext[x=5.488250in,y=1.924230in,right,base]{\color{textcolor}\rmfamily\fontsize{10.000000}{12.000000}\selectfont \(\displaystyle \times{10^{\ensuremath{-}1}}{}\)}%
\end{pgfscope}%
\begin{pgfscope}%
\pgfpathrectangle{\pgfqpoint{0.605401in}{0.693354in}}{\pgfqpoint{4.882849in}{1.189210in}}%
\pgfusepath{clip}%
\pgfsetrectcap%
\pgfsetroundjoin%
\pgfsetlinewidth{1.505625pt}%
\definecolor{currentstroke}{rgb}{0.000000,0.000000,0.000000}%
\pgfsetstrokecolor{currentstroke}%
\pgfsetdash{}{0pt}%
\pgfpathmoveto{\pgfqpoint{0.827349in}{0.747409in}}%
\pgfpathlineto{\pgfqpoint{1.144417in}{1.524285in}}%
\pgfpathlineto{\pgfqpoint{1.461485in}{1.274575in}}%
\pgfpathlineto{\pgfqpoint{1.778553in}{1.209282in}}%
\pgfpathlineto{\pgfqpoint{2.095621in}{1.337287in}}%
\pgfpathlineto{\pgfqpoint{2.412690in}{1.270769in}}%
\pgfpathlineto{\pgfqpoint{2.729758in}{1.265700in}}%
\pgfpathlineto{\pgfqpoint{3.046826in}{1.245284in}}%
\pgfpathlineto{\pgfqpoint{3.363894in}{1.282217in}}%
\pgfpathlineto{\pgfqpoint{3.680962in}{1.359826in}}%
\pgfpathlineto{\pgfqpoint{3.998030in}{1.486491in}}%
\pgfpathlineto{\pgfqpoint{4.315098in}{1.726997in}}%
\pgfpathlineto{\pgfqpoint{4.632166in}{1.828509in}}%
\pgfpathlineto{\pgfqpoint{4.949234in}{1.786493in}}%
\pgfpathlineto{\pgfqpoint{5.266302in}{1.729371in}}%
\pgfusepath{stroke}%
\end{pgfscope}%
\begin{pgfscope}%
\pgfsetrectcap%
\pgfsetmiterjoin%
\pgfsetlinewidth{0.803000pt}%
\definecolor{currentstroke}{rgb}{0.000000,0.000000,0.000000}%
\pgfsetstrokecolor{currentstroke}%
\pgfsetdash{}{0pt}%
\pgfpathmoveto{\pgfqpoint{0.605401in}{0.693354in}}%
\pgfpathlineto{\pgfqpoint{0.605401in}{1.882564in}}%
\pgfusepath{stroke}%
\end{pgfscope}%
\begin{pgfscope}%
\pgfsetrectcap%
\pgfsetmiterjoin%
\pgfsetlinewidth{0.803000pt}%
\definecolor{currentstroke}{rgb}{0.000000,0.000000,0.000000}%
\pgfsetstrokecolor{currentstroke}%
\pgfsetdash{}{0pt}%
\pgfpathmoveto{\pgfqpoint{5.488250in}{0.693354in}}%
\pgfpathlineto{\pgfqpoint{5.488250in}{1.882564in}}%
\pgfusepath{stroke}%
\end{pgfscope}%
\begin{pgfscope}%
\pgfsetrectcap%
\pgfsetmiterjoin%
\pgfsetlinewidth{0.803000pt}%
\definecolor{currentstroke}{rgb}{0.000000,0.000000,0.000000}%
\pgfsetstrokecolor{currentstroke}%
\pgfsetdash{}{0pt}%
\pgfpathmoveto{\pgfqpoint{0.605401in}{0.693354in}}%
\pgfpathlineto{\pgfqpoint{5.488250in}{0.693354in}}%
\pgfusepath{stroke}%
\end{pgfscope}%
\begin{pgfscope}%
\pgfsetrectcap%
\pgfsetmiterjoin%
\pgfsetlinewidth{0.803000pt}%
\definecolor{currentstroke}{rgb}{0.000000,0.000000,0.000000}%
\pgfsetstrokecolor{currentstroke}%
\pgfsetdash{}{0pt}%
\pgfpathmoveto{\pgfqpoint{0.605401in}{1.882564in}}%
\pgfpathlineto{\pgfqpoint{5.488250in}{1.882564in}}%
\pgfusepath{stroke}%
\end{pgfscope}%
\end{pgfpicture}%
\makeatother%
\endgroup%

    \end{center}
\end{figure}

\subsection{A pornográfia és a szexuális orientáció kapcsolata}

A szexuális orientáció és pornográf tartalmak kapcsolata némiképp eltér az eddig vizsgált összefüggésektől. A szexuális orientáció nem egy külső hatásokból internalizált tulajdonság, hanem annál jóval korábban kialakulhat. Ennek megfelelően a pornográfia véleményem szerint nincs hatással a szexuális orientációra. Ugyanakkor a pornográfia, mint iparág függ a szexuális orientációtól, mivel az iparágnak célja a fogyasztók minél széleskörűbb kielégítése.

A \ref{homosexuality.homosexual}. ábrán azt lehet megfigyelni, hogy a homoszexualitás aránya a nőknél évről évre növekedett, a férfiak esetében pedig ez szintén igaz 2018 kivételével, ahol ez az arány erőteljesen lecsökkent. A homoszexuális tartalmak nézettségének aránya ezalatt 2012-ig folyamatosan növekedett körülbelül 12,5\%-ra, majd csökkenni kezdett és 2021-re csupán 7,5\%-ra esett vissza.

Az ábra alapján a két változó közötti kapcsolat nem egyértelmű. A homoszexuális tartalmak fogyasztása visszaesik, ami annak lehet is a következménye, hogy a homoszexuális fogyasztók más oldalakra látogatnak a hasonló tartalmak után keresve, de az is lehet, hogy csupán lecsökkent a pornográf tartalom iránti keresletük, kevesebb ilyen terméket akarnak fogyasztani. Az ábráról azonban elmondható, hogy a pornóiparban megjelenő kereslet nem követi a fogyasztók valós szexuális orientációját.

\begin{figure}[h]
    \caption[Homoszexuális videók és homoszexualitás]{\footnotesize{A homoszexualitás és a homoszexuális videók nézettségi arányának alakulása. Forrás: saját ábra}}
    \label{homosexuality.homosexual}
    \begin{center}
        %% Creator: Matplotlib, PGF backend
%%
%% To include the figure in your LaTeX document, write
%%   \input{<filename>.pgf}
%%
%% Make sure the required packages are loaded in your preamble
%%   \usepackage{pgf}
%%
%% Figures using additional raster images can only be included by \input if
%% they are in the same directory as the main LaTeX file. For loading figures
%% from other directories you can use the `import` package
%%   \usepackage{import}
%%
%% and then include the figures with
%%   \import{<path to file>}{<filename>.pgf}
%%
%% Matplotlib used the following preamble
%%
\begingroup%
\makeatletter%
\begin{pgfpicture}%
\pgfpathrectangle{\pgfpointorigin}{\pgfqpoint{6.234856in}{2.128799in}}%
\pgfusepath{use as bounding box, clip}%
\begin{pgfscope}%
\pgfsetbuttcap%
\pgfsetmiterjoin%
\pgfsetlinewidth{0.000000pt}%
\definecolor{currentstroke}{rgb}{1.000000,1.000000,1.000000}%
\pgfsetstrokecolor{currentstroke}%
\pgfsetstrokeopacity{0.000000}%
\pgfsetdash{}{0pt}%
\pgfpathmoveto{\pgfqpoint{0.000000in}{0.000000in}}%
\pgfpathlineto{\pgfqpoint{6.234856in}{0.000000in}}%
\pgfpathlineto{\pgfqpoint{6.234856in}{2.128799in}}%
\pgfpathlineto{\pgfqpoint{0.000000in}{2.128799in}}%
\pgfpathclose%
\pgfusepath{}%
\end{pgfscope}%
\begin{pgfscope}%
\pgfsetbuttcap%
\pgfsetmiterjoin%
\definecolor{currentfill}{rgb}{1.000000,1.000000,1.000000}%
\pgfsetfillcolor{currentfill}%
\pgfsetlinewidth{0.000000pt}%
\definecolor{currentstroke}{rgb}{0.000000,0.000000,0.000000}%
\pgfsetstrokecolor{currentstroke}%
\pgfsetstrokeopacity{0.000000}%
\pgfsetdash{}{0pt}%
\pgfpathmoveto{\pgfqpoint{0.569136in}{0.693354in}}%
\pgfpathlineto{\pgfqpoint{5.451985in}{0.693354in}}%
\pgfpathlineto{\pgfqpoint{5.451985in}{1.882564in}}%
\pgfpathlineto{\pgfqpoint{0.569136in}{1.882564in}}%
\pgfpathclose%
\pgfusepath{fill}%
\end{pgfscope}%
\begin{pgfscope}%
\pgfpathrectangle{\pgfqpoint{0.569136in}{0.693354in}}{\pgfqpoint{4.882849in}{1.189210in}}%
\pgfusepath{clip}%
\pgfsetbuttcap%
\pgfsetmiterjoin%
\definecolor{currentfill}{rgb}{0.121569,0.466667,0.705882}%
\pgfsetfillcolor{currentfill}%
\pgfsetlinewidth{0.000000pt}%
\definecolor{currentstroke}{rgb}{0.000000,0.000000,0.000000}%
\pgfsetstrokecolor{currentstroke}%
\pgfsetstrokeopacity{0.000000}%
\pgfsetdash{}{0pt}%
\pgfpathmoveto{\pgfqpoint{0.791084in}{0.693354in}}%
\pgfpathlineto{\pgfqpoint{1.006867in}{0.693354in}}%
\pgfpathlineto{\pgfqpoint{1.006867in}{1.118072in}}%
\pgfpathlineto{\pgfqpoint{0.791084in}{1.118072in}}%
\pgfpathclose%
\pgfusepath{fill}%
\end{pgfscope}%
\begin{pgfscope}%
\pgfpathrectangle{\pgfqpoint{0.569136in}{0.693354in}}{\pgfqpoint{4.882849in}{1.189210in}}%
\pgfusepath{clip}%
\pgfsetbuttcap%
\pgfsetmiterjoin%
\definecolor{currentfill}{rgb}{0.121569,0.466667,0.705882}%
\pgfsetfillcolor{currentfill}%
\pgfsetlinewidth{0.000000pt}%
\definecolor{currentstroke}{rgb}{0.000000,0.000000,0.000000}%
\pgfsetstrokecolor{currentstroke}%
\pgfsetstrokeopacity{0.000000}%
\pgfsetdash{}{0pt}%
\pgfpathmoveto{\pgfqpoint{1.407605in}{0.693354in}}%
\pgfpathlineto{\pgfqpoint{1.623388in}{0.693354in}}%
\pgfpathlineto{\pgfqpoint{1.623388in}{1.118072in}}%
\pgfpathlineto{\pgfqpoint{1.407605in}{1.118072in}}%
\pgfpathclose%
\pgfusepath{fill}%
\end{pgfscope}%
\begin{pgfscope}%
\pgfpathrectangle{\pgfqpoint{0.569136in}{0.693354in}}{\pgfqpoint{4.882849in}{1.189210in}}%
\pgfusepath{clip}%
\pgfsetbuttcap%
\pgfsetmiterjoin%
\definecolor{currentfill}{rgb}{0.121569,0.466667,0.705882}%
\pgfsetfillcolor{currentfill}%
\pgfsetlinewidth{0.000000pt}%
\definecolor{currentstroke}{rgb}{0.000000,0.000000,0.000000}%
\pgfsetstrokecolor{currentstroke}%
\pgfsetstrokeopacity{0.000000}%
\pgfsetdash{}{0pt}%
\pgfpathmoveto{\pgfqpoint{2.024127in}{0.693354in}}%
\pgfpathlineto{\pgfqpoint{2.239909in}{0.693354in}}%
\pgfpathlineto{\pgfqpoint{2.239909in}{1.259644in}}%
\pgfpathlineto{\pgfqpoint{2.024127in}{1.259644in}}%
\pgfpathclose%
\pgfusepath{fill}%
\end{pgfscope}%
\begin{pgfscope}%
\pgfpathrectangle{\pgfqpoint{0.569136in}{0.693354in}}{\pgfqpoint{4.882849in}{1.189210in}}%
\pgfusepath{clip}%
\pgfsetbuttcap%
\pgfsetmiterjoin%
\definecolor{currentfill}{rgb}{0.121569,0.466667,0.705882}%
\pgfsetfillcolor{currentfill}%
\pgfsetlinewidth{0.000000pt}%
\definecolor{currentstroke}{rgb}{0.000000,0.000000,0.000000}%
\pgfsetstrokecolor{currentstroke}%
\pgfsetstrokeopacity{0.000000}%
\pgfsetdash{}{0pt}%
\pgfpathmoveto{\pgfqpoint{2.640648in}{0.693354in}}%
\pgfpathlineto{\pgfqpoint{2.856430in}{0.693354in}}%
\pgfpathlineto{\pgfqpoint{2.856430in}{1.401217in}}%
\pgfpathlineto{\pgfqpoint{2.640648in}{1.401217in}}%
\pgfpathclose%
\pgfusepath{fill}%
\end{pgfscope}%
\begin{pgfscope}%
\pgfpathrectangle{\pgfqpoint{0.569136in}{0.693354in}}{\pgfqpoint{4.882849in}{1.189210in}}%
\pgfusepath{clip}%
\pgfsetbuttcap%
\pgfsetmiterjoin%
\definecolor{currentfill}{rgb}{0.121569,0.466667,0.705882}%
\pgfsetfillcolor{currentfill}%
\pgfsetlinewidth{0.000000pt}%
\definecolor{currentstroke}{rgb}{0.000000,0.000000,0.000000}%
\pgfsetstrokecolor{currentstroke}%
\pgfsetstrokeopacity{0.000000}%
\pgfsetdash{}{0pt}%
\pgfpathmoveto{\pgfqpoint{3.257169in}{0.693354in}}%
\pgfpathlineto{\pgfqpoint{3.472952in}{0.693354in}}%
\pgfpathlineto{\pgfqpoint{3.472952in}{1.401217in}}%
\pgfpathlineto{\pgfqpoint{3.257169in}{1.401217in}}%
\pgfpathclose%
\pgfusepath{fill}%
\end{pgfscope}%
\begin{pgfscope}%
\pgfpathrectangle{\pgfqpoint{0.569136in}{0.693354in}}{\pgfqpoint{4.882849in}{1.189210in}}%
\pgfusepath{clip}%
\pgfsetbuttcap%
\pgfsetmiterjoin%
\definecolor{currentfill}{rgb}{0.121569,0.466667,0.705882}%
\pgfsetfillcolor{currentfill}%
\pgfsetlinewidth{0.000000pt}%
\definecolor{currentstroke}{rgb}{0.000000,0.000000,0.000000}%
\pgfsetstrokecolor{currentstroke}%
\pgfsetstrokeopacity{0.000000}%
\pgfsetdash{}{0pt}%
\pgfpathmoveto{\pgfqpoint{3.873691in}{0.693354in}}%
\pgfpathlineto{\pgfqpoint{4.089473in}{0.693354in}}%
\pgfpathlineto{\pgfqpoint{4.089473in}{1.118072in}}%
\pgfpathlineto{\pgfqpoint{3.873691in}{1.118072in}}%
\pgfpathclose%
\pgfusepath{fill}%
\end{pgfscope}%
\begin{pgfscope}%
\pgfpathrectangle{\pgfqpoint{0.569136in}{0.693354in}}{\pgfqpoint{4.882849in}{1.189210in}}%
\pgfusepath{clip}%
\pgfsetbuttcap%
\pgfsetmiterjoin%
\definecolor{currentfill}{rgb}{0.121569,0.466667,0.705882}%
\pgfsetfillcolor{currentfill}%
\pgfsetlinewidth{0.000000pt}%
\definecolor{currentstroke}{rgb}{0.000000,0.000000,0.000000}%
\pgfsetstrokecolor{currentstroke}%
\pgfsetstrokeopacity{0.000000}%
\pgfsetdash{}{0pt}%
\pgfpathmoveto{\pgfqpoint{4.798473in}{0.693354in}}%
\pgfpathlineto{\pgfqpoint{5.014255in}{0.693354in}}%
\pgfpathlineto{\pgfqpoint{5.014255in}{1.684362in}}%
\pgfpathlineto{\pgfqpoint{4.798473in}{1.684362in}}%
\pgfpathclose%
\pgfusepath{fill}%
\end{pgfscope}%
\begin{pgfscope}%
\pgfpathrectangle{\pgfqpoint{0.569136in}{0.693354in}}{\pgfqpoint{4.882849in}{1.189210in}}%
\pgfusepath{clip}%
\pgfsetbuttcap%
\pgfsetmiterjoin%
\definecolor{currentfill}{rgb}{1.000000,0.498039,0.054902}%
\pgfsetfillcolor{currentfill}%
\pgfsetlinewidth{0.000000pt}%
\definecolor{currentstroke}{rgb}{0.000000,0.000000,0.000000}%
\pgfsetstrokecolor{currentstroke}%
\pgfsetstrokeopacity{0.000000}%
\pgfsetdash{}{0pt}%
\pgfpathmoveto{\pgfqpoint{1.006867in}{0.693354in}}%
\pgfpathlineto{\pgfqpoint{1.222649in}{0.693354in}}%
\pgfpathlineto{\pgfqpoint{1.222649in}{1.259644in}}%
\pgfpathlineto{\pgfqpoint{1.006867in}{1.259644in}}%
\pgfpathclose%
\pgfusepath{fill}%
\end{pgfscope}%
\begin{pgfscope}%
\pgfpathrectangle{\pgfqpoint{0.569136in}{0.693354in}}{\pgfqpoint{4.882849in}{1.189210in}}%
\pgfusepath{clip}%
\pgfsetbuttcap%
\pgfsetmiterjoin%
\definecolor{currentfill}{rgb}{1.000000,0.498039,0.054902}%
\pgfsetfillcolor{currentfill}%
\pgfsetlinewidth{0.000000pt}%
\definecolor{currentstroke}{rgb}{0.000000,0.000000,0.000000}%
\pgfsetstrokecolor{currentstroke}%
\pgfsetstrokeopacity{0.000000}%
\pgfsetdash{}{0pt}%
\pgfpathmoveto{\pgfqpoint{1.623388in}{0.693354in}}%
\pgfpathlineto{\pgfqpoint{1.839170in}{0.693354in}}%
\pgfpathlineto{\pgfqpoint{1.839170in}{1.259644in}}%
\pgfpathlineto{\pgfqpoint{1.623388in}{1.259644in}}%
\pgfpathclose%
\pgfusepath{fill}%
\end{pgfscope}%
\begin{pgfscope}%
\pgfpathrectangle{\pgfqpoint{0.569136in}{0.693354in}}{\pgfqpoint{4.882849in}{1.189210in}}%
\pgfusepath{clip}%
\pgfsetbuttcap%
\pgfsetmiterjoin%
\definecolor{currentfill}{rgb}{1.000000,0.498039,0.054902}%
\pgfsetfillcolor{currentfill}%
\pgfsetlinewidth{0.000000pt}%
\definecolor{currentstroke}{rgb}{0.000000,0.000000,0.000000}%
\pgfsetstrokecolor{currentstroke}%
\pgfsetstrokeopacity{0.000000}%
\pgfsetdash{}{0pt}%
\pgfpathmoveto{\pgfqpoint{2.239909in}{0.693354in}}%
\pgfpathlineto{\pgfqpoint{2.455692in}{0.693354in}}%
\pgfpathlineto{\pgfqpoint{2.455692in}{1.259644in}}%
\pgfpathlineto{\pgfqpoint{2.239909in}{1.259644in}}%
\pgfpathclose%
\pgfusepath{fill}%
\end{pgfscope}%
\begin{pgfscope}%
\pgfpathrectangle{\pgfqpoint{0.569136in}{0.693354in}}{\pgfqpoint{4.882849in}{1.189210in}}%
\pgfusepath{clip}%
\pgfsetbuttcap%
\pgfsetmiterjoin%
\definecolor{currentfill}{rgb}{1.000000,0.498039,0.054902}%
\pgfsetfillcolor{currentfill}%
\pgfsetlinewidth{0.000000pt}%
\definecolor{currentstroke}{rgb}{0.000000,0.000000,0.000000}%
\pgfsetstrokecolor{currentstroke}%
\pgfsetstrokeopacity{0.000000}%
\pgfsetdash{}{0pt}%
\pgfpathmoveto{\pgfqpoint{2.856430in}{0.693354in}}%
\pgfpathlineto{\pgfqpoint{3.072213in}{0.693354in}}%
\pgfpathlineto{\pgfqpoint{3.072213in}{1.401217in}}%
\pgfpathlineto{\pgfqpoint{2.856430in}{1.401217in}}%
\pgfpathclose%
\pgfusepath{fill}%
\end{pgfscope}%
\begin{pgfscope}%
\pgfpathrectangle{\pgfqpoint{0.569136in}{0.693354in}}{\pgfqpoint{4.882849in}{1.189210in}}%
\pgfusepath{clip}%
\pgfsetbuttcap%
\pgfsetmiterjoin%
\definecolor{currentfill}{rgb}{1.000000,0.498039,0.054902}%
\pgfsetfillcolor{currentfill}%
\pgfsetlinewidth{0.000000pt}%
\definecolor{currentstroke}{rgb}{0.000000,0.000000,0.000000}%
\pgfsetstrokecolor{currentstroke}%
\pgfsetstrokeopacity{0.000000}%
\pgfsetdash{}{0pt}%
\pgfpathmoveto{\pgfqpoint{3.472952in}{0.693354in}}%
\pgfpathlineto{\pgfqpoint{3.688734in}{0.693354in}}%
\pgfpathlineto{\pgfqpoint{3.688734in}{1.542789in}}%
\pgfpathlineto{\pgfqpoint{3.472952in}{1.542789in}}%
\pgfpathclose%
\pgfusepath{fill}%
\end{pgfscope}%
\begin{pgfscope}%
\pgfpathrectangle{\pgfqpoint{0.569136in}{0.693354in}}{\pgfqpoint{4.882849in}{1.189210in}}%
\pgfusepath{clip}%
\pgfsetbuttcap%
\pgfsetmiterjoin%
\definecolor{currentfill}{rgb}{1.000000,0.498039,0.054902}%
\pgfsetfillcolor{currentfill}%
\pgfsetlinewidth{0.000000pt}%
\definecolor{currentstroke}{rgb}{0.000000,0.000000,0.000000}%
\pgfsetstrokecolor{currentstroke}%
\pgfsetstrokeopacity{0.000000}%
\pgfsetdash{}{0pt}%
\pgfpathmoveto{\pgfqpoint{4.089473in}{0.693354in}}%
\pgfpathlineto{\pgfqpoint{4.305256in}{0.693354in}}%
\pgfpathlineto{\pgfqpoint{4.305256in}{1.825935in}}%
\pgfpathlineto{\pgfqpoint{4.089473in}{1.825935in}}%
\pgfpathclose%
\pgfusepath{fill}%
\end{pgfscope}%
\begin{pgfscope}%
\pgfpathrectangle{\pgfqpoint{0.569136in}{0.693354in}}{\pgfqpoint{4.882849in}{1.189210in}}%
\pgfusepath{clip}%
\pgfsetbuttcap%
\pgfsetmiterjoin%
\definecolor{currentfill}{rgb}{1.000000,0.498039,0.054902}%
\pgfsetfillcolor{currentfill}%
\pgfsetlinewidth{0.000000pt}%
\definecolor{currentstroke}{rgb}{0.000000,0.000000,0.000000}%
\pgfsetstrokecolor{currentstroke}%
\pgfsetstrokeopacity{0.000000}%
\pgfsetdash{}{0pt}%
\pgfpathmoveto{\pgfqpoint{5.014255in}{0.693354in}}%
\pgfpathlineto{\pgfqpoint{5.230037in}{0.693354in}}%
\pgfpathlineto{\pgfqpoint{5.230037in}{1.825935in}}%
\pgfpathlineto{\pgfqpoint{5.014255in}{1.825935in}}%
\pgfpathclose%
\pgfusepath{fill}%
\end{pgfscope}%
\begin{pgfscope}%
\pgfsetbuttcap%
\pgfsetroundjoin%
\definecolor{currentfill}{rgb}{0.000000,0.000000,0.000000}%
\pgfsetfillcolor{currentfill}%
\pgfsetlinewidth{0.803000pt}%
\definecolor{currentstroke}{rgb}{0.000000,0.000000,0.000000}%
\pgfsetstrokecolor{currentstroke}%
\pgfsetdash{}{0pt}%
\pgfsys@defobject{currentmarker}{\pgfqpoint{0.000000in}{-0.048611in}}{\pgfqpoint{0.000000in}{0.000000in}}{%
\pgfpathmoveto{\pgfqpoint{0.000000in}{0.000000in}}%
\pgfpathlineto{\pgfqpoint{0.000000in}{-0.048611in}}%
\pgfusepath{stroke,fill}%
}%
\begin{pgfscope}%
\pgfsys@transformshift{1.006867in}{0.693354in}%
\pgfsys@useobject{currentmarker}{}%
\end{pgfscope}%
\end{pgfscope}%
\begin{pgfscope}%
\definecolor{textcolor}{rgb}{0.000000,0.000000,0.000000}%
\pgfsetstrokecolor{textcolor}%
\pgfsetfillcolor{textcolor}%
\pgftext[x=1.006867in,y=0.596131in,,top]{\color{textcolor}\rmfamily\fontsize{10.000000}{12.000000}\selectfont \(\displaystyle {2008}\)}%
\end{pgfscope}%
\begin{pgfscope}%
\pgfsetbuttcap%
\pgfsetroundjoin%
\definecolor{currentfill}{rgb}{0.000000,0.000000,0.000000}%
\pgfsetfillcolor{currentfill}%
\pgfsetlinewidth{0.803000pt}%
\definecolor{currentstroke}{rgb}{0.000000,0.000000,0.000000}%
\pgfsetstrokecolor{currentstroke}%
\pgfsetdash{}{0pt}%
\pgfsys@defobject{currentmarker}{\pgfqpoint{0.000000in}{-0.048611in}}{\pgfqpoint{0.000000in}{0.000000in}}{%
\pgfpathmoveto{\pgfqpoint{0.000000in}{0.000000in}}%
\pgfpathlineto{\pgfqpoint{0.000000in}{-0.048611in}}%
\pgfusepath{stroke,fill}%
}%
\begin{pgfscope}%
\pgfsys@transformshift{1.623388in}{0.693354in}%
\pgfsys@useobject{currentmarker}{}%
\end{pgfscope}%
\end{pgfscope}%
\begin{pgfscope}%
\definecolor{textcolor}{rgb}{0.000000,0.000000,0.000000}%
\pgfsetstrokecolor{textcolor}%
\pgfsetfillcolor{textcolor}%
\pgftext[x=1.623388in,y=0.596131in,,top]{\color{textcolor}\rmfamily\fontsize{10.000000}{12.000000}\selectfont \(\displaystyle {2010}\)}%
\end{pgfscope}%
\begin{pgfscope}%
\pgfsetbuttcap%
\pgfsetroundjoin%
\definecolor{currentfill}{rgb}{0.000000,0.000000,0.000000}%
\pgfsetfillcolor{currentfill}%
\pgfsetlinewidth{0.803000pt}%
\definecolor{currentstroke}{rgb}{0.000000,0.000000,0.000000}%
\pgfsetstrokecolor{currentstroke}%
\pgfsetdash{}{0pt}%
\pgfsys@defobject{currentmarker}{\pgfqpoint{0.000000in}{-0.048611in}}{\pgfqpoint{0.000000in}{0.000000in}}{%
\pgfpathmoveto{\pgfqpoint{0.000000in}{0.000000in}}%
\pgfpathlineto{\pgfqpoint{0.000000in}{-0.048611in}}%
\pgfusepath{stroke,fill}%
}%
\begin{pgfscope}%
\pgfsys@transformshift{2.239909in}{0.693354in}%
\pgfsys@useobject{currentmarker}{}%
\end{pgfscope}%
\end{pgfscope}%
\begin{pgfscope}%
\definecolor{textcolor}{rgb}{0.000000,0.000000,0.000000}%
\pgfsetstrokecolor{textcolor}%
\pgfsetfillcolor{textcolor}%
\pgftext[x=2.239909in,y=0.596131in,,top]{\color{textcolor}\rmfamily\fontsize{10.000000}{12.000000}\selectfont \(\displaystyle {2012}\)}%
\end{pgfscope}%
\begin{pgfscope}%
\pgfsetbuttcap%
\pgfsetroundjoin%
\definecolor{currentfill}{rgb}{0.000000,0.000000,0.000000}%
\pgfsetfillcolor{currentfill}%
\pgfsetlinewidth{0.803000pt}%
\definecolor{currentstroke}{rgb}{0.000000,0.000000,0.000000}%
\pgfsetstrokecolor{currentstroke}%
\pgfsetdash{}{0pt}%
\pgfsys@defobject{currentmarker}{\pgfqpoint{0.000000in}{-0.048611in}}{\pgfqpoint{0.000000in}{0.000000in}}{%
\pgfpathmoveto{\pgfqpoint{0.000000in}{0.000000in}}%
\pgfpathlineto{\pgfqpoint{0.000000in}{-0.048611in}}%
\pgfusepath{stroke,fill}%
}%
\begin{pgfscope}%
\pgfsys@transformshift{2.856430in}{0.693354in}%
\pgfsys@useobject{currentmarker}{}%
\end{pgfscope}%
\end{pgfscope}%
\begin{pgfscope}%
\definecolor{textcolor}{rgb}{0.000000,0.000000,0.000000}%
\pgfsetstrokecolor{textcolor}%
\pgfsetfillcolor{textcolor}%
\pgftext[x=2.856430in,y=0.596131in,,top]{\color{textcolor}\rmfamily\fontsize{10.000000}{12.000000}\selectfont \(\displaystyle {2014}\)}%
\end{pgfscope}%
\begin{pgfscope}%
\pgfsetbuttcap%
\pgfsetroundjoin%
\definecolor{currentfill}{rgb}{0.000000,0.000000,0.000000}%
\pgfsetfillcolor{currentfill}%
\pgfsetlinewidth{0.803000pt}%
\definecolor{currentstroke}{rgb}{0.000000,0.000000,0.000000}%
\pgfsetstrokecolor{currentstroke}%
\pgfsetdash{}{0pt}%
\pgfsys@defobject{currentmarker}{\pgfqpoint{0.000000in}{-0.048611in}}{\pgfqpoint{0.000000in}{0.000000in}}{%
\pgfpathmoveto{\pgfqpoint{0.000000in}{0.000000in}}%
\pgfpathlineto{\pgfqpoint{0.000000in}{-0.048611in}}%
\pgfusepath{stroke,fill}%
}%
\begin{pgfscope}%
\pgfsys@transformshift{3.472952in}{0.693354in}%
\pgfsys@useobject{currentmarker}{}%
\end{pgfscope}%
\end{pgfscope}%
\begin{pgfscope}%
\definecolor{textcolor}{rgb}{0.000000,0.000000,0.000000}%
\pgfsetstrokecolor{textcolor}%
\pgfsetfillcolor{textcolor}%
\pgftext[x=3.472952in,y=0.596131in,,top]{\color{textcolor}\rmfamily\fontsize{10.000000}{12.000000}\selectfont \(\displaystyle {2016}\)}%
\end{pgfscope}%
\begin{pgfscope}%
\pgfsetbuttcap%
\pgfsetroundjoin%
\definecolor{currentfill}{rgb}{0.000000,0.000000,0.000000}%
\pgfsetfillcolor{currentfill}%
\pgfsetlinewidth{0.803000pt}%
\definecolor{currentstroke}{rgb}{0.000000,0.000000,0.000000}%
\pgfsetstrokecolor{currentstroke}%
\pgfsetdash{}{0pt}%
\pgfsys@defobject{currentmarker}{\pgfqpoint{0.000000in}{-0.048611in}}{\pgfqpoint{0.000000in}{0.000000in}}{%
\pgfpathmoveto{\pgfqpoint{0.000000in}{0.000000in}}%
\pgfpathlineto{\pgfqpoint{0.000000in}{-0.048611in}}%
\pgfusepath{stroke,fill}%
}%
\begin{pgfscope}%
\pgfsys@transformshift{4.089473in}{0.693354in}%
\pgfsys@useobject{currentmarker}{}%
\end{pgfscope}%
\end{pgfscope}%
\begin{pgfscope}%
\definecolor{textcolor}{rgb}{0.000000,0.000000,0.000000}%
\pgfsetstrokecolor{textcolor}%
\pgfsetfillcolor{textcolor}%
\pgftext[x=4.089473in,y=0.596131in,,top]{\color{textcolor}\rmfamily\fontsize{10.000000}{12.000000}\selectfont \(\displaystyle {2018}\)}%
\end{pgfscope}%
\begin{pgfscope}%
\pgfsetbuttcap%
\pgfsetroundjoin%
\definecolor{currentfill}{rgb}{0.000000,0.000000,0.000000}%
\pgfsetfillcolor{currentfill}%
\pgfsetlinewidth{0.803000pt}%
\definecolor{currentstroke}{rgb}{0.000000,0.000000,0.000000}%
\pgfsetstrokecolor{currentstroke}%
\pgfsetdash{}{0pt}%
\pgfsys@defobject{currentmarker}{\pgfqpoint{0.000000in}{-0.048611in}}{\pgfqpoint{0.000000in}{0.000000in}}{%
\pgfpathmoveto{\pgfqpoint{0.000000in}{0.000000in}}%
\pgfpathlineto{\pgfqpoint{0.000000in}{-0.048611in}}%
\pgfusepath{stroke,fill}%
}%
\begin{pgfscope}%
\pgfsys@transformshift{5.014255in}{0.693354in}%
\pgfsys@useobject{currentmarker}{}%
\end{pgfscope}%
\end{pgfscope}%
\begin{pgfscope}%
\definecolor{textcolor}{rgb}{0.000000,0.000000,0.000000}%
\pgfsetstrokecolor{textcolor}%
\pgfsetfillcolor{textcolor}%
\pgftext[x=5.014255in,y=0.596131in,,top]{\color{textcolor}\rmfamily\fontsize{10.000000}{12.000000}\selectfont \(\displaystyle {2021}\)}%
\end{pgfscope}%
\begin{pgfscope}%
\pgfsetbuttcap%
\pgfsetroundjoin%
\definecolor{currentfill}{rgb}{0.000000,0.000000,0.000000}%
\pgfsetfillcolor{currentfill}%
\pgfsetlinewidth{0.803000pt}%
\definecolor{currentstroke}{rgb}{0.000000,0.000000,0.000000}%
\pgfsetstrokecolor{currentstroke}%
\pgfsetdash{}{0pt}%
\pgfsys@defobject{currentmarker}{\pgfqpoint{-0.048611in}{0.000000in}}{\pgfqpoint{-0.000000in}{0.000000in}}{%
\pgfpathmoveto{\pgfqpoint{-0.000000in}{0.000000in}}%
\pgfpathlineto{\pgfqpoint{-0.048611in}{0.000000in}}%
\pgfusepath{stroke,fill}%
}%
\begin{pgfscope}%
\pgfsys@transformshift{0.569136in}{0.693354in}%
\pgfsys@useobject{currentmarker}{}%
\end{pgfscope}%
\end{pgfscope}%
\begin{pgfscope}%
\definecolor{textcolor}{rgb}{0.000000,0.000000,0.000000}%
\pgfsetstrokecolor{textcolor}%
\pgfsetfillcolor{textcolor}%
\pgftext[x=0.294444in, y=0.645128in, left, base]{\color{textcolor}\rmfamily\fontsize{10.000000}{12.000000}\selectfont \(\displaystyle {0.0}\)}%
\end{pgfscope}%
\begin{pgfscope}%
\pgfsetbuttcap%
\pgfsetroundjoin%
\definecolor{currentfill}{rgb}{0.000000,0.000000,0.000000}%
\pgfsetfillcolor{currentfill}%
\pgfsetlinewidth{0.803000pt}%
\definecolor{currentstroke}{rgb}{0.000000,0.000000,0.000000}%
\pgfsetstrokecolor{currentstroke}%
\pgfsetdash{}{0pt}%
\pgfsys@defobject{currentmarker}{\pgfqpoint{-0.048611in}{0.000000in}}{\pgfqpoint{-0.000000in}{0.000000in}}{%
\pgfpathmoveto{\pgfqpoint{-0.000000in}{0.000000in}}%
\pgfpathlineto{\pgfqpoint{-0.048611in}{0.000000in}}%
\pgfusepath{stroke,fill}%
}%
\begin{pgfscope}%
\pgfsys@transformshift{0.569136in}{1.047285in}%
\pgfsys@useobject{currentmarker}{}%
\end{pgfscope}%
\end{pgfscope}%
\begin{pgfscope}%
\definecolor{textcolor}{rgb}{0.000000,0.000000,0.000000}%
\pgfsetstrokecolor{textcolor}%
\pgfsetfillcolor{textcolor}%
\pgftext[x=0.294444in, y=0.999060in, left, base]{\color{textcolor}\rmfamily\fontsize{10.000000}{12.000000}\selectfont \(\displaystyle {2.5}\)}%
\end{pgfscope}%
\begin{pgfscope}%
\pgfsetbuttcap%
\pgfsetroundjoin%
\definecolor{currentfill}{rgb}{0.000000,0.000000,0.000000}%
\pgfsetfillcolor{currentfill}%
\pgfsetlinewidth{0.803000pt}%
\definecolor{currentstroke}{rgb}{0.000000,0.000000,0.000000}%
\pgfsetstrokecolor{currentstroke}%
\pgfsetdash{}{0pt}%
\pgfsys@defobject{currentmarker}{\pgfqpoint{-0.048611in}{0.000000in}}{\pgfqpoint{-0.000000in}{0.000000in}}{%
\pgfpathmoveto{\pgfqpoint{-0.000000in}{0.000000in}}%
\pgfpathlineto{\pgfqpoint{-0.048611in}{0.000000in}}%
\pgfusepath{stroke,fill}%
}%
\begin{pgfscope}%
\pgfsys@transformshift{0.569136in}{1.401217in}%
\pgfsys@useobject{currentmarker}{}%
\end{pgfscope}%
\end{pgfscope}%
\begin{pgfscope}%
\definecolor{textcolor}{rgb}{0.000000,0.000000,0.000000}%
\pgfsetstrokecolor{textcolor}%
\pgfsetfillcolor{textcolor}%
\pgftext[x=0.294444in, y=1.352991in, left, base]{\color{textcolor}\rmfamily\fontsize{10.000000}{12.000000}\selectfont \(\displaystyle {5.0}\)}%
\end{pgfscope}%
\begin{pgfscope}%
\pgfsetbuttcap%
\pgfsetroundjoin%
\definecolor{currentfill}{rgb}{0.000000,0.000000,0.000000}%
\pgfsetfillcolor{currentfill}%
\pgfsetlinewidth{0.803000pt}%
\definecolor{currentstroke}{rgb}{0.000000,0.000000,0.000000}%
\pgfsetstrokecolor{currentstroke}%
\pgfsetdash{}{0pt}%
\pgfsys@defobject{currentmarker}{\pgfqpoint{-0.048611in}{0.000000in}}{\pgfqpoint{-0.000000in}{0.000000in}}{%
\pgfpathmoveto{\pgfqpoint{-0.000000in}{0.000000in}}%
\pgfpathlineto{\pgfqpoint{-0.048611in}{0.000000in}}%
\pgfusepath{stroke,fill}%
}%
\begin{pgfscope}%
\pgfsys@transformshift{0.569136in}{1.755148in}%
\pgfsys@useobject{currentmarker}{}%
\end{pgfscope}%
\end{pgfscope}%
\begin{pgfscope}%
\definecolor{textcolor}{rgb}{0.000000,0.000000,0.000000}%
\pgfsetstrokecolor{textcolor}%
\pgfsetfillcolor{textcolor}%
\pgftext[x=0.294444in, y=1.706923in, left, base]{\color{textcolor}\rmfamily\fontsize{10.000000}{12.000000}\selectfont \(\displaystyle {7.5}\)}%
\end{pgfscope}%
\begin{pgfscope}%
\definecolor{textcolor}{rgb}{0.000000,0.000000,0.000000}%
\pgfsetstrokecolor{textcolor}%
\pgfsetfillcolor{textcolor}%
\pgftext[x=0.238889in,y=1.287959in,,bottom,rotate=90.000000]{\color{textcolor}\rmfamily\fontsize{10.000000}{12.000000}\selectfont Homoszexualitás (\%)}%
\end{pgfscope}%
\begin{pgfscope}%
\definecolor{textcolor}{rgb}{0.000000,0.000000,0.000000}%
\pgfsetstrokecolor{textcolor}%
\pgfsetfillcolor{textcolor}%
\pgftext[x=0.569136in,y=1.924230in,left,base]{\color{textcolor}\rmfamily\fontsize{10.000000}{12.000000}\selectfont \(\displaystyle \times{10^{\ensuremath{-}2}}{}\)}%
\end{pgfscope}%
\begin{pgfscope}%
\pgfsetrectcap%
\pgfsetmiterjoin%
\pgfsetlinewidth{0.803000pt}%
\definecolor{currentstroke}{rgb}{0.000000,0.000000,0.000000}%
\pgfsetstrokecolor{currentstroke}%
\pgfsetdash{}{0pt}%
\pgfpathmoveto{\pgfqpoint{0.569136in}{0.693354in}}%
\pgfpathlineto{\pgfqpoint{0.569136in}{1.882564in}}%
\pgfusepath{stroke}%
\end{pgfscope}%
\begin{pgfscope}%
\pgfsetrectcap%
\pgfsetmiterjoin%
\pgfsetlinewidth{0.803000pt}%
\definecolor{currentstroke}{rgb}{0.000000,0.000000,0.000000}%
\pgfsetstrokecolor{currentstroke}%
\pgfsetdash{}{0pt}%
\pgfpathmoveto{\pgfqpoint{5.451985in}{0.693354in}}%
\pgfpathlineto{\pgfqpoint{5.451985in}{1.882564in}}%
\pgfusepath{stroke}%
\end{pgfscope}%
\begin{pgfscope}%
\pgfsetrectcap%
\pgfsetmiterjoin%
\pgfsetlinewidth{0.803000pt}%
\definecolor{currentstroke}{rgb}{0.000000,0.000000,0.000000}%
\pgfsetstrokecolor{currentstroke}%
\pgfsetdash{}{0pt}%
\pgfpathmoveto{\pgfqpoint{0.569136in}{0.693354in}}%
\pgfpathlineto{\pgfqpoint{5.451985in}{0.693354in}}%
\pgfusepath{stroke}%
\end{pgfscope}%
\begin{pgfscope}%
\pgfsetrectcap%
\pgfsetmiterjoin%
\pgfsetlinewidth{0.803000pt}%
\definecolor{currentstroke}{rgb}{0.000000,0.000000,0.000000}%
\pgfsetstrokecolor{currentstroke}%
\pgfsetdash{}{0pt}%
\pgfpathmoveto{\pgfqpoint{0.569136in}{1.882564in}}%
\pgfpathlineto{\pgfqpoint{5.451985in}{1.882564in}}%
\pgfusepath{stroke}%
\end{pgfscope}%
\begin{pgfscope}%
\definecolor{textcolor}{rgb}{1.000000,1.000000,1.000000}%
\pgfsetstrokecolor{textcolor}%
\pgfsetfillcolor{textcolor}%
\pgftext[x=0.926753in, y=0.800773in, left, base,rotate=90.000000]{\color{textcolor}\rmfamily\fontsize{8.000000}{9.600000}\selectfont 0.03}%
\end{pgfscope}%
\begin{pgfscope}%
\definecolor{textcolor}{rgb}{1.000000,1.000000,1.000000}%
\pgfsetstrokecolor{textcolor}%
\pgfsetfillcolor{textcolor}%
\pgftext[x=1.543274in, y=0.800773in, left, base,rotate=90.000000]{\color{textcolor}\rmfamily\fontsize{8.000000}{9.600000}\selectfont 0.03}%
\end{pgfscope}%
\begin{pgfscope}%
\definecolor{textcolor}{rgb}{1.000000,1.000000,1.000000}%
\pgfsetstrokecolor{textcolor}%
\pgfsetfillcolor{textcolor}%
\pgftext[x=2.159796in, y=0.871559in, left, base,rotate=90.000000]{\color{textcolor}\rmfamily\fontsize{8.000000}{9.600000}\selectfont 0.04}%
\end{pgfscope}%
\begin{pgfscope}%
\definecolor{textcolor}{rgb}{1.000000,1.000000,1.000000}%
\pgfsetstrokecolor{textcolor}%
\pgfsetfillcolor{textcolor}%
\pgftext[x=2.776317in, y=0.942345in, left, base,rotate=90.000000]{\color{textcolor}\rmfamily\fontsize{8.000000}{9.600000}\selectfont 0.05}%
\end{pgfscope}%
\begin{pgfscope}%
\definecolor{textcolor}{rgb}{1.000000,1.000000,1.000000}%
\pgfsetstrokecolor{textcolor}%
\pgfsetfillcolor{textcolor}%
\pgftext[x=3.392838in, y=0.942345in, left, base,rotate=90.000000]{\color{textcolor}\rmfamily\fontsize{8.000000}{9.600000}\selectfont 0.05}%
\end{pgfscope}%
\begin{pgfscope}%
\definecolor{textcolor}{rgb}{1.000000,1.000000,1.000000}%
\pgfsetstrokecolor{textcolor}%
\pgfsetfillcolor{textcolor}%
\pgftext[x=4.009360in, y=0.800773in, left, base,rotate=90.000000]{\color{textcolor}\rmfamily\fontsize{8.000000}{9.600000}\selectfont 0.03}%
\end{pgfscope}%
\begin{pgfscope}%
\definecolor{textcolor}{rgb}{1.000000,1.000000,1.000000}%
\pgfsetstrokecolor{textcolor}%
\pgfsetfillcolor{textcolor}%
\pgftext[x=4.934142in, y=1.083918in, left, base,rotate=90.000000]{\color{textcolor}\rmfamily\fontsize{8.000000}{9.600000}\selectfont 0.07}%
\end{pgfscope}%
\begin{pgfscope}%
\definecolor{textcolor}{rgb}{1.000000,1.000000,1.000000}%
\pgfsetstrokecolor{textcolor}%
\pgfsetfillcolor{textcolor}%
\pgftext[x=1.142536in, y=0.871559in, left, base,rotate=90.000000]{\color{textcolor}\rmfamily\fontsize{8.000000}{9.600000}\selectfont 0.04}%
\end{pgfscope}%
\begin{pgfscope}%
\definecolor{textcolor}{rgb}{1.000000,1.000000,1.000000}%
\pgfsetstrokecolor{textcolor}%
\pgfsetfillcolor{textcolor}%
\pgftext[x=1.759057in, y=0.871559in, left, base,rotate=90.000000]{\color{textcolor}\rmfamily\fontsize{8.000000}{9.600000}\selectfont 0.04}%
\end{pgfscope}%
\begin{pgfscope}%
\definecolor{textcolor}{rgb}{1.000000,1.000000,1.000000}%
\pgfsetstrokecolor{textcolor}%
\pgfsetfillcolor{textcolor}%
\pgftext[x=2.375578in, y=0.871559in, left, base,rotate=90.000000]{\color{textcolor}\rmfamily\fontsize{8.000000}{9.600000}\selectfont 0.04}%
\end{pgfscope}%
\begin{pgfscope}%
\definecolor{textcolor}{rgb}{1.000000,1.000000,1.000000}%
\pgfsetstrokecolor{textcolor}%
\pgfsetfillcolor{textcolor}%
\pgftext[x=2.992099in, y=0.942345in, left, base,rotate=90.000000]{\color{textcolor}\rmfamily\fontsize{8.000000}{9.600000}\selectfont 0.05}%
\end{pgfscope}%
\begin{pgfscope}%
\definecolor{textcolor}{rgb}{1.000000,1.000000,1.000000}%
\pgfsetstrokecolor{textcolor}%
\pgfsetfillcolor{textcolor}%
\pgftext[x=3.608621in, y=1.013132in, left, base,rotate=90.000000]{\color{textcolor}\rmfamily\fontsize{8.000000}{9.600000}\selectfont 0.06}%
\end{pgfscope}%
\begin{pgfscope}%
\definecolor{textcolor}{rgb}{1.000000,1.000000,1.000000}%
\pgfsetstrokecolor{textcolor}%
\pgfsetfillcolor{textcolor}%
\pgftext[x=4.225142in, y=1.154704in, left, base,rotate=90.000000]{\color{textcolor}\rmfamily\fontsize{8.000000}{9.600000}\selectfont 0.08}%
\end{pgfscope}%
\begin{pgfscope}%
\definecolor{textcolor}{rgb}{1.000000,1.000000,1.000000}%
\pgfsetstrokecolor{textcolor}%
\pgfsetfillcolor{textcolor}%
\pgftext[x=5.149924in, y=1.154704in, left, base,rotate=90.000000]{\color{textcolor}\rmfamily\fontsize{8.000000}{9.600000}\selectfont 0.08}%
\end{pgfscope}%
\begin{pgfscope}%
\pgfsetbuttcap%
\pgfsetmiterjoin%
\pgfsetlinewidth{0.000000pt}%
\definecolor{currentstroke}{rgb}{0.800000,0.800000,0.800000}%
\pgfsetstrokecolor{currentstroke}%
\pgfsetstrokeopacity{0.000000}%
\pgfsetdash{}{0pt}%
\pgfpathmoveto{\pgfqpoint{2.226224in}{0.100000in}}%
\pgfpathlineto{\pgfqpoint{3.794898in}{0.100000in}}%
\pgfpathquadraticcurveto{\pgfqpoint{3.822675in}{0.100000in}}{\pgfqpoint{3.822675in}{0.127778in}}%
\pgfpathlineto{\pgfqpoint{3.822675in}{0.307562in}}%
\pgfpathquadraticcurveto{\pgfqpoint{3.822675in}{0.335339in}}{\pgfqpoint{3.794898in}{0.335339in}}%
\pgfpathlineto{\pgfqpoint{2.226224in}{0.335339in}}%
\pgfpathquadraticcurveto{\pgfqpoint{2.198446in}{0.335339in}}{\pgfqpoint{2.198446in}{0.307562in}}%
\pgfpathlineto{\pgfqpoint{2.198446in}{0.127778in}}%
\pgfpathquadraticcurveto{\pgfqpoint{2.198446in}{0.100000in}}{\pgfqpoint{2.226224in}{0.100000in}}%
\pgfpathclose%
\pgfusepath{}%
\end{pgfscope}%
\begin{pgfscope}%
\pgfsetbuttcap%
\pgfsetmiterjoin%
\definecolor{currentfill}{rgb}{0.121569,0.466667,0.705882}%
\pgfsetfillcolor{currentfill}%
\pgfsetlinewidth{0.000000pt}%
\definecolor{currentstroke}{rgb}{0.000000,0.000000,0.000000}%
\pgfsetstrokecolor{currentstroke}%
\pgfsetstrokeopacity{0.000000}%
\pgfsetdash{}{0pt}%
\pgfpathmoveto{\pgfqpoint{2.254002in}{0.182562in}}%
\pgfpathlineto{\pgfqpoint{2.531779in}{0.182562in}}%
\pgfpathlineto{\pgfqpoint{2.531779in}{0.279784in}}%
\pgfpathlineto{\pgfqpoint{2.254002in}{0.279784in}}%
\pgfpathclose%
\pgfusepath{fill}%
\end{pgfscope}%
\begin{pgfscope}%
\definecolor{textcolor}{rgb}{0.000000,0.000000,0.000000}%
\pgfsetstrokecolor{textcolor}%
\pgfsetfillcolor{textcolor}%
\pgftext[x=2.642890in,y=0.182562in,left,base]{\color{textcolor}\rmfamily\fontsize{10.000000}{12.000000}\selectfont Férfi}%
\end{pgfscope}%
\begin{pgfscope}%
\pgfsetbuttcap%
\pgfsetmiterjoin%
\definecolor{currentfill}{rgb}{1.000000,0.498039,0.054902}%
\pgfsetfillcolor{currentfill}%
\pgfsetlinewidth{0.000000pt}%
\definecolor{currentstroke}{rgb}{0.000000,0.000000,0.000000}%
\pgfsetstrokecolor{currentstroke}%
\pgfsetstrokeopacity{0.000000}%
\pgfsetdash{}{0pt}%
\pgfpathmoveto{\pgfqpoint{3.204620in}{0.182562in}}%
\pgfpathlineto{\pgfqpoint{3.482397in}{0.182562in}}%
\pgfpathlineto{\pgfqpoint{3.482397in}{0.279784in}}%
\pgfpathlineto{\pgfqpoint{3.204620in}{0.279784in}}%
\pgfpathclose%
\pgfusepath{fill}%
\end{pgfscope}%
\begin{pgfscope}%
\definecolor{textcolor}{rgb}{0.000000,0.000000,0.000000}%
\pgfsetstrokecolor{textcolor}%
\pgfsetfillcolor{textcolor}%
\pgftext[x=3.593508in,y=0.182562in,left,base]{\color{textcolor}\rmfamily\fontsize{10.000000}{12.000000}\selectfont Nő}%
\end{pgfscope}%
\begin{pgfscope}%
\pgfsetbuttcap%
\pgfsetroundjoin%
\definecolor{currentfill}{rgb}{0.000000,0.000000,0.000000}%
\pgfsetfillcolor{currentfill}%
\pgfsetlinewidth{0.803000pt}%
\definecolor{currentstroke}{rgb}{0.000000,0.000000,0.000000}%
\pgfsetstrokecolor{currentstroke}%
\pgfsetdash{}{0pt}%
\pgfsys@defobject{currentmarker}{\pgfqpoint{0.000000in}{0.000000in}}{\pgfqpoint{0.048611in}{0.000000in}}{%
\pgfpathmoveto{\pgfqpoint{0.000000in}{0.000000in}}%
\pgfpathlineto{\pgfqpoint{0.048611in}{0.000000in}}%
\pgfusepath{stroke,fill}%
}%
\begin{pgfscope}%
\pgfsys@transformshift{5.451985in}{0.753556in}%
\pgfsys@useobject{currentmarker}{}%
\end{pgfscope}%
\end{pgfscope}%
\begin{pgfscope}%
\definecolor{textcolor}{rgb}{0.000000,0.000000,0.000000}%
\pgfsetstrokecolor{textcolor}%
\pgfsetfillcolor{textcolor}%
\pgftext[x=5.549207in, y=0.705330in, left, base]{\color{textcolor}\rmfamily\fontsize{10.000000}{12.000000}\selectfont \(\displaystyle {0.50}\)}%
\end{pgfscope}%
\begin{pgfscope}%
\pgfsetbuttcap%
\pgfsetroundjoin%
\definecolor{currentfill}{rgb}{0.000000,0.000000,0.000000}%
\pgfsetfillcolor{currentfill}%
\pgfsetlinewidth{0.803000pt}%
\definecolor{currentstroke}{rgb}{0.000000,0.000000,0.000000}%
\pgfsetstrokecolor{currentstroke}%
\pgfsetdash{}{0pt}%
\pgfsys@defobject{currentmarker}{\pgfqpoint{0.000000in}{0.000000in}}{\pgfqpoint{0.048611in}{0.000000in}}{%
\pgfpathmoveto{\pgfqpoint{0.000000in}{0.000000in}}%
\pgfpathlineto{\pgfqpoint{0.048611in}{0.000000in}}%
\pgfusepath{stroke,fill}%
}%
\begin{pgfscope}%
\pgfsys@transformshift{5.451985in}{1.088614in}%
\pgfsys@useobject{currentmarker}{}%
\end{pgfscope}%
\end{pgfscope}%
\begin{pgfscope}%
\definecolor{textcolor}{rgb}{0.000000,0.000000,0.000000}%
\pgfsetstrokecolor{textcolor}%
\pgfsetfillcolor{textcolor}%
\pgftext[x=5.549207in, y=1.040389in, left, base]{\color{textcolor}\rmfamily\fontsize{10.000000}{12.000000}\selectfont \(\displaystyle {0.75}\)}%
\end{pgfscope}%
\begin{pgfscope}%
\pgfsetbuttcap%
\pgfsetroundjoin%
\definecolor{currentfill}{rgb}{0.000000,0.000000,0.000000}%
\pgfsetfillcolor{currentfill}%
\pgfsetlinewidth{0.803000pt}%
\definecolor{currentstroke}{rgb}{0.000000,0.000000,0.000000}%
\pgfsetstrokecolor{currentstroke}%
\pgfsetdash{}{0pt}%
\pgfsys@defobject{currentmarker}{\pgfqpoint{0.000000in}{0.000000in}}{\pgfqpoint{0.048611in}{0.000000in}}{%
\pgfpathmoveto{\pgfqpoint{0.000000in}{0.000000in}}%
\pgfpathlineto{\pgfqpoint{0.048611in}{0.000000in}}%
\pgfusepath{stroke,fill}%
}%
\begin{pgfscope}%
\pgfsys@transformshift{5.451985in}{1.423672in}%
\pgfsys@useobject{currentmarker}{}%
\end{pgfscope}%
\end{pgfscope}%
\begin{pgfscope}%
\definecolor{textcolor}{rgb}{0.000000,0.000000,0.000000}%
\pgfsetstrokecolor{textcolor}%
\pgfsetfillcolor{textcolor}%
\pgftext[x=5.549207in, y=1.375447in, left, base]{\color{textcolor}\rmfamily\fontsize{10.000000}{12.000000}\selectfont \(\displaystyle {1.00}\)}%
\end{pgfscope}%
\begin{pgfscope}%
\pgfsetbuttcap%
\pgfsetroundjoin%
\definecolor{currentfill}{rgb}{0.000000,0.000000,0.000000}%
\pgfsetfillcolor{currentfill}%
\pgfsetlinewidth{0.803000pt}%
\definecolor{currentstroke}{rgb}{0.000000,0.000000,0.000000}%
\pgfsetstrokecolor{currentstroke}%
\pgfsetdash{}{0pt}%
\pgfsys@defobject{currentmarker}{\pgfqpoint{0.000000in}{0.000000in}}{\pgfqpoint{0.048611in}{0.000000in}}{%
\pgfpathmoveto{\pgfqpoint{0.000000in}{0.000000in}}%
\pgfpathlineto{\pgfqpoint{0.048611in}{0.000000in}}%
\pgfusepath{stroke,fill}%
}%
\begin{pgfscope}%
\pgfsys@transformshift{5.451985in}{1.758731in}%
\pgfsys@useobject{currentmarker}{}%
\end{pgfscope}%
\end{pgfscope}%
\begin{pgfscope}%
\definecolor{textcolor}{rgb}{0.000000,0.000000,0.000000}%
\pgfsetstrokecolor{textcolor}%
\pgfsetfillcolor{textcolor}%
\pgftext[x=5.549207in, y=1.710506in, left, base]{\color{textcolor}\rmfamily\fontsize{10.000000}{12.000000}\selectfont \(\displaystyle {1.25}\)}%
\end{pgfscope}%
\begin{pgfscope}%
\definecolor{textcolor}{rgb}{0.000000,0.000000,0.000000}%
\pgfsetstrokecolor{textcolor}%
\pgfsetfillcolor{textcolor}%
\pgftext[x=5.948128in, y=0.843513in, left, base,rotate=90.000000]{\color{textcolor}\rmfamily\fontsize{10.000000}{12.000000}\selectfont Homoszexuális}%
\end{pgfscope}%
\begin{pgfscope}%
\definecolor{textcolor}{rgb}{0.000000,0.000000,0.000000}%
\pgfsetstrokecolor{textcolor}%
\pgfsetfillcolor{textcolor}%
\pgftext[x=6.100134in, y=0.992819in, left, base,rotate=90.000000]{\color{textcolor}\rmfamily\fontsize{10.000000}{12.000000}\selectfont videó (\%)}%
\end{pgfscope}%
\begin{pgfscope}%
\definecolor{textcolor}{rgb}{0.000000,0.000000,0.000000}%
\pgfsetstrokecolor{textcolor}%
\pgfsetfillcolor{textcolor}%
\pgftext[x=5.451985in,y=1.924230in,right,base]{\color{textcolor}\rmfamily\fontsize{10.000000}{12.000000}\selectfont \(\displaystyle \times{10^{\ensuremath{-}1}}{}\)}%
\end{pgfscope}%
\begin{pgfscope}%
\pgfpathrectangle{\pgfqpoint{0.569136in}{0.693354in}}{\pgfqpoint{4.882849in}{1.189210in}}%
\pgfusepath{clip}%
\pgfsetrectcap%
\pgfsetroundjoin%
\pgfsetlinewidth{1.505625pt}%
\definecolor{currentstroke}{rgb}{0.000000,0.000000,0.000000}%
\pgfsetstrokecolor{currentstroke}%
\pgfsetdash{}{0pt}%
\pgfpathmoveto{\pgfqpoint{1.006867in}{0.747409in}}%
\pgfpathlineto{\pgfqpoint{1.315127in}{1.234595in}}%
\pgfpathlineto{\pgfqpoint{1.623388in}{1.274940in}}%
\pgfpathlineto{\pgfqpoint{1.931648in}{1.506742in}}%
\pgfpathlineto{\pgfqpoint{2.239909in}{1.756806in}}%
\pgfpathlineto{\pgfqpoint{2.548170in}{1.828509in}}%
\pgfpathlineto{\pgfqpoint{2.856430in}{1.782534in}}%
\pgfpathlineto{\pgfqpoint{3.164691in}{1.779565in}}%
\pgfpathlineto{\pgfqpoint{3.472952in}{1.769656in}}%
\pgfpathlineto{\pgfqpoint{3.781212in}{1.680645in}}%
\pgfpathlineto{\pgfqpoint{4.089473in}{1.550296in}}%
\pgfpathlineto{\pgfqpoint{4.397734in}{1.413785in}}%
\pgfpathlineto{\pgfqpoint{4.705994in}{1.274155in}}%
\pgfpathlineto{\pgfqpoint{5.014255in}{1.176488in}}%
\pgfusepath{stroke}%
\end{pgfscope}%
\begin{pgfscope}%
\pgfsetrectcap%
\pgfsetmiterjoin%
\pgfsetlinewidth{0.803000pt}%
\definecolor{currentstroke}{rgb}{0.000000,0.000000,0.000000}%
\pgfsetstrokecolor{currentstroke}%
\pgfsetdash{}{0pt}%
\pgfpathmoveto{\pgfqpoint{0.569136in}{0.693354in}}%
\pgfpathlineto{\pgfqpoint{0.569136in}{1.882564in}}%
\pgfusepath{stroke}%
\end{pgfscope}%
\begin{pgfscope}%
\pgfsetrectcap%
\pgfsetmiterjoin%
\pgfsetlinewidth{0.803000pt}%
\definecolor{currentstroke}{rgb}{0.000000,0.000000,0.000000}%
\pgfsetstrokecolor{currentstroke}%
\pgfsetdash{}{0pt}%
\pgfpathmoveto{\pgfqpoint{5.451985in}{0.693354in}}%
\pgfpathlineto{\pgfqpoint{5.451985in}{1.882564in}}%
\pgfusepath{stroke}%
\end{pgfscope}%
\begin{pgfscope}%
\pgfsetrectcap%
\pgfsetmiterjoin%
\pgfsetlinewidth{0.803000pt}%
\definecolor{currentstroke}{rgb}{0.000000,0.000000,0.000000}%
\pgfsetstrokecolor{currentstroke}%
\pgfsetdash{}{0pt}%
\pgfpathmoveto{\pgfqpoint{0.569136in}{0.693354in}}%
\pgfpathlineto{\pgfqpoint{5.451985in}{0.693354in}}%
\pgfusepath{stroke}%
\end{pgfscope}%
\begin{pgfscope}%
\pgfsetrectcap%
\pgfsetmiterjoin%
\pgfsetlinewidth{0.803000pt}%
\definecolor{currentstroke}{rgb}{0.000000,0.000000,0.000000}%
\pgfsetstrokecolor{currentstroke}%
\pgfsetdash{}{0pt}%
\pgfpathmoveto{\pgfqpoint{0.569136in}{1.882564in}}%
\pgfpathlineto{\pgfqpoint{5.451985in}{1.882564in}}%
\pgfusepath{stroke}%
\end{pgfscope}%
\end{pgfpicture}%
\makeatother%
\endgroup%

    \end{center}
\end{figure}

\section{Összegzés}

A dolgozatom kutatási kérdése az volt, hogy milyen összefüggés tapasztalható Amerikában a pornográf tartalmak fogyasztása, valamint a lakosságon megfigyelhető különböző jelenségek között.

Az elemzésben arra jutottam, hogy az óvszerhasználat fordított kapcsolatban van a pornográf tartalmak iránti kereslettel, tehát minél többen néznek ilyen tartalmakat, annál kevesebben használnak óvszert az együttlét során. Az eredmény megfelel a szakirodalomnak is. Ez az összefüggés szintúgy fordított volt, amikor az amatőr videók fogyasztásának arányát vizsgáltam.

A szexuális zaklatások aránya összefüggött a pornográf tartalmakkal. Ez az arány akkor volt magasabb, amikor a hardcore pornográf videók magasabb arányban voltak fogyasztva, tehát ha a pornográf tartalmak fogyasztói nagyobb részben néztek erőszakos, durva jeleneteket. Ez az eredmény is megfelel a szakirodalomban tárgyaltaknak.

A homoszexualitás aránya és a homoszexuális tartalmak nem voltak egyértelmű kapcsolatban. A dolgozatomban arra következtetek, hogy ennek az az oka, hogy a homoszexuális fogyasztókban lecsökkent a termék iránti kereslet, valamint, hogy a PornHub weboldala nem követi a lakosságban ténylegesen megfigyelhető homoszexualitás mértékét.

A videók és forgalmi adatokból először megbecsültem minden évre az egyes videók nézettségét, majd az elemzés során egyszerű vizualizációs technikákat alkalmaztam, mivel az adatok nem tettek lehetővé az ennél mélyre menőbb módszertanok alkalmazását.

Az alkalmazott módszerek nem alkalmasak oksági kapcsolatok vizsgálatára, csupán különböző jelenségek közötti időbeni kapcsolatot próbálnak meg bemutatni. Továbbá a mért nézettségek és arányok nem veszik figyelembe a mögöttes egyének egyéb tulajdonságait, így például a demográfiai és a szexuális életre vonatkozó jellemzőket sem.

A pornográfia hatásának vizsgálata fontos téma, hiszen az egyre több ilyen tartalmat fogyasztanak az emberek, ami egyértelmű hatással van rájuk. Továbbá egyre hamarabb találkoznak a fiatalok ilyen tartalmakkal és ez a még nem kifejlett szexualitásra nagyobb erővel bírhat. A dolgozat rávilágít a vizsgált téma fontosságára, valamint a különböző aggregált tulajdonságok és a pornográfia kapcsolatára, de azok mélyebb megértéséhez egyén szintű elemzésre van szükség, ami a további kutatásoknak is alapját képezheti.

\pagebreak
\bibliographystyle{apalike2}
\bibliography{references}

\end{document}

















